%  A simple AAU report template.   
%  2014-09-13 v. 1.1.0
%  Copyright 2010-2014 by Jesper Kjær Nielsen <jkn@es.aau.dk>
%
%  This is free software: you can redistribute it and/or modify
%  it under the terms of the GNU General Public License as published by
%  the Free Software Foundation, either version 3 of the License, or
%  (at your option) any later version.
%
%  This is distributed in the hope that it will be useful,
%  but WITHOUT ANY WARRANTY; without even the implied warranty of
%  MERCHANTABILITY or FITNESS FOR A PARTICULAR PURPOSE.  See the
%  GNU General Public License for more details.
%
%  You can find the GNU General Public License at <http://www.gnu.org/licenses/>.
%
%
%
% see, e.g., http://en.wikibooks.org/wiki/LaTeX/Customizing_LaTeX#New_commands
% for more information on how to create macros

%%%%%%%%%%%%%%%%%%%%%%%%%%%%%%%%%%%%%%%%%%%%%%%%
% Macros for the titlepage
%%%%%%%%%%%%%%%%%%%%%%%%%%%%%%%%%%%%%%%%%%%%%%%%
%Creates the aau titlepage
\newcommand{\aautitlepage}[3]{%
  {
    %set up various length
    \ifx\titlepageleftcolumnwidth\undefined
      \newlength{\titlepageleftcolumnwidth}
      \newlength{\titlepagerightcolumnwidth}
    \fi
    \setlength{\titlepageleftcolumnwidth}{0.5\textwidth-\tabcolsep}
    \setlength{\titlepagerightcolumnwidth}{\textwidth-2\tabcolsep-\titlepageleftcolumnwidth}
    %create title page
    \thispagestyle{empty}
    \noindent%
    \begin{tabular}{@{}ll@{}}
      \parbox{\titlepageleftcolumnwidth}{
        \iflanguage{danish}{%
          \includegraphics[width=\titlepageleftcolumnwidth]{figures/aau_logo_da}
        }{%
          \includegraphics[width=\titlepageleftcolumnwidth]{figures/aau_logo_en}
        }
      } &
      \parbox{\titlepagerightcolumnwidth}{\raggedleft\sf\small
        #2
      }\bigskip\\
       #1 &
      \parbox[t]{\titlepagerightcolumnwidth}{%
      \textbf{Abstract:}\bigskip\par
        \fbox{\parbox{\titlepagerightcolumnwidth-2\fboxsep-2\fboxrule}{%
          #3
        }}
      }\\
    \end{tabular}
    \vfill
    \iflanguage{danish}{%
      \noindent{\footnotesize\emph{Rapportens indhold er frit tilgængeligt, men offentliggørelse (med kildeangivelse) må kun ske efter aftale med forfatterne.}}
    }{%
      \noindent{\footnotesize\emph{The content of this report is freely available, but publication (with reference) may only be pursued due to agreement with the author.}}
    }
    \clearpage
  }
}

%Create english project info
\newcommand{\englishprojectinfo}[7]{%
  \parbox[t]{\titlepageleftcolumnwidth}{
    \textbf{Title:}\\ #1\bigskip\par
    \textbf{Theme:}\\ #2\bigskip\par
    \textbf{Project Period:}\\ #3\bigskip\par
    \textbf{Project Group:}\\ #4\bigskip\par
    \textbf{Participants:}\\ #5\bigskip\par
    \textbf{Supervisor:}\\ #6\bigskip\par
    \textbf{Page Numbers:} \pageref{LastPage}\bigskip\par
    \textbf{Date of Completion:}\\ #7
  }
}

%Create danish project info
\newcommand{\danishprojectinfo}[7]{%
  \parbox[t]{\titlepageleftcolumnwidth}{
    \textbf{Titel:}\\ #1\bigskip\par
    \textbf{Tema:}\\ #2\bigskip\par
    \textbf{Projektperiode:}\\ #3\bigskip\par
    \textbf{Projektgruppe:}\\ #4\bigskip\par
    \textbf{Deltager(e):}\\ #5\bigskip\par
    \textbf{Vejleder(e):}\\ #6\bigskip\par
    \textbf{Sidetal:} \pageref{LastPage}\bigskip\par
    \textbf{Afleveringsdato:}\\ #7
  }
}

\newcommand{\name}{\namep\ }

\newcommand{\ph}{Placeholder\ldots}

\newcommand{\say}[1]{\textit{``#1''}} 

\newcommand{\textc}[1]{\texttt{#1}} 

\newcommand{\ttt}[1]{\textc{#1}} 

\newcommand{\textbk}[1]{\textbf{\textit{#1}}} 


\newcommand{\names}{\namep '\
}

\newcommand{\fb}{filter bubble}
\newcommand{\fbp}{filter bubble.}

\newcommand{\namep}{Popping the Informational Bubble}

\newcommand{\titleOf}{Popping the Informational Bubble}
\newcommand{\subtitleOf}{Developing Web-based Systems for the Information Age}

\newcommand{\DonQuote}[1]{
\vspace*{\fill} 
\begin{quote} 
\centering 
#1 
\end{quote}
\vspace*{\fill}
}

\newif\iffirst
\firsttrue

\newif\ifps
\pstrue

\newif\ifret
\rettrue

\newif\ifsourcecheck
\sourcechecktrue

\newif\iftd
\tdtrue

\newif\ifbs
\bstrue

\newcounter{BadSourceCounter}
\newcommand{\BadSource}{
\ifbs
	\label{badsourceRefMacro}
	\global\bsfalse
\else
\fi
\addtocounter{BadSourceCounter}{1}{\textbf{\textcolor{red}{\large{Bad Source}}}}}

\newcounter{ToDoCounter}
\newcommand{\ToDo}{
\iftd
	\label{todoRefMacro}
	\global\tdfalse
\else
\fi
\addtocounter{ToDoCounter}{1}{\textbf{\textcolor{red}{\large{To Do}}}}}

\newcounter{PSCounter}
\newcommand{\Source}{
\ifps
	\label{psRefMacro}
	\global\psfalse
\else
\fi
\addtocounter{PSCounter}{1}{\textcolor{red}{\large{(Source missing)}}}}

\newcounter{FixCounter}
\newcommand{\fix}[2]
{\iffirst
	\label{fixRefMacro}
	\global\firstfalse
\else
\fi
\addtocounter{FixCounter}{1}{\textcolor{red}{#1 }\textcolor{blue}{\large{-
#2}}}}

\newcounter{RetCounter}
\newcommand{\ret}{
\ifret
	\label{retRefMacro}
	\global\retfalse
\else
\fi
\addtocounter{RetCounter}{1} {\textcolor{red}{\large{Rettelse
Mangler }}}}

\newcounter{KTCounter}
\newcommand{\KT}
{
\ifsourcecheck
	\global\sourcecheckfalse
	\label{ktRefMacro}
\else
\fi
\addtocounter{KTCounter}{1}
{\textcolor{purple}{\large{Source needs checking}}}}

\newcommand{\todoCount}{
\LARGE{There are \arabic{ToDoCounter} thing(s) To Do}
 \autopageref{todoRefMacro}\\}

\newcommand{\psCount}{
\LARGE{I dette dokument er der \arabic{PSCounter} steder der mangler
kilder} \autopageref{psRefMacro}\\}

\newcommand{\psCountEn}{
\LARGE{There is missing \arabic{PSCounter} sources} \autopageref{psRefMacro}\\}

\newcommand{\fixCount}{
\LARGE{I dette dokument er der \arabic{FixCounter} steder der skal
fixes} \autopageref{fixRefMacro}\\}

\newcommand{\fixCountEn}{
\LARGE{There are \arabic{FixCounter} places that needs fixing} \autopageref{fixRefMacro}\\}

\newcommand{\retCount}{
\LARGE{I dette dokument er der \arabic{RetCounter} steder der skal
rettes} \autopageref{retRefMacro}\\}

\newcommand{\retCountEn}{
\LARGE{There are \arabic{RetCounter} places that needs to be corrected} \autopageref{retRefMacro}\\}

\newcommand{\KTCount}{
\LARGE{There are \arabic{KTCounter} sources that needs checking}
\autopageref{ktRefMacro}\\}

\newcommand{\BSCount}{
\LARGE{There are \arabic{BadSourceCounter} bad sources}
\autopageref{badsourceRefMacro}\\}

\newcommand{\PFR}{
\ifps
\else
	\psCountEn
\fi
\iffirst
\else
	\fixCountEn
\fi
\ifret
\else
	\retCountEn
\fi
\ifsourcecheck
\else
	\KTCount
\fi
\ifbs
\else
	\BSCount
\fi
\iftd
\else
	\todoCount
\fi
}

\newcommand{\figx}[3][0.5]{
\begin{figure}[H]
\begin{center}
  \includegraphics[keepaspectratio=true,scale=#1]{figures/#2}
  \caption{#3}
  \label{#2}
\end{center}
\end{figure}
}

\newcommand{\figPH}[3][0.5]{
\begin{figure}[H]
\begin{center}
  \includegraphics[keepaspectratio=true,scale=#1]{figures/Placeholder}
  \caption{Placeholder: #3}
  \label{#2}
\end{center}
\end{figure}
}

\newcommand{\figxs}[4][0.5]{
\begin{figure}[H]
\begin{center}
  \includegraphics[keepaspectratio=true,scale=#1]{figures/#2}
  \caption{#3}
  \label{#2}
  \caption*{\textbf{Source:} #4}
\end{center}
\end{figure}
}

\newcounter{Graph}
\renewcommand{\theGraph}{Graph \arabic{Graph}}
\newenvironment{graphEnv}{%
\refstepcounter{Graph}%
}%
{}%
\newfloat{graphFloat}{H}{gop}
\floatname{graphFloat}{Graph}

\newcommand{\graph}[5]{%
\begin{graphEnv}
\begin{graphFloat}
\centering
\begin{tikzpicture}
	\begin{axis}[xlabel={#1}, ylabel={#2}, xmin=0,
	ymin=0, minor tick num = 3, /pgfplots/grid = both, axis lines = left] 
	\forcsvlist{\ap}{#3}
	\end{axis}
\end{tikzpicture}
\caption{#4}
\end{graphFloat}
\label{graph:#5}
\end{graphEnv}
}
\newcommand{\graphNL}[5]{%
\begin{graphEnv}
\begin{graphFloat}
\centering
\begin{tikzpicture}
	\begin{axis}[xlabel={#1}, ylabel={#2}, xmin=0,
	ymin=0, minor tick num = 3,only marks, /pgfplots/grid = both, axis lines =
	left]
	\forcsvlist{\ap}{#3}
	\end{axis}
\end{tikzpicture}
\caption{#4}
\end{graphFloat}
\label{graph:#5}
\end{graphEnv}
}
\newcommand{\dataTable}[3]{%
\begin{table}[H]
\centering
\pgfplotstabletypeset[every head row/.style={before row=\hline,after row=\hline\hline},
every last row/.style={after row=\hline},
every first column/.style={
column type/.add={|}{}},
every column/.style={
column type/.add={}{|}
},string type]{data/#1.dat}
\caption{#2}
\label{table:#3}
\end{table}
}

\newcommand{\graphL}[6]{%
\begin{graphEnv}
\begin{graphFloat}
\centering
\begin{tikzpicture}
	\begin{axis}[xlabel={#1}, ylabel={#2}, xmin=0,
	ymin=0, minor tick num = 3, /pgfplots/grid = both, axis lines = left,legend
	style={ at={(1.3,0.5)}, anchor=center},] 
	\forcsvlist{\ap}{#3}
	\forcsvlist{\al}{#4}
	\end{axis}
\end{tikzpicture}
\caption{#5}
\end{graphFloat}
\label{graph:#6}
\end{graphEnv}
}

\newcommand{\graphLNo}[6]{%
\begin{graphEnv}
\begin{graphFloat}
\centering
\begin{tikzpicture}
	\begin{axis}[xlabel={#1}, ylabel={#2}, xmin=0,
	ymin=0, minor tick num = 3, only marks, /pgfplots/grid = both, axis lines =
	left,legend style={ at={(1.3,0.5)}, anchor=center},] 
	\forcsvlist{\ap}{#3}
	\forcsvlist{\al}{#4}
	\end{axis}
\end{tikzpicture}
\caption{#5}
\end{graphFloat}
\label{graph:#6}
\end{graphEnv}
}
\newcommand{\ap}[1]{\addplot table {data/#1.dat};}
\newcommand{\al}[1]{\addlegendentry{#1}}



\newcounter{def}
\renewcommand{\thedef}{definition \thechapter.\arabic{def}}
\newenvironment{defi}[1]{%
\refstepcounter{def}%
\noindent\textbf{Definition \thechapter.\arabic{def}: #1}%
}%
{}%

\newcommand{\definition}[2]{
\begin{center}
\colorbox{diff}{
\begin{minipage}{0.8\linewidth}
\begin{defi}{#1}

#2\nl

\end{defi}
\end{minipage}\nl}
\end{center}
}

\newcommand{\inputPA}[1]{
\input{sections/Problemanalyse/#1.tex}}
\newcommand{\inputPS}[1]{
\input{sections/Problemloesning/#1.tex}}
\newcommand{\inputS}[1]{
\input{sections/#1.tex}}

\newcounter{req}
\renewcommand{\thereq}{Requirement \thechapter.\arabic{req}}
\newenvironment{requirement}[2][??]{%
\refstepcounter{req}%
\IfStrEq{#1}{??}{}{%
  \label{#1}%
}%
\noindent\textbf{Requirement \thechapter.\arabic{req}: #2}\newline%
}%
{\nl%
}


%%%%%%%%%%%%%%%%%%%%%%%%%%%%%%%%%%%%%%%%%%%%%%%%
% An example environment
%%%%%%%%%%%%%%%%%%%%%%%%%%%%%%%%%%%%%%%%%%%%%%%%
\theoremheaderfont{\normalfont\bfseries}
\theorembodyfont{\normalfont}
\theoremstyle{break}
\def\theoremframecommand{{\color{aaublue!50}\vrule width 5pt \hspace{5pt}}}
\newshadedtheorem{exa}{Example}[chapter]
\newenvironment{example}[1]{%
		\begin{exa}[#1]
}{%
		\end{exa}
}
