% To create an acronym, use: \acro{<acronym>}[<shortname>]{<fullname>}
%   - <shortname> is optional and is for the cases where the short form requires something like inline math mode
%   - Example: \acro{H2O}[$\mathrm{H_2O}$]{water}
% 
% To use the acronym normally, use \ac{<acronym>}
% To specifically use the short or long form, use \acs{<acronym>} and \acl{<acronym>} respectively
% To pluralise the forms, use \acp{<acronym>}, \acsp{<acronym>}, \aclp{<acronym>}
\section{Acronyms Used}
\begin{acronym}[SHA-1]
  \acro{AJAX}{Asynchronous JavaScript And XML}
  \acro{API}{Application Programming Interface}
  \acro{DB}{Database}
  \acro{DBS}{Database System}
  \acro{GUI}{Graphical User Interface}
  \acro{HTTP}{HyperText Transfer Protocol}
  \acro{HTML}{HyperText Markup Language}
  \acro{MVC}{Model--View--Controller}
  \acro{ORM}{Object--Relational Mapper}
  \acro{REST}{REpresentational State Transfer}
  \acro{SHA-1}{Secure Hash Algorithm}
  \acro{SQL}{Structured Query Language}
  \acro{UI}{User Interface}
  \acro{URI}{Uniform Resource Identifier}
  \acro{URL}{Uniform Resource Locator}
\end{acronym}