{\selectlanguage{english} \pdfbookmark[0]{Title page}{label:titlepage_en}
\aautitlepage{%
  \englishprojectinfo{
   \titleOf \\
   \subtitleOf
  }{%
    Internet Technology
	%theme
  }{%
    Fall 2017%project period
  }{%
    SW709E17 % project group
  }{%
    % list of group members 
    Tristan Carl Benjamin Bendixen\\
    Jonas Alberg Ibrahim\\
    Thomas Henrik Stagstrup Krause-Kjær\\
    Mathias Jørgen Bjørnum Leding\\
    Jonathan Nygaard Magnussen\\
    Christoffer Donskov Mouritzen
    
  }{%
    % list of supervisors
	Klaus-Tycho Förster  
  }{%
    \today % date of completion
  }%
}{%department and address
  \textbf{Computer Science}\\
  Aalborg Universitet\\
  \url{http://www.aau.dk}
}{% the abstract

Whenever people make use of search engines like Google, or social media like
Facebook or Twitter, they are likely to find themselves in a filter bubble: a
system of positive reinforcement where they only engage with like minded people,
or are shown products or news which create a positive reaction. This can
cause problems around topics such as politics, because people might be unaware
of the existence of different argument or opinions. As such, This project has
developed a web-application which can determine a user's filter bubble on the
social media site. This is done using a combination of a Naive Bayes Network,
and a sentiment analyzer based around political keywords, and the sharing of
news media. The developed system consists of a database to store analyzed data,
a server back-end to gather information and classify users, and a
web-application front-end.


 }
 }
