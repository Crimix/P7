\section{Data Gathering Conclusion}\label{sec:social-media-conclusion}
Based on the analysis of social media sites and web crawlers, we have decided to
use Twitter, as the short form of tweets is expected to provide many results of
a smaller size, resulting in fewer data to analyse overall. Since the tweets
are short we expect that they are more direct, as Twitter enforces a
strict 280 character limit on each tweet. Therefore, sifting
through the data should require less effort.\nl

Crawlers were decided against for the project, as there are many restrictions on
what it is allowed to access. We have determined that, while it \textit{is}
possible to use a regular crawler to gather data, it is preferable to use the
official Twitter \ac{API} designed for the purpose. We made this decision as the
\ac{API} gives access to all relevant data by using a simple set of \ac{HTTP}
requests, while a crawler has to adhere to a stricter set of limitations.\nl

Based on our choice to use Twitter as our source of data, we can elaborate on
the definition of our target audience made in \autoref{sec:target}, and redefine it
as: \say{Politically active americans, who makes use of the social media site
Twitter.} \nl

Based on the choice to use Twitter as a source of data, \autoref{cha:DA} looks
at different approaches as to how this data could be analysed.

% Burde vi nævne noget med at vi risikere ændringer i API version som fucker os
% over? I forhold til Facebook som blev meget restrictive og måske kunne
% Twitter finde på det samme?