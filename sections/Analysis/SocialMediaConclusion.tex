\section{Data Gathering Conclusion}\label{sec:social-media-conclusion}
After considering the three social media sites Facebook, Reddit, Twitter, and a
web crawler and analysing the possibilities they provide. The decision, as to
the best fit for the project, is based on ease of use, level of access and form
of the data.\nl

We decided to use Twitter, as the short form of tweets is expected to provide
many results of a smaller size resulting in fewer data to analyse overall.
Twitter is also less locked down with a public default setting, making it easier
to access enough information to identify a \fb. Because the tweets are short, we
expect that they are more direct, because Twitter enforces a strict 280
character limit on each tweet. Therefore, sifting through the data should
require less effort.\nl

Crawlers were decided against for the project, as there are many restrictions as
to what it is allowed to access. We have determined that while it is possible to
use a regular crawler to gather data, it is preferable to use the official
Twitter \ac{API} designed for the purpose. We made this decision, as the
\ac{API} gives access to all relevant data by using a simple set of \ac{HTTP}
requests, while a crawler has to adhere to a stricter set of limitations.\nl

Based on the choice to user Twitter as a source of data, \autoref{cha:DA} will
be used to look at different approaches to how this data could be analyzed.

% Burde vi nævne noget med at vi risikere ændringer i API version som fucker os
% over? I forhold til Facebook som blev meget restrictive og måske kunne
% Twitter finde på det samme?