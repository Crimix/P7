\section{Facebook Analysis}\label{sec:facebook-analysis}
In the context of social networks, Facebook is difficult to miss \citep{FacebookPopularity}.
There are billions of users sharing posts and personal details, all of which is useful in the analysis and
detection of a person's \fb\ .\nl

Facebook has different entities that can interact with one another in different ways.
First and foremost, there are the users, which are a central part of the
platform. These are connected via ``friendships''.
A similar entity is ``pages'' which is essentially profiles for companies or
public personalities.
Users can mark these pages as ``liked'' which can be used both to get updates
from the page, as well as to show others, what they like.\nl

Facebook prohibits crawling/scraping of their site with a few exceptions such as
specific search crawlers \citep{FacebookRobotsTxt}.
To access data on their site, they have created the Facebook Graph API as part
of their developer platform, which requires a registration of ones application, along with an accompanying API key used when performing queries.
In addition to this requirement, there is also a strict requirement of consent
from the user of an application, which needs to specify all the different types
of data that the application wants to access \citep{FacebookGraphApiAccessTokens}.\nl

An analysis of Facebook's Graph API \citep{FacebookGraphApiDocumentation} revealed that while the platform had
previously allowed for a lot of freedom in gathering information about a user
and their social circles, this was no longer the case since version 2.0 of the
API \citep{FacebookChangesInGraphTwoPointOh}.
Older versions of the API are phased out over time, as new versions are released, meaning it is not possible to access
the needed information for the initial plan using the Graph
API.\nl

The current API is limited to only listing those of the user's friends that are using the same app and have given
consent to let the app use their details \citep{FacebookChangesInGraphTwoPointOh}.
Access to the user's ``likes'' is limited to the interests listed on their profile, such as their favourite movies and
pages on Facebook that they have marked as ``liked''
\citep{FacebookGraphApiUserLikes} \citep{FacebookGraphApiUserEdges}.\nl

It is possible to access the user's posts and details regarding these, including whether they were originally
posted by the user or shared from another poster.
These shared posts can then be traced back one or more steps depending on the
privacy settings of each link in the ``share--chain'', but only limited
information can be gathered about the users on the chain \citep{FacebookGraphApiUserFeed}.\nl

Based on this information, it is possible to use Facebook's API to find
further connections between existing users of the system.
Knowledge of these users, based on what the system has found out via other means, can then be used to either confirm
or update the system's knowledge of these individuals.
In addition, a lot of news outlets and other companies have pages on Facebook,
which can be used to identify the ideology and political orientation of
users by looking at what the user has marked as ``liked''.
