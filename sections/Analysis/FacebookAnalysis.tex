\section{Facebook Analysis}\label{sec:facebook-analysis}
Facebooks is a social media site with more than one billion users, each
sharing posts and personal details, all of which are useful in the analysis and
detection of a person's \fbp. Facebook is used by both private citizens and
businesses in order to organize events, communicate and share ideas. 

\subsection{Using Facebook}
Facebook has two different entities that can interact with one another in
different ways \Source.

\subsubsection{User}
A user is a private citizen to share information. A core part of the user is
``feed'' which continuously shows new posts from users and pages. Users are able
to connect with other users via ``friendships'', which allows the user to see
some of the posts on their feed.

\subsubsection{Page}
A page can be used to either organize an event, a discussion forum or as a
profile for a company. Users can mark these pages as ``liked'' which can be
used both to get updates from the page, as well as to show others, what they
like

\subsection{Facebook API}
Facebook prohibits crawling/scraping of their site with a few exceptions such as
crawlers made by Google and Apple respectively
\citep{FacebookRobotsTxt}\fix{}{How can a normal non-crawler making reader see
this}.
To access data on their site, they have created the Facebook Graph \ac{API} as
part of their developer platform, which requires a registration of one's
application, along with an accompanying \ac{API} key used when performing
queries.
In addition to this requirement, there is also a strict requirement of consent
from the user of an application, which needs to specify all the different types
of data that the application wants to access
\citep{FacebookGraphApiAccessTokens} \fix{}{Cant seem to find where this strict
thing is defined in the text so\ldots.}\Source .\nl

An analysis of Facebook's Graph \ac{API} \citep{FacebookGraphApiDocumentation}
reveals that the platform previously allowed for more freedom when gathering
information about a user and their social circles. However, this is no longer
the case as of version 2.0 of the \ac{API}
\citep{FacebookChangesInGraphTwoPointOh}.

The current \ac{API} is limited to only listing those of the user's friends
that are using the same app and have given consent to let the app use their
details \citep{FacebookChangesInGraphTwoPointOh}. Access to the user's ``likes''
is limited to the interests listed on their profile, such as their favourite
movies and pages on Facebook, that they have marked as ``liked''
\citep{FacebookGraphApiUserEdges} \citep{FacebookGraphApiUserLikes}
\fix{}{Might be a bit hard for a reader to understand where you got that
from these two sources}.\nl

It is possible to access the user's posts and details regarding them,
including whether they were originally posted by the user or shared from
another poster. These shared posts can then be traced back one or more steps
depending on the privacy settings of each link in the ``share--chain'', but
only limited information can be gathered about the users on the chain
\citep{FacebookGraphApiUserFeed}\fix{}{Please explain to me, where you can
see that about the share--chain}.\nl

Based on this information, it is possible to use Facebook's \ac{API} to find
further connections between existing users of the system.
Knowledge of these users, based on what the system has found out via other
sources, can then be used to either confirm or update the system's knowledge of
these individuals.