\subsection{Twitter Analysis}\label{sec:twitter-analysis}
Twitter is a global social media site with more than 328 million active users
\citep{SocialMediaStats}. Twitter's users
can generally be split into two groups, private citizens and businesses. While businesses might use Twitter for
announcements and marketing, private citizens are more likely to use Twitter for
general communication and to discuss global and local events. In general,
Twitter defines itself as being used for the following purposes
\citep{StartingTwitter}:

\begin{enumerate}    
  \item News and Politics.
  \item Sports.
  \item Pop Culture. 
  \item Influencers.
  \item Utility.
\end{enumerate} %Klaus: A screenshot would be nice that highlights these
% elements

\subsubsection{Using Twitter}
A Twitter user can make use of the following elements \citep{StartingTwitter}
in order to make tweets and communicate with other users:

\begin{itemize}
  \item \textbf{Tweet:} A tweet is a message, which a user can post from their
  account. A tweet can contain 140 characters \citep{StartingTwitter2}
  \fix{}{That is not someting that source says} of text, links, pictures and
  similar media sources, and be seen by all other users. Users who ``follow''
  the account will automatically be presented with the tweet on their Twitter
  feed.
  \item \textbf{Follow:} A user can choose to follow other peoples accounts.
  Then they will get a notification whenever a followed account makes a tweet
  or retweets an existing tweet.
  \item \textbf{Retweet:} Instead of making a tweet, a user can choose to
  forward a tweet made by another account. A user can either retweet the tweet
  as it is, or embed the tweet into a tweet of their own. A user can make
  comments or remarks in relation to the original tweet \fix{}{The last part,
  does the source not say}.
  \item \textbf{Hashtag:} When a user tweets or retweets, they can add a
  ``hashtag'', which can be used to group the tweet, with other tweets using
  the same hashtag. A hashtag is used as a general tag, such as \#Politics or
  \#Sports, or to specify the user's opinion on a subject e.g. \#war \#Idiotic.
  In general, hashtags are used for grouping, searching and filtering 
  \fix{}{Much more than the source say}.
\end{itemize}

% \subsubsection{Tweet}
% A tweet is a message, which a user can post from their account. A tweet can
% contain 140 characters \citep{StartingTwitter2} \fix{}{That is not someting
% that source says} of text, links, pictures and similar media sources, and be
% seen by all other users. Users who ``follow'' the account will automatically be
% presented with the tweet on their Twitter feed.
% 
% Twitter is currently testing tweet lengths of 280 characters, the tweet length
% is changed because 9\% of the English tweets are exactly 140 characters and
% therefore reached the limit. The reasoning being that it is unlikely that all of
% those tweets were not fiddled with to fit the length \citep{TweetL} \fix{}{Is
% that not out now, we are using extended mode for all tweets right now}.
% %Klaus: Aren't they testing 280 char tweets as well for some users?
% 
% \subsubsection{Follow}
% A user can choose to follow other peoples accounts. Then they will get a
% notification whenever a followed account makes a tweet or retweets an existing
% tweet.
% 
% \subsubsection{Retweet}
% Instead of making a tweet, a user can choose to forward a tweet made by another
% account. A user can either retweet the tweet as it is, or embed the tweet into a
% tweet of their own. A user can make comments or remarks in relation to the
% original tweet \fix{}{The last part, does the source not say}.
% 
% \subsubsection{Hashtag}
% When a user tweets or retweets, they can add a ``hashtag'', which can be used to
% group the tweet, with other tweets using the same hashtag. A hashtag is used as
% a general tag, such as \#Politics or \#Sports, or to specify the user's
% opinion on a subject e.g. \#war \#Idiotic. In general, hashtags are used for
% grouping, searching and filtering  \fix{}{Much more than the source say}.

\subsubsection{Twitter \acp{API}} \label{sub:twitterapi}
Twitter permit developers to retrieve data from their users, tweets and
trending hashtags. Twitter has three main \acp{API}, namely the \ac{REST} \ac{API}, the Streaming
\ac{API} and the Ads \ac{API} \citep{TwitterDevDocs}. For the purpose of this
project, the \ac{REST} \ac{API} is the most useful, as it permits reading
Twitter' user data.

\subsubsection{REST API}
The \ac{REST} \ac{API} makes use of \ac{HTTP} requests to allow developers
access to Twitter's data \citep{TwitterREST}. In addition, by using the OAuth protocol
\citep{TwitterOAuth}, Twitter tries to enforce that only authorized applications
can access Twitter's data. All requests must be made using the
\ac{HTTP} protocol. These requests are structured differently based on the
requested data. An example of a \ac{HTTP} request can be seen in
\autoref{httpReq}, where each of the requests parts have been denoted with a
number, and is further described in \autoref{httpElaboration}.

\figx{httpReq}{Structure of Twitter HTTP Request.}

\begin{table}[H] 
\begin{centering}
\begin{tabular}{|l|p{9cm}|l|}
\hline
\textbf{No}&	\textbf{Description}										\\\hline
1			&	Denotes the type of request: GET, POST, DELETE.				\\\hline
2			&	Address for the Twitter \ac{API}.							\\\hline
3			&	Name of the requested resource.	   							\\\hline
4			&	Format of the requested resource.							\\\hline
5			&	Index of the page of the requested data. In case all data can not be
retrieved in a single page.													\\\hline 
6			&	Extra parameters such as the name of the requested user.	\\\hline
7			&	Number of data points requested. Maximum of 5000 per page.	\\\hline
\end{tabular}
\caption{Elaboration on the Twitter HTTP requests.}
\label{httpElaboration}
\end{centering}
\end{table}
