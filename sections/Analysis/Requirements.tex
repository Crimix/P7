\chapter{Requirements Definition}\label{cha:req}
Following the problem statement in \autoref{ch:problem}, this chapter will be
used to determine the actual requirements of the developed software solution.

% Based on the analysis of social media and classification algorithms in the
% previous chapters, this chapter will be used to determine the functional requirements, and
% usability requirements, which will be used to lead the development of the
% system. In \autoref{sec:funcReq} we present the functional requirements, which
% define how the system architecture should be structured, and what features the
% system should contain. In \autoref{sec:userReq} we present the usability
% requirements, which define how the front-end should be designed, and how the
% system should be designed in order to appeal to our target audience.

\section{Functional Requirements}\label{sec:funcReq}

\begin{requirement}[req:twitter-user-retrieval]{Retrieve user and pertinent information from Twitter}
Given a Twitter username, the program should be capable of retrieving
tweets and information about a user, including the accounts they are following.
\end{requirement}

\begin{requirement}{Follow Twitter \ac{API} guideline} 
The program should follow the Twitter \ac{API} guidelines such as not making
more requests than the guidelines specifies per. 15 minutes. This guideline
means that the program should also be capable of keeping track of how many
requests it has used, how many it is allowed to use and when the limit resets. 
\end{requirement}

\begin{requirement}{Identify political beliefs of a user}
The program should be able to identify and inform the user of his or her
political belief and of that of those he or she follows. It should do this by
analysing the words, hashtags and links.

% Saved for a little while 12-12-17
% Using the links and comparing them to the list of bias news site and also taking
% into account the emotional words, the program should be able to identify and
% inform the user of his/her political beliefs. 
% The program should be able to identify hashtags and \acp{URL} in the downloaded
% tweets such that they can be used for bias analysis of the user who posted it or
% the user following the user that posted it.
% The program should be able to use the links identified in the tweets and compare
% them to a list of bias news site which should be created and stored in the
% program. 
% The program should be able to identify and use emotional words to examine a
% users view on different subjects in the tweets.
\end{requirement}

\begin{requirement}{Web-Application Front-End}
The application should make use of a front-end developed as a web application.
This application should be hostable on a server such that it can be presented as
a web page.
\end{requirement}

\begin{requirement}{Server Back-end}
The logic of the developed software should be hosted on a server, which receives
requests from the front-end, and returns analysed data and the determined
conclusions.
\end{requirement}

\begin{requirement}{Store Analysis Data on a Database}
The program should be able to efficiently store the downloaded tweets or similar
relevant data on a database, to reduce the need for retrieving new.
\end{requirement}

\begin{requirement}{Scalable Architecture}
The server back-end must be capable of handling an undefined number of
concurrent user requests as if it acted as the back-end of an actual web page.
As such, the server must queue requests, and make sure to handle them all.
\end{requirement}

\begin{requirement}{Twitter API Resource Limits}
As an individual Twitter user or application has a limited amount of available
calls to the Twitter API, the front-end should make use of the official embedded
Twitter authentication implementation. Doing this should allow the system to use
that user's pool of API calls. This should make the application scalable for a
larger potential user base.
\end{requirement}

\begin{requirement}{\ac{API} Communication}
The developed software must make use of \ac{API}s to communicate between each
other. These \ac{API}s must use verification such that only approved requests
get processed. The database and the server back-end should use these \ac{API}s.
\end{requirement}

%Klaus: Some sort of recommendation would be nice as you want to break out of
% the fiflter bubble. (``popping'') (This is where some sort of clustering
% could come into play, though it could be optional - lets talk live))



\section{User Interaction Requirements}\label{sec:userReq}
In order to improce the user's interaction with our system, the design of the
Graphical User Interface has based on the 3 fundamentals of human-computer
interaction specified by \citeauthor{benyon2013designing} in
\citep[ch.4]{benyon2013designing}, namely accessibility, usability and
acceptability. 

Here, accessibility refers to the ability of all users of all backgrounds, or
with possible disabilities, to be able to interact with the system
\citep[p.77-80]{benyon2013designing}.
Usability refers to the effort people have to spend when using the system, the
ease of learning the system, and the appropriate organization of buttons,
illustrations, and text \citep[p.81-84]{benyon2013designing}. Acceptability
refers to the political, social, and cultural contexts of the system. Namely if
the system is trustworthy, convenient, non-intrusive, or politically correct
\citep[p.84-85]{benyon2013designing}. Based on these elements, the GUI
requirements are presented in \autoref{tab:userReq}.

\begin{table}[H]\centering
\begin{tabular}{|l|p{11cm}|}\hline
\textbf{Category:} & \textbf{Requirement:} \\\hline 
Accessibility  & The system should use a color scheme which is suitable for
people with color blindness.\\\hline 
Accessibility  & When presenting multiple elements side by side (such as data
in graphs), the system should use patterns to distinguish between
elements.\\\hline
Accessibility & Text and images should be resizable, such that people with
limited eye-sight can still use the system.\\\hline
Usability & The system should minimize the amount of user actions necessary in
order to use the system. \\\hline
Usability & The \ac{GUI} should be simplistic upon launch, with every element
immediately recognisable. \\\hline
Usability & There should be no doubts about what information is needed and which
button(s) to press. \\\hline
Usability & The user should have easy access to assisting information on how to
use the system. \\\hline
Usability & When presenting data it should be clearly labeled, and explained in
helping text what the data represents.\\\hline
Acceptability & It should be clearly stated what user data will be used for.\\\hline
Acceptability & The system should use neutral language and imply no political
bias when presenting data in a political context.\\\hline
\end{tabular}
\caption{Requirements for user interaction}
\label{tab:userReq}
\end{table}