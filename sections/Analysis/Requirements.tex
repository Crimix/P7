\section{Functional Requirements}

\begin{requirement}[req:twitter-user-retrieval]{Retrieve user and pertinent information from Twitter}

\end{requirement}

\begin{requirement}{Retrieve which users a user is following}
The program should be capable of retrieving the users a user is
following given a specific user.
\end{requirement}

\begin{requirement}{Retrieve tweets from a user}
The program should be capable of retrieving tweets which a user has posted. The
program should be capable of doing this from a supplied id or username.
\end{requirement}

\begin{requirement}{Follow Twitter \ac{API} guideline}
The program should follow the Twitter \ac{API} guidelines such as not making
more requests than the guidelines specifies per. 15 minutes. This means that the
program should also be capable of keeping track of how many requests it has
used, how many it is allowed to use and when the limit resets.
\end{requirement}

\begin{requirement}{Store / Load tweets}
The program should be able to store the downloaded tweets in a way such that it
is possible to 
\end{requirement}

\begin{requirement}{Identify hashtags / links}

\end{requirement}

\begin{requirement}{Compare links}

\end{requirement}

\begin{requirement}{Identify political beliefs of a user }

\end{requirement}
\begin{itemize}
 \item Be capable of collecting the followers of a specific user from a username
 \item Be capable of collecting tweets from a user
 \item Follow the guidelines and rules of the Twitter API
 \item Be capable of storeing the collected tweets in files such that they can
 be loaded again
 \item Identify hashtags and links in tweets
 \item Compare links from tweets to a built in list of baised news sites
 \item Indentify from the links of tweets the political beliefs of a user
\end{itemize}

\section{Usability Requirements}
