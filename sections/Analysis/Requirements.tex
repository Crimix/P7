\section{Functional Requirements}

\begin{requirement}[req:twitter-user-retrieval]{Retrieve user and pertinent information from Twitter}
The program should be capable of retrieving the users a user is
following given a specific user's username.
\end{requirement}

\begin{requirement}{Retrieve tweets from a user}
The program should be capable of retrieving tweets which a user has posted. The
program should be capable of doing this from a supplied id or username.
\end{requirement}

\begin{requirement}{Follow Twitter \ac{API} guideline}
The program should follow the Twitter \ac{API} guidelines such as not making
more requests than the guidelines specifies per. 15 minutes. This means that the
program should also be capable of keeping track of how many requests it has
used, how many it is allowed to use and when the limit resets.
\end{requirement}

\begin{requirement}{Store / Load tweets}
The program should be able to store the downloaded tweets and store them
efficiently in such a way that it is possible to store a large amount of tweets
and also be able to load them into the program again, such that the data can be
used for further analysis.
\end{requirement}

\begin{requirement}{Identify hashtags / links}
The program should be able to identify hashtags and links in the downloaded
tweets such that they can be used for bias analysis of the user who posted it or
the user following the user that posted it.
\end{requirement}

\begin{requirement}{Compare links}
The program should be able to use the links identified in the tweets and compare
them to a list of bias news site which should be created and stored in the
program. 
\end{requirement}

\begin{requirement}{Emotional  words}
The program should be able to identify and use emotional word to examine a users
view on different subjects in the tweets. 
\end{requirement}

\begin{requirement}{Identify political beliefs of a user }
Using the links and comparing them to the list of bias news site and also taking
into account the emotional words, the program should be able to identify and
inform the user of his/her political beliefs. 
\end{requirement}



\section{Usability Requirements}

\begin{requirement}[req:unnamed-usability]{Userfriendly interface}
\begin{itemize}
  \item The \ac{GUI} should be simplistic upon launch, with every element immediately recognisable.
  \item There should be no doubts about what information is needed and which button(s) to press.
  \item In case of any options being implemented they should be self-explanatory when possible.
  \item When this is not possible there should be a helpful description accompanying the option.
  \item If descriptive texts are necessary they should be easily accessible and it should be obvious how to reach them.
  \item Before the user submits their information there should be no doubts about what the information is used for, how
        it is used, who will have access to it and what type of data they will get in return.
  \item The returned data should be processed and presented in a simple manner that is immediately understood by the
        target user.
\end{itemize}
\end{requirement}