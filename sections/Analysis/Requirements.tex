\chapter{Requirements Definition}\label{cha:req}
Following the problem statement in \autoref{ch:problem}, this chapter will be
used to determine the actual requirements of the developed software solution.

% Klaus: The web aspect seems to be missing? From the study regulations: A
% suitable user interface implies that it is possible to interact with the
% developed solution using a web based interface.
% E.G., an end user application has a web based user interface. An agent or a
% service may not necessarily have a web based user interface, but would usually
% have a web based A further reuirement is that the implementation needs to have
% ascalaable acrhitecture

\section{Functional Requirements}

\begin{requirement}[req:twitter-user-retrieval]{Retrieve user and pertinent information from Twitter}
Given a Twitter username, the program should be capable of retrieving
tweets and information about a user, including the accounts they are following.
\end{requirement}

\begin{requirement}{Follow Twitter \ac{API} guideline} 
The program should follow the Twitter \ac{API} guidelines such as not making
more requests than the guidelines specifies per. 15 minutes. This guideline
means that the program should also be capable of keeping track of how many
requests it has used, how many it is allowed to use and when the limit resets. 
\end{requirement}

\begin{requirement}{Identify political beliefs of a user}
The program should be able to identify and inform the user of his or her
political belief and of that of those he or she follows. It should do this by
analysing the words, hashtags and links.

% Saved for a little while 12-12-17
% Using the links and comparing them to the list of bias news site and also taking
% into account the emotional words, the program should be able to identify and
% inform the user of his/her political beliefs. 
% The program should be able to identify hashtags and \acp{URL} in the downloaded
% tweets such that they can be used for bias analysis of the user who posted it or
% the user following the user that posted it.
% The program should be able to use the links identified in the tweets and compare
% them to a list of bias news site which should be created and stored in the
% program. 
% The program should be able to identify and use emotional words to examine a
% users view on different subjects in the tweets.
\end{requirement}

\begin{requirement}{Web-Application Front-End}
The application should make use of a front-end developed as a web application.
This application should be hostable on a server such that it can be presented as
a web page.
\end{requirement}

\begin{requirement}{Server Back-end}
The logic of the developed software should be hosted on a server, which receives
requests from the front-end, and returns analysed data and the determined
conclusions.
\end{requirement}

\begin{requirement}{Store Analysis Data on a Database}
The program should be able to efficiently store the downloaded tweets or similar
relevant data on a database, to reduce the need for retrieving new.
\end{requirement}

\begin{requirement}{Scalable Architecture}
The server back-end must be capable of handling an undefined number of
concurrent user requests as if it acted as the back-end of an actual web page.
As such, the server must queue requests, and make sure to handle them all.
\end{requirement}

\begin{requirement}{Twitter API Resource Limits}
As an individual Twitter user or application has a limited amount of available
calls to the Twitter API, the front-end should make use of the official embedded
Twitter authentication implementation. Doing this should allow the system to use
that user's pool of API calls. This should make the application scalable for a
larger potential user base.
\end{requirement}

\begin{requirement}{\ac{API} Communication}
The developed software must make use of \ac{API}s to communicate between each
other. These \ac{API}s must use verification such that only approved requests
get processed. The database and the server back-end should use these \ac{API}s.
\end{requirement}

%Klaus: Some sort of recommendation would be nice as you want to break out of
% the fiflter bubble. (``popping'') (This is where some sort of clustering
% could come into play, though it could be optional - lets talk live))



\section{Usability Requirements}
\fix{}{some cite here would be nice - do you follow some recommendations from
the literature?}
\begin{requirement}[req:userfriendly-interface]{Userfriendly interface} 
\begin{itemize}
  \item The \ac{GUI} should be simplistic upon launch, with every element
  immediately recognisable.
  \item There should be no doubts about what information is needed and which
  button(s) to press.
  \item Options should be self-explanatory.
  \item If descriptive texts are necessary, they should be easily accessible,
  and it should be obvious how to reach them.
  \item Before the user submits their information, there should be no doubts
  about what the information is used for, how it is used, who will have access
  to it and what type of data they will get in return.
  \item The returned data should be processed and presented in a simple manner
  that is immediately understood by the target user.
  \item The overall aesthetics of the interface should be visually pleasing and
  utilise meaningful colours.
\end{itemize}
\end{requirement}