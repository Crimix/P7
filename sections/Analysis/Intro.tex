\chapter{Filter Bubble Analysis}
A user's web-activity is an unfathomable amount of data, including data on how
many times, how often, the duration and activity on every page a user visit and
their searches. This activity data gets stored from a users action on any device
the user is logged onto. Aside from storing all the data, it also requires
analysing in order to get anything useful.
There are reports showing that given a fairly active user on social media sites,
then much can be learned by sifting the enormous amount of stored data. That
data can be used to find enough information, to provide a qualified guess at
what the user is voting for, or if there are new movies coming soon they would
enjoy watching \citep{Personality}.\nl

This chapter will be used to describe the structure and data gathering
possibilities of social media sites, particularly Facebook, Twitter and Reddit.
It will then describe what a crawler is. Finally the chapter will be used to
describe how the data can be analysed.
