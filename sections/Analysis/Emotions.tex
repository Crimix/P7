\subsection{Grouping by emotional wording}
Even when limiting the information to social media then it will likely still be
too immense, with a lot of unnecessary information safely omitted. The data can
be sifted through to find the specific special words, which can be used to
identify a users filter bubble \citep{EmotionalWords}. The article shows that
through 563,312 tweets it is possible to identify the filter bubble of an
individual, on a topic, by looking at whom the user follows and what is being
tweeted. Words like ``\#MoreGunControl'' and  ´´\#GunLeadsToMurder'' would
indicate a political stance on that topic. The paper further makes the
observation that the more emotional words there are in a tweet the more useful
the tweet becomes for identifying a bubble. By making connections between
multiple words like this, it is shown that the filter bubble can be identified.
Even with an error margin such as trolls, miss use and takeover of the meaning
of a word. Another issue is reporters paid to be objective when they post, and
their ability to stay neutral is required for them to keep their job, such
personalities should be white-listed in the system since they will likely mess
up accuracy.