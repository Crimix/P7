\chapter{Initial Ideas}
By researching how to break the filter bubble, we have identified two potential
solutions. These sulutions aim to break the filter bubble by analyzing what
information the user is presented with, and categorizing it based on the media
source, and the words the users friends use to describe this information.\nl

In general, both solutions will make use of a neural network in order to
categorize tweets and determine the filter bubble.

\section{Analyze News Media}
The first solution is to analyze tweets and retweets in order to identify all
news articles, and what news media they originate from. These news media would
be classified based on their political bias, and this information will be use to
determine the filter bubble for the given user.\nl

The benefit of this solution is that it should be easy to determine a users
source of news information, and that information already exists about news
sources politiacal biases. In addition, if we can identify a political bias
amound the tweets a user is presented with, it should be possible to break the
filter bubble by presenting information from another point on the political
spectrum.

\section{Analyze Words}
Another solution is to analyze the content of each individual tweet, and
determine any nouns and the adjectives used to describe them. Based on the
positive/negative connotation of these words, we can determine what people think
about a given subject.\nl

The benefit of this solution would be, that it is not limited to using news
sources, but instead can be used on all nouns present in tweets. As such, it can
be used to determine a users filter bubble on any given subject. The problem
with this solution is that it would be difficult to break the filter bubble, as
we cannot easily determine what information to present the user.

\section{Combined Solution}
Another solution would be a combination of the prior two, where we attempt to
determine the bias of a users news sources, and attempt to strengthen our model
by analyzing what users think of the presented news articles. This would
strengthen our model, as we would be able to identify if a retweeted article is
based in a users support or dislike of said article.
