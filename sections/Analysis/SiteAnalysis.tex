A user's web-activity is an unfathomable amount of data, including data on how
many times, how often, the duration and activity on every page a user visit and
their searches. This activity-data gets stored from a user's actions on any
device the user is logged onto. Aside from storing all the data, it also
requires extensive analysing to get anything useful. There is research showing
that given a fairly active user on social media sites, then much can be learned
by sifting the enormous amount of stored data. That data can be used to find
enough information about a user, to provide a qualified guess at what the user
is voting for \citep{Personality}.\nl

Currently, the biggest social media sites with most activity are Facebook and
YouTube \citep{SocialMediaStats}. However, YouTube is largely based on videos
which are a less easy medium to manage, therefore YouTube will not be inspected
further. Other services with a primary focus on private direct messages between
users will also be dismissed. The most suitable opinion based social media for
data gathering are sites with many users openly interacting with each other,
where opinions and information gets provided through text.\nl

This chapter will focus on Facebook with 2,047, Twitter with 328 and Reddit with
1285 million users \citep{FacebookPopularity}. It will describe how the each of
the site's users interacts as well what the API can do. Finally, we will
conclude on which social media we will use.
