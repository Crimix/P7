%Klaus: it would be nice to provide an overview to this chapter either here or
% on analisis part (Har smidt den efter initerende problem)

To identify a user's filter bubble, we need access and analyse a user's online
activity, and determine what people and informational sources they engage. This
web-activity consists of an immense amount of information, including data on how
many times and how often, a user visits a page, and which people they follow or
in other ways engage with on social media. This activity data get stored on the
pages the user frequents, in the case of social sites, or locally on the user's
device, in the form of search and browsing history. While the user data contains
a lot of both useful and irrelevant information, the vast amount of user data
makes it require extensive analysing to filter and combine it to deduce
something about a user.\nl

This chapter analyses different social media sites, suited to gather user
activity to break the filter bubble. Some social media sites inspected and the
best fit will the one used for the project, at which point we examine how to
gather the user data from the web page, and conclude upon the chapter.


\section{Social Media}
This section will focus on Facebook with 2061 million, Twitter with 328 million
and Reddit with 1575 million active users \citep{SocialMediaStats,
AdvertiseOnReddit}. It will examine how each of the site's users interacts, and
the options they provide regarding data-gathering on their website. In the end,
a conclusion on the chosen media sites derived from the inspected sources.

Currently, the most prominent social media sites with the highest number of
activities are Facebook and YouTube \citep{SocialMediaStats}. However, YouTube
is used to share videos which are a different medium than text to manage;
therefore YouTube will not be inspected further. Services with a primary focus
on direct messages between users are omitted. Thus the most suitable
opinion-based social media sites to gather text from is websites with many users
openly interacting with each other, where opinions and information get exchanged
through posts and comments.\nl

If given reasonably active users on social media sites, then much can be learned
by sifting the enormous amount of data. Through an analysis of tweets and
combining hashtags such as ”\#ImWithHer” or ”\#TrumpTrain” with
”\#election2016”, finding multiple of these gives an idea about the user's
political view. Thus with enough user data being analysed a qualified guess
about a users opinion on a subject can be derived.\nl
