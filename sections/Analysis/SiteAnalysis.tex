%Klaus: it would be nice to provide an overview to this chapter either here or
% on analisis part (Har smidt den efter initerende problem)

To identify a user's filter bubble, we need access and analyze a user's online
activity, and determine what people and informational sources they engage. This
web-activity consists of an immense amount of information, including data on how
many times and how often, a user visits a page, and which people they follow or
in other ways engage with on social media. This activity data get stored on the
pages the user frequents, in the case of social sites, or locally on the user's
device, in the form of search and browsing history. While the user data contains
a lot of both useful and irrelevant information, the vast amount of user data
makes it require extensive analyzing to filter and combine it to deduce
something about a user.\nl

This chapter analyzes different social media sites, suited to gather user
activity to break the filter bubble. Some social media sites inspected and the
best fit will the one used for the project, at which point we examine how to
gather the user data from the web page. Finally an examination on how to analyze
and use the user data, and conclude upon the chapter.


\section{Social Media}
This section will focus on Facebook with 2061 million, Twitter with 328 million
and Reddit with 1575 million active users \citep{SocialMediaStats,
AdvertiseOnReddit}.
It will examine how each of the site's users interacts, and what options they
provide regarding data-gathering on their site. In the end, a conclusion on the
chosen media sites will be derived from the inspected sources.

Currently, the biggest social media sites with most activity are Facebook and
YouTube \citep{SocialMediaStats}. However, YouTube is based on videos which are
a different medium to manage, therefore YouTube will not be inspected further.
Services with a primary focus on direct messages between users will also be
dismissed. The most suitable opinion-based social media for data gathering, are
sites with many users openly interacting with each other, where opinions and
information gets provided through text.\nl

If given fairly active users on social media sites, then much can be learned by
sifting the enormous amount of data. By analyzing hashtags and combine
”\#ImWithHer” or ”\#TrumpTrain” with ”\#election2016” which gives an idea about
a users political view. With enough user data being analysed much about a user
can be found, perhaps enough to provide a qualified guess about a user's
political point of view.\nl

\fix{}{already mention outcome}


