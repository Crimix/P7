%Claus: it would be nice to provide an overview to this chapter either here or
% on analisis part

To identify a user's filter bubble, we need to analyze their online activity, in
order to determine what people and informational sources they engage with. This
web-activity consists of a large amount of information, including data on how
many times and how often, a user visits a page, and which people they follow or
in other ways engage with on social media. This activity data get stored on the
pages the user frequents, in the case of social sites, or locally on the device
the user is logged onto in the form of search and browsing history. While this
data contains a lot of information, the raw nature of it makes it require
extensive analysing to get useful information.\nl

Research shows, that if given a fairly active user on social media sites, then
much can be learned by sifting the enormous amount of data. With this, enough
information about a user can be found to provide a qualified guess about the
user's political point of view \citep{Personality} \KT.\nl

Currently, the biggest social media sites with most activity are Facebook and
YouTube \citep{SocialMediaStats}. However, YouTube is based on videos which are
a different medium to manage, therefore YouTube will not be inspected further.
Services with a primary focus on direct messages between users will also be
dismissed. The most suitable opinion-based social media for data gathering, are
sites with many users openly interacting with each other, where opinions and
information gets provided through text.\nl

This chapter will focus on Facebook with 2,047, Twitter with 328 and Reddit with
1285 million activ users \citep{FacebookPopularity}. It will examine how each of
the site's users interacts, and what options they provide regarding data-gathiering
on their site.

%Claus: And a conclusion section?
