\section*{Week 40}
\subsection*{What Happend This Week}
This week we finished the initial problem and some parts of the analysis which
is also starting to be near a finishing point. We are now able to retrieve 3200
twwets from each account a given user is following, that means if a user follows
five Twitter accounts then we will find 16000 asuming each account have 3200
tweets or more.
We have spend some time thinking further upon how to use this vast amount of
userdata, and are currently considering focusing on ``emotional wording'' such
as which hashtags goes together and with what, and if it is a positive or
negative tweet.

\subsection*{Evaluation}
The progress on the applications is progressing nicely, but it is needed to
further brainstorm upon how to use the data. We have told our suporvisor
that we do not expect much to happen duo to deliveries in other modules
this week. We are also starting to wonder if we fullfill the requirements for
the semester.

\subsection*{Next Week}
We will mainly perform edits and improve upon the code, because there are a huge
task in Programming Paradigms then we do not expect much more progress.




% Mail
% Our Progress:
% Report: This week we have finished the following sections: Twitter Analysis,
% Initiating Problem and minor corrections (The acronyms in parentheses are marked
% because we haven’t added them properly yet).
% There are still sections in progress, not yet in the master.
% Code: We have managed to retrieve 3200 tweets for each account a given user is
% following. That is, if you follow "denmarkdotdk" and "VisitDenmark", we retrieve
% the 3200 most recent tweets from both of these accounts. We can store this data
% locally, and determine which words and hashtags in the tweets. Now we need to
% determine how we want to analyze the data, and the model we want to use to
% detect the "filter bubble". (We have mostly been talking about a neural network
% which attempts to classify the user's stance on a given subject by using words
% such as "stupid", "amazing", "idiot" etc.) Twitter seems like the best options
% from our current research.
% 
% The agenda for the meeting:
% - It's usually a requirement for us to bring something from our courses, how is
% it this semester? - Feedback

%  regarding the report: I only have minor comments, but rather a major comment
% outside the pdf: in the final report, there will be the scientific part and
% the reflection part (which is also scientific somehow .. ). However, it seems
% to me that in the report, there is also a sort of progress report included a
% bit, which won't be part of the final report. Let's talk about this tomorrow.
%  the code part: sounds exciting. Looking forward to hear more tomorrow.