\subsection{Week 39}
\subsubsection{What Happened This Week}
We have begun defining what we expect to be in the report and set a couple of
deadlines. This includes finish writing how we implemented the software by the
25th of November. Our goal is to leave us plenty of time to finish writing
tests and performing a user evaluation.\nl

We have found that Twitter \ac{API} is fairly open and easy to use. There is a
limit to the number of requests we can make, but it seems to be reasonable. We
have managed to build a small program using Twitters API which allows us to
retrieve tweets. We have also decided to focus the project on finding political
filter bubbles.

\subsubsection{Reflection}
Web crawlers seem to be very limited in what they can access from social media
sites, especially as the APIs seem to do it easier. It would have
benefitted us if we at this point had finished writing about the social media
sites. Especially, as the workflow is slower than we would have liked. This is
probably due to the week being full, with only Monday being open for everyone.
We should have drawn more on the knowledge from prior semesters as we have been
told that we need to be more direct and concise in our report.\nl

In an attempt to get more done each week, we will try with another rule, stating
that a member meets every day at 9:00 if there is no lecture for that person.
This should probably have been organised earlier.

\subsubsection{Next Week}
Next week, we want to finish the analysis part of the report and be able to pull
some data from Twitter with the Twitter \ac{API}.
	


% Our Progress:
%  We have had a busy schedule this week so we don't have much progress to show.
% However, we do have 2 chapters ready for you to read.
%  We have researched further on basic crawlers since it works well with one of
% our courses as well.
%    The agenda for the meeting:
%  1. Feedback on chapters.
%  2. We want to help an individual break out of the filter bubble. The agent
% checks the user's web activity and provides information from outside that
% bubble (how much and what is not finalized yet).

% Research Social medial Data gathering Web agent?  
% Problem Formulation 
% GUI design  
% Implementation (done by 25. nov)  Test User tests Evaluation of
% the UI Other tests?  Evaluation  Rapport:
% Introduction seems good.
% Chapter 2 - Should be more distinctly separated, between formal and
% reflection.
% Be more concrete about everything, less would, should…  Discussion:
% Twitter has good possibilities, as we can access a lot of data through their
% API, though we can only make a limited amount of calls every 15 min.
%  Thoughts on UI:
% What should the user be shown? Can we find enough info to break the bubble We
% will need to briefly describe the process of designing a GUI but we do not
% need to describe the whole process of iterating through the designs.
%  Reflections:
% For the process we can describe some of the difficulties of having all the
% different combination of study. As we need to be better to manage our time. We
% should add our experiences with researching facebook, twitter and reddit and
% why we chose what we chose.