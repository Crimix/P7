\subsection{Week 48} 
\subsubsection{What Happened This Week}
During this week we had a meeting with the Dr. Stefan Schmid again, this time
we had more to show. During the meeting, we talked about the different parts of the
project, and how we plan on putting them together. Unfortunately, we could not
show all of it together, because some of the parts still need more work.
During this week we had a social work night with food, during this our Naive
Bayes model finally got functional the Bag-of-Words model was optimised a lot a,
going from over 2 hours to 6 minutes to a couple of seconds. There were some
problems fixed with the Queue Server \ac{API}, which caused some problems for
the querying of workers. We are currently behind schedule which means we have to
work harder to catch up where we want to be at.

\subsubsection{Reflection}
We are once again very happy with the results of our meeting with Dr. Stefan
Schmid, and it is something we want to do in future projects as well. The work
night we had was not as productive for everyone as hoped, since we spent a
lot of time on cooking food. However, progress still happened, morale got
boosted, and it was after usual work hours, which meant the work was a
bonus overall on that week. It is definitely something we will do again this
semester, perhaps for editing. 

\subsubsection{Next Week}
Next week we want to finish the remaining code parts, stitch them together and
finish the documentation. We want to reach a state where we can perform user
evaluation for our solution, or at least get one planned for the week after.


% Implemented new features such as the possibility to analyse tweets while
% retrieving them.
% Fixed some problems with the code, such as the OAuthHelper, which did not work.
% Now it is possible to make request on behalf of a user.
% The algorithm has been optimised further. Now it only takes about 1 second to
% process 80000 tweets. (Down from 2 hours at the start of the week!) Started to
% comment the code in the different projects.
% Documented parts of the code.
% Fixed Started documenting the algorithm in the report. (So far we have written
% about the general idea behind it, and the sentiment analysis).
% 
% 
% Fixed problems with querying REST APIs from the GUI.
% 
% Fixed the various problems with the Naive Bayes, it now has an accuracy of
% around 60\%, and added merged it with the main program. Cleaned the
% TextProcessorLib library of unused methods.
% Identified problems with the returned data, leading to an overhaul of the
% calculated values and how they’re stored in the DB backend.
% 
% Mention why it only works on english (future words - language detection is a field)
% 
% ML:
% Consider using PCA - printable component library
% To look at the components
% 			Mention related works in the ML sections
% 
% Algorithm:
% Where’s the bottleneck?
% How accurate is it? - needs to be tested
% Report that we have made improvement from 2 hours to 6 minutes
% 
% Program:
% Benchmark it - find the bottleneck, and maybe comment the user
% 
% 
% Are there any related works to finding opposite opinions
% 
% We have a bit more work with privacy
% 
% Thomas has an idea for contacts in america we can ask for user input.
% 
% For the repport:
% Why is sentiment analysis important to know?
% Add source and explain what if any we have added to feature extraction
% 
% 		(Though Klaus does not think it’s suited to reddit)
% For exam - 	ask reddit to use it for a bit - could it be a weapon?
% 		Add a disclaimer?
% 	Borrow machines in the basement
% 	      	Could use review “quotes” from users