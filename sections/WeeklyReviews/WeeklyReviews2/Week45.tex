\subsection{Week 45}
\subsubsection{What Happened This Week}
We managed to make most of the individual parts work, though they are currently
not able to act together. We have a simple mockup \ac{GUI}. The \ac{API}
for the queue server has been created in C\# such that it is possible to create
jobs given only a Twitter account name and an \ac{API} key. We have also begun
to devote slightly more time to write the report. An issue we have encountered when
we tried using a Neural Network is that if a feature vector contains more than
about 100 features, the network use up all available memory and crash. This is a
problem as we estimate that a simple training set should have about 1000. We
have also looked at other models such as Naive Bayes, Bag-of-Words and Support
Vector Machines.

\subsubsection{Reflection} 
In hindsight, it might have been better to focus on the simpler models such as
Naive Bayes and Bag-of-Words, as the more advanced models, such as Neural
Networks, have problems with too many features. 

\subsubsection{Next Week}
Next week, we want to continue the development of the solution and begin
describing it in the report.

% The primary aspects of the GUI has been completed. It can initiate a new
% request, asking the user for permission to use their account for requests, as
% well as asking for the Twitter account they wish to check and the
% email-address they would like to receive notifications on when the work has
% been completed. The requests are added to a queue for asynchronous
% transmission of details to the QueueServer.
%  The API for the QueueServer has been created in C# such that it is possible
% to create jobs given just a twitter name and API key. In the report the
% implementation section now contains documentation about the Database API and
% the new API, the new API is not that well documented yet.
% It is now possible to add requests to the queue in the form of tasks, which
% are executed in parallel.
%  This week we have continued using libraries to create a neural network
% several issues have appeared as the we tried the network on a data sample of
% reviews.
% Currently we are using a sparse feature vector, each feature is represented by
% a position in a array, this means that as the dictionary grows, so too does
% the feature vector. This represents a problem as on a sample of 5000 reviews
% provides more than 20000 features. With the current library we are using we
% quickly run out of memory as the features grow, limit seems to be around 150
% features.
% We have tried shifting to using Naive Bayes and Support Vector Machines, which
% allows us to have a larger dictionary before we run out of memory.
%  A couple of questions regarding machine learning and models:
% Do you know who we need to ask for further information use of a server
% cluster? Do you have any experience with Dense Feature Vectors? Do you any
% possible solutions/ideas?   Meeting:
% Spent some time getting Klaus op to speed.
%  Using a web gui, which communicates with a queue server, which forwards work
% to a worker.
% The worker does the number crunching.
%  The database server uses a rest interface, it connects the workers and the
% queue server.
%  We have two different approaches:
% Use a Machine Learning model:
% Neural Network is highly limited as it runs out of memory if there’s more than
% 150 inputs Naive Bayes runs out of memory at around 70.000 different words
% Support Vector Machine is slow.
% Use an Algorithm:
% Use links, keywords and hashtags which each has a value Different parts have
% different values Needs to be optimized as it feels slow  Simple GUI which
% allows for connections with Twitter, it does not have any connections with the
% other parts currently.
%  Need to contact Stefan regarding the server.
