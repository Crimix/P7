\section{Reflections on the Second Half}\label{sec:SecRef}
Before the meeting with Stefan Schmid, we had been spending a lot of work
fixated on figuring out how to interact with Twitter's API and developing a simple
implementation to do this automatically. A problem though, was that we had not
really considered what we wanted to do with the data.\nl

As such, following the meeting, we realized the scope of creating a
web-application capable of doing everything we wanted. While we recognized the
size of the task, it provided us with a clear goal to work towards, which
allowed us to easier plan what to do and to split the workload between us.\nl  

Following our meeting with Stefan, we also chose to focus on developing two
different approaches to classifying users political leanings. While this
increased the necessary workload, we still found it useful, as we could compare
the two models, and choose the one which provided the best results. It also
gave us the opportunity to fall back onto one of the models, if one did not work
as well as intended.\nl

Following our second meeting with Stefan Schmid, we got a lot of good feedback
on our classification models, and critisism of the fact that our Bag-of-Words
model had a runtime of 6 minutes (Down from 2 hours). Following this critisism,
we reviewed the system, and found a major mistake (using the wrong way to acces
elements of dictionaries), and reduced the runtime to less than 1 second. This
was especially nice, as we had just accepted 6 minute runtime as being better
than 2 hours, but never stopped to do the math, and realized that 6 minutes was
still very unreasonable for the work it was doing. As such, we have learned to
be more critical when reviewing our systems.\nl

Finally, we also tried having a working in a social context, where we went home
to one of the group members, and made food together. While the time spent was
slightly less productive, it provided us with an amazing opportunity for
strengthening our social bonds, and discussing the project and our work in a
less professional setting.