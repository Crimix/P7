\subsection{Week 43}\label{Week43}
\subsubsection{What Happened This Week}
Based on the meeting with Dr. Stefan Schmid and the following meeting with our
supervisor, we have decided to make a sizable change to the project. We have
decided on the overall architecture, this can be seen in \autoref{ch:sysview}.
We have begun working on all the different parts in groups of 1 to 2 people,
with the goal of researching what is possible and how to do it.



\subsubsection{Reflection}
During the presentation Monday we received a lot of feedback on everything,
especially in terms of how to show it to the user and how to analyse the data.
The meeting has been a very useful way of increasing ambition and understanding
some of the parts that were a little bit vague and undefined. It certainly
brought forth a lot of new ideas. We would definitely recommend doing something
similar again in future projects.

\subsubsection{Next Week}
Next week we want to continue with the research and development we have done
this week. We will also need to make some considerations to what our target
audience is.


% All corrections on the report applied and a further grammar check on new
% stuff, introductions and conclusions are also added to the chapters.
%  After the meeting Monday, we discussed how to expand the program we have
% written to be a web application, as such we decided that we need a gui web
% application in Laravel which connects to a queue server in c# through a rest
% api, then the queue server should start workers which are a modified version
% of our first program. The gui and workers communicate with a database though
% another rest api. The rest of the week was spent on trying first to make the
% database rest api in c# but it was then after many problems it was decided
% that the database rest api should be made in a separate Laravel project. Each
% of the for a mentioned items has been worked on during the week.
%   ML-stuff In regards to classifying tweets the week has been spent
% investigating different types of ML, especially Support Vector Machines, Long
% Short Term Memory and basic Neural Networks, which have shown to be promising.
% There has been some thought regarding using different methods for sentiment
% analysis, such as bag of words vs NNs.
%  There has also been devoted time to figuring out whether we should write the
% ML part in python, as it has a lot of good libraries and documentation, or C#
% which has a what appears to be well regarded libraries and somewhat less
% documentation.
%  There has been work on figuring out how to transform tweets into something
% useful, and so far a tweet parser has been built, which transform the tweet
% from a string to a list of string (tokens).
% There has been thought about further transforming the list to a vector by
% mapping words with unique integers, which should be useful when used by a NN.
%  We have considered options for different methods of training, and arrived
% that we are going use the supervised method, though that means we will have to
% provide a lot of tweets which we will have to classify regarding political
% leaning.
%  Meeting:
% Presentation:
% Click on spectrum. Which friends are where.
% Tune what kind og people you want suggested. Very different, slightly
% different from you? Suggest people with a minimum of intellect (University
% degree? Known speaker?) Lookup existing recommendation systems. Current
% systems, reports, solutions? ADs API can maybe group people a bit? Maybe look
% at hashtags? What is our end goal? A plugin? Standalone application? Take
% additional input from the user. What do they think at the moment? What other
% information to supply? Accuracy can partly be determined by user feedback. Do
% you agree with our conclusion? Lookup study about what which hashtags imply
% about a political stance/leaning.
% Look at related work to ours.
% What is the complexity of our algorithm.
% Store information for lookup later.
% Have two approaches system based/neural network. How do they compare? If we
% want to do preprocessing/plugin/(neural network?) we can ask for access to
% (something? computer??)  Meeting:
% Problem with information about reddit user numbers.
% We could "potentially" just write about why our application would be easy to
% scale, and how we would do it  Should we find people to test the application?
% - Maybe, if we can.
%  How do we expand our application:
% - How would we deal with increased popularity - We should probably do
% something with a browser (plugin/site/server/etc.)  What to do:
% - User interface (Laravel) - Describe Architecture MVC front-end. Unknown on
% back-end - Algorithm which uses emotional words, keywords, and hashtags - If
% people log in using their own account, they get an amount of requests for
% themselves - Convert system to run on a server - Definitely do something to
% consider scalability e.g. multiple servers, or using users own requests -
% Begin writing reflections (what should we have done? What did we do? The
% plan?) - Plan how to systematically evaluate our application. User interface,
% testers, usability, functionality (does it work as intended).
% - Look at lecture notes on how to design interfaces. Then do what the slides
% tell us to do.
% - Expand the webcrawler chapter with more info on the different parts of it,
% and in the end a conclusion with the reasonings why we won’t use it (and
% causes like spoofing and a very naughty crawler) - Lookup existing
% recommendation systems. Current systems, reports, solutions?
