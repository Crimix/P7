\section*{Week 39}
\subsection*{What Happend This Week}
We created an idea for the structure of the report, and what we wanted in the
chapters. We also set the date 25. November, as the day we would like to be done
with programming. The introduction part is finished, the other analysis chapter
needs to be more coherent.
We have also found that the Twitter \ac{API} gives us access to a vast amount of
user data, although they limit the number of calls we can make every 15 minutes, it
should not cause us any problems. We have made some considerations on the
\ac{UI} of our application. We are starting to consider how much user data we need to break
a filter bubble and if we can get from Twitter.
Our current idea for the semester goal: "We want to help an individual break out
of the filter bubble. The agent checks the user's web activity and provides
information from outside that bubble (how much and what is not finalized yet)."


\subsection*{Reflection}
It was a setback to find out how difficult it was to
get access to social media site's user data through a web crawler. We are now
investigating if there is a way to still use our crawler, we should definitely
have investigated further upon how we could use a crawler instead of its
development.
The Twitter \ac{API} does seem like a suitable option to gather enough user data
about a person, sources explaining that tweets can be used to break the bubble
have been found. Workflow is still a little slower than we liked, we think one
of the problems is that the group only have a full day of work on Mondays. We
learned that we need to be more direct and concise in our report and used too
loose terms such as many "could", "would" and "should".

\subsection*{Next Week}
In an attempt to get more done each week we will try
with a rule, stating that a member meets every day at 9:00 if there is no
lecture for that group member. Next week we want to finish the analysis part of
the report and be able to pull some data from Twitter with the Twitter \ac{API}.
	


% Our Progress:
%  We have had a busy schedule this week so we don't have much progress to show.
% However, we do have 2 chapters ready for you to read.
%  We have researched further on basic crawlers since it works well with one of
% our courses as well.
%    The agenda for the meeting:
%  1. Feedback on chapters.
%  2. We want to help an individual break out of the filter bubble. The agent
% checks the user's web activity and provides information from outside that
% bubble (how much and what is not finalized yet).

% Research Social medial Data gathering Web agent?  
% Problem Formulation 
% GUI design  
% Implementation (done by 25. nov)  Test User tests Evaluation of
% the UI Other tests?  Evaluation  Rapport:
% Introduction seems good.
% Chapter 2 - Should be more distinctly separated, between formal and
% reflection.
% Be more concrete about everything, less would, should…  Discussion:
% Twitter has good possibilities, as we can access a lot of data through their
% API, though we can only make a limited amount of calls every 15 min.
%  Thoughts on UI:
% What should the user be shown? Can we find enough info to break the bubble We
% will need to briefly describe the process of designing a GUI but we do not
% need to describe the whole process of iterating through the designs.
%  Reflections:
% For the process we can describe some of the difficulties of having all the
% different combination of study. As we need to be better to manage our time. We
% should add our experiences with researching facebook, twitter and reddit and
% why we chose what we chose.