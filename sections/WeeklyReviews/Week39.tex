\section*{Week 39}
\subsection*{What Happend This Week}
We created an idea for the structure of the report, and what we wanted in the
chapters. We also set the date 25. November, as the day we would like
to be done with programming. The introduction part is finished, the other
analysis chapter need to more coherent. We have also found that thte Twitter API
gives us access to a vast amount of userdata, althouth they limit the amount of
calls we can make every 15 minutes. We have also made some thoughts on the UI of
our application, and if we can get enough informaion to break the filter
bubble.

\subsection*{Evaluation}
It was a setback to find how difficult it was to get access to social medea
sites's userdata through a web crawler. We are now investigating if there is a
way to still use it. However, Twitter seems like a good choice, there were
also sources explaining that tweets can be used to break the bubble.
Workflow is still a little slower than we liked, and there are not as much being
finished as expected.

\subsection*{Next Week}
We are going to try setting a new rule with people meeting everyday at 9:00 if
theres no lectures, since it is only monday which is completely free of
lectures. We hope that this will enabel us to get some more work done.
Next week we want to finish the analysis part of the report, and be able to pull
some data from Twitter.



% Our Progress:
%  We have had a busy schedule this week so we don't have much progress to show.
% However, we do have 2 chapters ready for you to read.
%  We have researched further on basic crawlers since it works well with one of
% our courses as well.
%    The agenda for the meeting:
%  1. Feedback on chapters.
%  2. We want to help an individual break out of the filter bubble. The agent
% checks the user's web activity and provides information from outside that
% bubble (how much and what is not finalized yet).

% Research Social medial Data gathering Web agent?  
% Problem Formulation 
% GUI design  
% Implementation (done by 25. nov)  Test User tests Evaluation of
% the UI Other tests?  Evaluation  Rapport:
% Introduction seems good.
% Chapter 2 - Should be more distinctly separated, between formal and
% reflection.
% Be more concrete about everything, less would, should…  Discussion:
% Twitter has good possibilities, as we can access a lot of data through their
% API, though we can only make a limited amount of calls every 15 min.
%  Thoughts on UI:
% What should the user be shown? Can we find enough info to break the bubble We
% will need to briefly describe the process of designing a GUI but we do not
% need to describe the whole process of iterating through the designs.
%  Reflections:
% For the process we can describe some of the difficulties of having all the
% different combination of study. As we need to be better to manage our time. We
% should add our experiences with researching facebook, twitter and reddit and
% why we chose what we chose.