\section*{Week 48} 
\subsection*{What Happend This Week}
During this week we had a meeting with the expert Stefan again, this time we had
more to show. During the meeting we talked about the different parts of the
project, and how we plan on putting them together. Unfortunately we could not
show all of it toogether, since it was still being worked upon. A reflection
upon the meeting can be found in \fix{\ref{}}{kilde til reflection på møde}.
During this week we also had a social work night with food, during this our
Naive Bayes method got functional for identifying tweets. Our algorthm got
optimized even furter and it now takes just one second, down from two hours at
the beginning of the week. There were some problems fixed with the \ac{GUI}'s
\ac{REST} \ac{API}, which have been fixed for querying the workers.

\subsection*{Reflection}
We are once again very happy with the results from our meeting with Stefan, and
it is something we want to do in future projects as well. The work night we had
was not as productive for everyone as hoped, since some made food. However,
progress still happend and morale was boosted, and it was after normal hours at
uni which meant the work was still a bonus overall on that week. It is definetly
something we will do again this semester, perhaps for editing. Our combined area
of knowledge could be a little btter which is something we figured with the
meeting, all of a sudden we had much more to do, which meant that knowledge of
what is happening gets shared less. That problem is something we need to be
aware of and attempt to fix, for now we will try with weekly reflections
on our individual work. 

\subsection*{Next Week}
Next week we want to finish the code parts, stich themm together and get some of
hte documentation done. It would also be amazing to get to a state where we can
perform usertest for our solution, or atleast start trying to get contact with
some candidates from the usergroup, which is described in \autoref{sec:target}.



% Implemented new features such as the possibility to analyse tweets while
% retrieving them.
% Fixed some problems with the code, such as the OAuthHelper, which did not work.
% Now it is possible to make request on behalf of a user.
% The algorithm has been optimised further. Now it only takes about 1 second to
% process 80000 tweets. (Down from 2 hours at the start of the week!) Started to
% comment the code in the different projects.
% Documented parts of the code.
% Fixed Started documenting the algorithm in the report. (So far we have written
% about the general idea behind it, and the sentiment analysis).
% 
% 
% Fixed problems with querying REST APIs from the GUI.
% 
% Fixed the various problems with the Naive Bayes, it now has an accuracy of
% around 60\%, and added merged it with the main program. Cleaned the
% TextProcessorLib library of unused methods.
% Identified problems with the returned data, leading to an overhaul of the
% calculated values and how they’re stored in the DB backend.
% 
% Mention why it only works on english (future words - language detection is a field)
% 
% ML:
% Consider using PCA - printable component library
% To look at the components
% 			Mention related works in the ML sections
% 
% Algorithm:
% Where’s the bottleneck?
% How accurate is it? - needs to be tested
% Report that we have made improvement from 2 hours to 6 minutes
% 
% Program:
% Benchmark it - find the bottleneck, and maybe comment the user
% 
% 
% Are there any related works to finding opposite opinions
% 
% We have a bit more work with privacy
% 
% Thomas has an idea for contacts in america we can ask for user input.
% 
% For the repport:
% Why is sentiment analysis important to know?
% Add source and explain what if any we have added to feature extraction
% 
% 		(Though Klaus does not think it’s suited to reddit)
% For exam - 	ask reddit to use it for a bit - could it be a weapon?
% 		Add a disclaimer?
% 	Borrow machines in the basement
% 	      	Could use review “quotes” from users