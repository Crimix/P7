\section{Meetings with an Expert} \label{expMeeting}
During the span of the project, we conducted two presentations for Dr. Stefan
Schmid who has made several publications in machine intelligence. After
each presentation, he provided feedback with suggestions on how to improve
the application.
 
\subsection{First Presentation}
This presentation was conducted on 23/10/17 on Cassiopeia. At this point, the
backend was in an early state and the \ac{GUI} was not yet started on. Our idea
for the application had a simple frontpage with a search bar, which lead
straight to the result screen. The idea for this screen is seen on
\autoref{initguisketch2}.\nl

%\figx[0.2]{initguisketch2}{Initial idea for the GUI design.}

This screen is split into general information in the top left, showing the user
name, the amount of followers and how many they are following. Underneath, is
the user's personal bias shown in both left/right and in
liberatarian/authoritarian. At this point in the process, we wanted to include
the latter as well, as that would correspond to the political compass, where the
economic left/right is the x-axis and liberatarian/authoritarian is the y-axis.
This was scraped later in the process, since it is often very hard to classify
every word on that scale. on the top right side, of the screen, the bias of the
filter bubble is shown. Below, is a recommendation section based on the bias of
the filter bubble. For example, a user with a predominantly right wing
filter bubble would get left wing accounts recommended. The recommendation
system was later moved to future work, as it was not possible to implement it within
the timeframe. 

\subsubsection{Feedback}
After the presentation, Stefan Schmid had a number of suggestions. He suggested
being able to know where each member of the filter bubble is placed on the
scale. For the recommendation part, he suggested implementing a filter. For
example, only suggest people with at least a university degree or politicians. 
He also suggested to make feedback about the suggestions possible.
To speed up the tweet retrieving, he suggested storing information from lookups
for a limited amount of time, since people's bias does not change drastically
every week. 

\subsection{Second Presentation}\label{subsec:Exp2}
The presentation was conducted on 29/11/17 on Cassiopeia. When the second
presentation was conducted, the application was nearing its final stages. The
classifier that deems people apolitical if less than 1\% of their tweets are
political, thus excluding them from the filter bubble, was implemented shortly
after the meeting. Furthermore, some time before the meeting, the analyzation
part was optimised from 2 hours down to 6 minutes. Some time after the meeting,
this was further optimized down to a few seconds. The GUI was, around the time
of the meeting, in its very early stages.

\subsubsection{Feedback}
As an additional way of analyzing data, Stefan Schmid suggested using Principal
Component Analysis. Regarding the analysis of the data, while an optimisation
from 2 hours to 6 minutes was good, 6 minutes is still a lot. Therefore, he
suggested identifying the bottleneck. 

