\section{Meetings with an expert} \label{expMeeting}
During the span of the project, we conducted two presentations for Dr. Stefan
Schmid who has made several publications in machine intelligence. After
each presentation, he came with feedback with suggestions on how to improve the
application. 
\subsection{Presentation 1}
This presentation was conducted on 23/10/17 on Cassiopeia.
\subsubsection{State of the application}
At this point, the backend was in an early state and the \ac{GUI} was not yet
started on. Our idea for the application had a simple frontpage with a search
bar, which lead straight to the result screen. The idea for this screen is seen
on \autoref{guisketch2}.

\figx{guisketch2}{Initial idea for the GUI design.}



 the one shown in the review of week 41.
After the presentation of the ideas,



\section{Initial ideas for a GUI} \label{sec:GUI}
The final product has to have a user-friendly and intuitve design. The initial
thoughts for this, included a simple search bar, where the target twitter name
is inserted. This leads to a page that shows various information about the
account:
\subsection*{Basic information}
This part shows basic information about the account, such as name, username,
amount of followers and the amount of pages it follows.
\subsection*{Political spectrum}
This part shows a visual presentation of the user's political stance in the form
of two sliders. The first slider goes from 0 to 10, where 0 is very left
oriented and 10 is very right oriented. The first slider also goes from 0 to 10,
and goes from libertarian to authoritarian.
\subsection*{Your bubble}
This part shows the political spectrum of the ones the target twitter user
follows. As these dictate what is in the target twitter user's twitter feed,
they determine the filter bubble.
 \subsection*{Suggestions to break the filter bubble}
 From the political spectrum of the filter bubble, a list of suggested twitter
 users is shown. This list includes twitter users from the opposite site of the
 political spectrum.
