\section*{Week 43}\label{Week43}
\subsection*{What Happend This Week}
On the report many corrections, some missing introductions and conclusions have
been fixed. From the meeting we found out that we wanted a \ac{GUI} web
application in Laravel which should connect to a
queue server in C\# through a \ac{REST} \ac{API}\citep{laravelIntro}. Then the queue
should call a modified version of our workers: program fetching tweets from Twitter. The
\ac{GUI} and workers communicate with a database through another \ac{REST}
\ac{API}. We split all members into different parts and progressed on all,
investigating how and what we should do with them. We quickly found that the
database \ac{REST} \ac{API} should be made in a separate Laravel project. We
also conducted a presentation for Stefan Schmid which can be found in
\autoref{subsec:Exp1}.

\subsection*{Reflection}
During the presentation Monday we received a lot of feedback on everything, and
some keywords or methods extremely useful to us that we had not considered. Such
a presentation meeting is a very useful tool to strengthen a project and
something which should be done in future projects as well. New mini projects for
the complete project that we had not considered before were found through
brainstorming.

\subsection*{Next Week}
During the next week we will continue development on the parts, we have a lot to
do after this meeting. And we will make some considerations on which type of
user we want, university, high school or other? 
The web crawler chapter should be expanded with more information on the \ac{API}
and perhaps some information on why we describe the crawler since we will not
use it.


% All corrections on the report applied and a further grammar check on new
% stuff, introductions and conclusions are also added to the chapters.
%  After the meeting Monday, we discussed how to expand the program we have
% written to be a web application, as such we decided that we need a gui web
% application in Laravel which connects to a queue server in c# through a rest
% api, then the queue server should start workers which are a modified version
% of our first program. The gui and workers communicate with a database though
% another rest api. The rest of the week was spent on trying first to make the
% database rest api in c# but it was then after many problems it was decided
% that the database rest api should be made in a separate Laravel project. Each
% of the for a mentioned items has been worked on during the week.
%   ML-stuff In regards to classifying tweets the week has been spent
% investigating different types of ML, especially Support Vector Machines, Long
% Short Term Memory and basic Neural Networks, which have shown to be promising.
% There has been some thought regarding using different methods for sentiment
% analysis, such as bag of words vs NNs.
%  There has also been devoted time to figuring out whether we should write the
% ML part in python, as it has a lot of good libraries and documentation, or C#
% which has a what appears to be well regarded libraries and somewhat less
% documentation.
%  There has been work on figuring out how to transform tweets into something
% useful, and so far a tweet parser has been built, which transform the tweet
% from a string to a list of string (tokens).
% There has been thought about further transforming the list to a vector by
% mapping words with unique integers, which should be useful when used by a NN.
%  We have considered options for different methods of training, and arrived
% that we are going use the supervised method, though that means we will have to
% provide a lot of tweets which we will have to classify regarding political
% leaning.
%  Meeting:
% Presentation:
% Click on spectrum. Which friends are where.
% Tune what kind og people you want suggested. Very different, slightly
% different from you? Suggest people with a minimum of intellect (University
% degree? Known speaker?) Lookup existing recommendation systems. Current
% systems, reports, solutions? ADs API can maybe group people a bit? Maybe look
% at hashtags? What is our end goal? A plugin? Standalone application? Take
% additional input from the user. What do they think at the moment? What other
% information to supply? Accuracy can partly be determined by user feedback. Do
% you agree with our conclusion? Lookup study about what which hashtags imply
% about a political stance/leaning.
% Look at related work to ours.
% What is the complexity of our algorithm.
% Store information for lookup later.
% Have two approaches system based/neural network. How do they compare? If we
% want to do preprocessing/plugin/(neural network?) we can ask for access to
% (something? computer??)  Meeting:
% Problem with information about reddit user numbers.
% We could "potentially" just write about why our application would be easy to
% scale, and how we would do it  Should we find people to test the application?
% - Maybe, if we can.
%  How do we expand our application:
% - How would we deal with increased popularity - We should probably do
% something with a browser (plugin/site/server/etc.)  What to do:
% - User interface (Laravel) - Describe Architecture MVC front-end. Unknown on
% back-end - Algorithm which uses emotional words, keywords, and hashtags - If
% people log in using their own account, they get an amount of requests for
% themselves - Convert system to run on a server - Definitely do something to
% consider scalability e.g. multiple servers, or using users own requests -
% Begin writing reflections (what should we have done? What did we do? The
% plan?) - Plan how to systematically evaluate our application. User interface,
% testers, usability, functionality (does it work as intended).
% - Look at lecture notes on how to design interfaces. Then do what the slides
% tell us to do.
% - Expand the webcrawler chapter with more info on the different parts of it,
% and in the end a conclusion with the reasonings why we won’t use it (and
% causes like spoofing and a very naughty crawler) - Lookup existing
% recommendation systems. Current systems, reports, solutions?
