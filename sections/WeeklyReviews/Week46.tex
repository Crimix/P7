\section*{Week 46} \subsection*{What Happend This Week} 
A feature which retrieves tweets from a set of pre-classified political point of
view (left/right) users, and determines what words they are more likely to use.
This requires a large amount of users with pre-known alignments. To begin, we
have manually compiled a list of Twitter users, which the queue server \ac{API}
now interacts with. The queue server is able to queue jobs for execution. The
queue server instance uses the worker for each job. The worker has been adapted
for use by different threads, such that the limits are not the same for each
thread. We have classified 120 tweets by hand, into 5 classes: Right, little
right, neutral, little left, left.


\subsection*{Evaluation} 
We should have started earlier on these weekly summaries and made our notes
better from the beginning, it has caused some mixup and issues doing the clean
writing of them in the report. We have tried to get the same kind of result from
the Naïve Bayes model and algorithm to classify the tweets, this is good since
it permits us to use both or find the one which works best.


\subsection*{Next Week} 
Next week will be slow because of a mini project in Programming Paradigms. Else
everyone has something to work on which will likely not be finished doing that
week.



% We would like feedback on the two weekly summaries, we would like to know if
% the format is fitting.
%   Status:
% We have added a feature to the program which retrieves tweets from a set of
% pre-classified (left/right) users, and determine what words they are more
% likely to use. The idea behind this is, that if we can identify what words are
% most likely to be used by users of specific political alignments, we can
% dynamically update our list of keywords, which we already use for classifying
% users. This, however, requires a large amount of users with pre-known
% alignments. To begin, we have manually compiled a list of twitter users, which
% we use to get started with the word gathering. The idea is, that whenever we
% classify a user with a high degree of certainty, we can add that user to our
% list of users from which we identify words. We have also begun commenting the
% code.
%  The queue server api now interacts with the queue server instance and are
% able to queue jobs for execution. The queue server instance uses the worker
% for each job. The worker has been adapted for use by different threads, such
% that the limits are not the same for each thread.
% Then work began on the section for the queue server api.
%  Did more research into dense feature vectors and other solutions. Found an
% algorithm called word2vec that is interesting.
% With 120 tweets which we have classified by hand, into 5 classes: Right,
% little right, neutral, little left, left. With these tweets we trained a Naive
% Bayesian with bad results, with 5 classes it had a 26% success rate, with 3
% classes it had 44% success. Used a simple SVM with a 3 classes, it had a
% success rate of 30%.
%  Added a TextProcessing Library to the project, which allows us to process
% tweets.
%   As for the agenda:
% We would like feedback on the weekly summaries from the report
