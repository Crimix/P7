\subsection{What You Want is What You Get}
Today when a user uses the in internet, then their activity is logged and used to provide the user with the kind of information that would make them the happiest, this leads to the user finding people agreeing with their point of view instead of challenging it, which leads to personal growth and insight on a topic (https://www.ted.com/talks/eli_pariser_beware_online_filter_bubbles). A users web activity is an enormous amount of data, which will be difficult to manage. But if gathered it will be possible to detect the bubble and show the user that they are in it.

\subsection{Social Media}
A way to limit the enormous amount of data from a user and still get the needed information could be focusing on their activity on social media. There are reports showing that a fairly active user on social medias, then it is possible to sift through the huge amount of data they have and through it find enough information, to come with a qualified guess at what the user is voting for or if there are new movies coming soon they would enjoy watching (http://www.telegraph.co.uk/technology/facebook/11838515/This-online-tool-reveals-your-personality-based-on-Facebook-likes.html).

\subsection{Data Gathering}
Looking at where there is most social activity on the internet then Facebook and YouTube are at the top (https://www.statista.com/statistics/272014/global-social-networks-ranked-by-number-of-users/). However YouTube is largely based on videos and will therefore not be inspected further, other services which focus on private messages between people will also be dismissed. The most suitable choices for data gathering is social sites with many users interacting with each other and opinions and information comes through text. This report will examine Facebook with 2,047, Twitter with 328 and Reddit with 1285 million users. To find out which one is the best fit, to find out if a user is in a bubble on a subject.

\subsection{Grouping by emotional wording}
The information gathered will likely be so vast, that it needs to be sifted to find the special words which permit us to identify a bubble, and in the process removes unimportant text making it manageable. These special words in the texts, can be used to identify a filter bubble (https://www.researchgate.net/publication/317947723_Emotion_shapes_the_diffusion_of_moralized_content_in_social_networks). The article shows that through 563,312 tweets it’s possible to identify the bubble of an individual by looking at whom follows and whats being tweeted, words like “#MoreGunControl” would indicate a political stance on that topic. By making connections between multiple words like this, it is shown that the filter bubble can be located. Even with issues like trolls, miss use and takeover of meaning on words. Some reporters being paid to show objective news are white-listed in the system since they mess up accuracy.

Web Agents


Problem statement