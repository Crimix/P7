\subsection{What You Want is What You Get}
The internet is widely used to find and gather information about any topic, it
provides a vast amount of information from many different sources, and it is
only growing. There are many companies logging what a user does when they are
using the internet, companies like Google use it to provide the user with the
kind of information that would make them the happiest. This leads to the user
finding sources agreeing with their point of view on topics instead of
challenging it, which leads to personal growth and insight on a topic
\citep{tedFilterBubble}. The results from searching on a site like Google return
different results depending on the user, a search on Google could return travel
guides, information on the government, tailored to the user from their web
activity. A users web activity is an enormous amount of data, which makes it
difficult to manage. But if gathered it will be possible to detect a filter
bubble for the user on a subject.

\subsection{Social Media}
A way to limit the enormous amount of data from a user and still get the needed
information could be focusing on their activity on social media.
There are reports showing that a fairly active user on social medias, then it is
possible to sift through the huge amount of data they have and through it find
enough information, to come with a qualified guess at what the user is voting
for or if there are new movies coming soon they would enjoy watching
\citep{Personality}.

\subsection{Data Gathering}
The biggest social medias with most activity, are currently Facebook and YouTube
\citep{SocialMediaStats}.
However, YouTube is largely based on videos which are a less easy media to
manage, therefore YouTube will not be inspected further. Other services which
focus on private direct messages between users will also be dismissed. The most
suitable opinion based social medias for data gathering is sites with many users
openly interacting with each other, and opinions and information comes through
text. This report will examine Facebook with 2,047, Twitter with 328 and Reddit
with 1285 million users. To find out which one is the best fit, to inspect and
find if a user is in a bubble on a subject.

\subsection{Grouping by emotional wording}
Even when limiting the information to social medias then it will likely still be
too immense, with a lot of unnecessary information safely omitted. The data can
be sifted through to find the specific special words, which can be used to
identify a users filter bubble  \citep{EmotionalWords}. The article shows that
through 563,312 tweets it is possible to identify the filter bubble of an
individual, on a topic, by looking at whom the user follows and whats being
tweeted. Words like ``\#MoreGunControl'' and  ´´\#GunLeadsToMurder'' would
indicate a political stance on that topic. The paper further makes the
observation that the more emotional words there are in a tweet the more useful
the tweet becomes for identifying a bubble. By making connections between
multiple words like this, it is shown that the filter bubble can be identified.
Even with an error margin such as trolls, miss use and takeover of the meaning
of a word. Another issue is reporters paid to be objective when they post, and
their ability to stay neutral is required for them to keep their job, such
personalities should be white-listed in the system since they will likely mess
up accuracy.