\chapter{Introduction}
We are currently living in an Age of Information, where people all around the
world, can use the internet in order to acquire insightful news and information
regarding any topic. While this is a benefit, problems can occur in regards to
where people choose to acquire this information. As of August 2017, upwards of
67\% of adults in the United States report that they use social media as their
main source of news information\citep{journalism2017}. The main issue with this
approach to acquiring information is, due to the phenomenon called the ``Filter
Bubble", which was coined by Eli Pariser in his 2007 book ``The Filter
Bubble"\citep{pariser2011filter}. This Bubble refers to the effect of either
selectively choosing information sources based on personal bias, or actual
algorithms encoded in the social media sites, which aim to present the user with
the information which is most likely to be regarded positively. The existence
and enforcement of this bubble can cause problems, in which users are never
exposed to information or views that are opposed to the bubble in which they
find themselves.

While this problem is most notable on social media sites, the approach where
sites selectively choose what information to present is omnipresent, and even
applies to large search engines like Google\citep{filterBubbleDef}. In an
interview with Eli Pariser, he named one of the notable cases of Google
filtering information based on the user's preferences. In this example, two
users would make a google search with the term ``Egypt'', and end up with two
widely different set of results. While one user was presented with information
about ongoing militaristic conflicts, the other user was presented with travel
guides and hotel booking information\citep{nusSduSearch}.
