\chapter{Introduction}
% Needs to be less fluffy.
The internet is widely used to find and gather information about any topics, it
provides a vast amount of information from many different sources, and it is
only growing larger.
While easy access to information is a clear benefit, the vast amount of
information results in each user acquiring a small subset of the available
knowledge.\\
The limits of each user's ability to acquire information puts a focus on the
sources they engage with, and especially how those sources pick and choose what
information to present.
As of August 2017, upwards of 67\% of adults in the United States report that
they use social media as a source of news information \citep{journalism2017}.
Theoretically, this would not be a problem, as users could use the dynamics
reach of social media to engage with information and news from different sources.
But in reality, users tend to surround themselves with like-minded people, who
in turn are prone to share information that correlate with their own opinions
and biases.\\

This phenomenon is known as the ``Filter Bubble", which is a term coined by Eli
Pariser in his 2007 book ``The Filter Bubble" \citep{pariser2011filter}.
This Bubble refers to the effect of either selectively choosing information
sources based on personal bias or actual algorithms encoded in the social media
sites or search engines.
The aim of which is to present the user with the information which is most
likely to be regarded positively.
The existence and enforcement of this bubble can cause problems, in which users
are prevented from being exposed to information or views that are opposed to the
bubble in which they find themselves \citep[p.59-73]{pariser2011filter}.\nl

While this problem is most notable on social media sites, where users actively
surround themselves with like-minded people, it is also present as some sites
selectively choose what information to present, this even applies to large
search engines like Google \citep{filterBubbleDef}.
In an interview with Eli Pariser, he named ``Egypt'' as a notable case of Google
filtering information based on the user's preferences.
In this example, two users would make a google search with the term ``Egypt'',
and end up with two widely different set of results.
While one user was presented with information about ongoing conflicts, the other
user was presented with travel guides and hotel booking information
\citep{nusSduSearch}.