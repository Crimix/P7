
Throughout the analysis, we have investigated how to identify and break out of a
filter bubble. As such, this section will be used summarize the conclusions, and
define a problem statement.\nl
%Klaus: I'm not sure I agree with this statement. I believe you rather compared
% social media sources and then a bit about data gathering

In \autoref{cha:SocialMediaAnalysis} we came to the conclusion, that Twitter is
most suitable for gathering data, as it provides an \ac{API} that allows us to
easily request all relevant data for a user. This data can be requested from any
user, who has not set their account to private. The data consists of tweets,
followers, and friends for any such user.\nl

In \autoref{cha:DataGathering} we discovered that there are restrictions to what
a crawler is allowed to access. We have determined that while it is possible to
use a regular crawler to gather data, it is preferable to use the official
Twitter \ac{API} designed for the purpose. We made this decision, as the
\ac{API} gives access to all relevant data by using a simple set of \ac{HTTP}
requests, while a crawler has to adhere to a stricter set of limitations.
However, we are still able to use the crawler architecture as a guideline for
how to design a solution.\nl




% In \autoref{emotions} we determined that we can identify a users filter bubble
% by how ``emotional'' they were about certain topics. This can be done by
% pairing key topical words like names of political figures, with emotional
% words like ``like", ``good", and ``hate".\nl

Based on these conclusions, we can establish the following problem statements:

\begin{center}
\begin{minipage}{0.95\linewidth} 

\textbk{How can a software solution be developed as to permit a user to break
out of a filter bubble by using information gathered from social media. And how
can the developed software allow identify a user's filter bubble, and present
the user with information from outside of it?.}

\end{minipage}
\end{center}




