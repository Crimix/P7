
Throughout the analysis we have investigated how to identify and break out of a
filter bubble. As such, this section will be used summarize the
conclusions, and define a final problem statement.\nl

In \autoref{sec:SocialMediaAnalysis} we came to the conclusion, that twitter
is most suitable for gathering data, as it provides an API that allows us to easily
request all relevant data for a user. This data can be requested for any user
who have not set their account to private. The data consists of tweets,
followers, and friends for any such user.\nl

In \autoref{sec:DataGathering} we determined that while it is possible to use
a regular crawler to gather data, we are better off using the official
Twitter API. We made this decision, as the API gives access to all relevant
data by using a simple set of HTML requests, while a crawler has to adhere to a
stricter set of limitations.\nl

In \autoref{emotions} we determined that we can identify a users filter
bubble by how ``emotional'' they were about certain topics. This can be done
by pairing key topical words like names of political figures, with emotional
words like ``like", ``good", and ``hate".\nl

Based on these conclusions, we can establish the following problem statements:

\begin{center}
\begin{minipage}{0.95\linewidth} 

\textbk{How can a software solution be developed as to permit a user to break
out of a filter bubble by using information gathered from social media. And how
can the developed software allow identify a user's filter bubble, and present
the user with information from outside of it?.}

\end{minipage}
\end{center}

\section{Features and Requirements}
Based on the previous conclusions we can now define the following features and
requirements:



