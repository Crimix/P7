\chapter{Problem Statement}\label{ch:problem}
% Our project for this semester is to break filter bubbles. Therefore, we
% first approached it by trying to figure out which social media would be most
% suitable and analyzed how to collect a vast amount of user data, from the sites
% to identify a filter bubble.
% This chapter will summarize the conclusions, and define a problem
% statement.\nl
%----------------

As stated in \autoref{sec:initproblem} the goal of this project is
to aid politically active american users in identifying and breaking their
filter bubbles on social media. Based on this problem, \autoref{cha:DG} and
\autoref{cha:DA} were used to identify ways of gathering and analyzing data that
could be used to determine a user's filter bubble.

In \autoref{sec:social-media-conclusion} we concluded, that Twitter is most
suitable for gathering data, as Twitter provides an \ac{API} that permits us
to easily request all relevant data from a user. This data can be requested from
any user, who has not set their account to private. The data consists of tweets,
followers, and friends for any such user.

In \autoref{sec:DAConc} we concluded that sentiment analysis can be used to
determine the filter bubbles for predefined domains. We also found two different
models that are capable of performing sentiment analysis, and a way to use news
media to determine political leanings.

Based on the subject of this semester described in \autoref{sec:initproblem},
namely creating a web-application, we have chosen to develop a system with a
web-page frontend, and a server-side backend to handle the processing of
information. Based on these conclusions, we establish the following problem
statement:\nl

\begin{center}
\begin{minipage}{0.95\linewidth}
\textbk{How can a web-application be developed as to assist a user in
identifying and breaking out of a filter bubble? And how can this be done by
using information gathered from Twitter in the form of tweets?}
\end{minipage}
\end{center}
\newpage


