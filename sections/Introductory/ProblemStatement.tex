\section{Problem Statement}

Throughtout the analysis we have investigated how to identify and break out of a
filter bubble, and can now summarize the conclusions.\nl

In \autoref{sec:SocialMediaAnalysis} we came to the conclusion the twitter is
most suitable for gathering data, as it provides an API that allows us to easily
request tweets from \ldots \fix{}{needs better conclusion based on SMA
conclusion}.\nl

In \autoref{sec:DataGathering} ??\nl

In \autoref{emotions} we decided to identifty a users filter bubble by how
``emotional'' they were about certain topics.\nl

In additions to these conclusions, there are certain requirements to what 


Based on these conclusions, we can establish the following problem statement:



\fix{}{Needs to be remade}
How can a software solution be developed to permit a user to break out of a
filter bubble, using information gathered from social media.
Can a user input a topic and find out that their are in a filter bubble and be
provided with articels from outside?