\chapter{Problem Statement}\label{ch:problem}
% Our project for this semester is to break filter bubbles. Therefore, we
% first approached it by trying to figure out which social media would be most
% suitable and analyzed how to collect a vast amount of user data, from the sites
% to identify a filter bubble.
% This chapter will summarize the conclusions, and define a problem
% statement.\nl
%----------------

Our project for this semester is to break filter bubbles. We have therefore
looked for ways to gather and analyse user data, that could be used to
determine a filter bubble. This chapter will summarize the conclusions and
define a problem statement.\nl

In \autoref{sec:social-media-conclusion} we concluded, that Twitter is most suitable
for gathering user data, as Twitter provide an \ac{API} that permits us to
easily request all relevant data from a user. This data can be requested from
any user, who has not set their account to private. The data consists of tweets,
followers, and friends for any such user.\nl

In \autoref{sec:DAConc} we concluded that sentiment analysis is an interesting
can be used to determine the filter bubbles for predefined domains. We also
found two different models that are capable of performing sentiment
analysis. Based on these conclusions, we establish the following problem
statement:


\begin{center}
\begin{minipage}{0.95\linewidth} 

\textbk{How can a software solution be developed as to permit a user to break
out of a filter bubble by using information gathered from social media. And how
can the developed software allow identify a user's filter bubble, and present
the user with information from outside of it?.}

\end{minipage}
\end{center}




