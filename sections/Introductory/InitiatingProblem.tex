\section{Initiating Problem}\label{sec:initproblem}
The implicit or explicit filtering of information that users are exposed to,
leads to a situation were users are not interacting with information outside
their filter bubble. We believe that this is most problematic when it comes to
politics, as this is the subject where varied discussion is most important. As
such, this project focuses on identifying and breaking a user's political
filter bubble. Based on this problem, we formulate the following initiating
problems:

\begin{center}
\begin{minipage}{0.95\linewidth}

\begin{enumerate}
  \item \textbk{How can a filter bubble be identified by using social media and
  a web service or agent?}
  
  \item \textbk{What information is available to be gathered from social media
  sites?}
  
  \item \textbk{Which technology, method, or approach is suited to gather
  relevant information from social media sites?}
  
  \item \textbk{How can a user's data on social media be analysed and used to
  break their filter bubble?}
  
%   \item \textbk{How can a web agent be developed to identify the user's filter
%   bubble from web activity, and how can it be designed to provide results from
%   outside that bubble?}
\end{enumerate}
\end{minipage}
\end{center}

\section{Target Audience}\label{sec:target}
Regarding the target audience, our observations seem to indicate that the market
best suited for this kind of software would be the United States. This choice is
due to the popularity of social media in the US, where up to 66\% of internet
users visits social media sites \citep{socialMediaActivity}.
In addition, Americans make up the largest user base for social media such as
Twitter and Facebook \citep{socialMediaUsersTwitter, socialMediaUsersFacebook}.
As such, we define our target audience as: \say{Americans who are politically
active on social media.}

\section{Project Outcome}
Over the course of this project, we work with the problem of filter bubbles on
social media and finding ways to identify and break them. Based on the book
``The Filter Bubble''\citep{pariser2011filter} by author Eli Pariser, we start
by defining the term ``filter bubble'' as being an invisible, enforced bubble,
wherein, a user only interacts with news, opinions and arguments which they
perceive positively. 

Based on this problem, we start by analysing different social media sites to
determine which are most suitable for data gathering, and offer the most
comprehensive and useful data for identifying a filter bubble.
Based on this analysis, we conclude that the social media platform Twitter is
best suited for our project. Following this, we develop two approaches to
identifying filter bubbles. The first model uses a Naive Bayes Classifier and
the other uses a modified Bag-of-Words approach.

As a framework for hosting these filter bubble identifiers, we develop a system
consisting of a website front-end, a server back-end and a database for storing
processed data. The project is concluded with two test: a usability test, which
is used to determine the quality of the system's front-end, and an accuracy
test which determines that the Bag-of-Words keywords search is the most
accurate with 96\% accuracy and the Naive Bayes Classifier failing to accurately
classify real users.
The project finishes with a reflection on the two models, where we conclude that
the Bag-of-Words model is good at the moment, but it's keywords are based on a
static image of the current political climate, while the Naive Bayes Classifier
may be bad right now, but it should become better than the Bag-of-Words model over
time, as more data can be used to train and refine the model.



















