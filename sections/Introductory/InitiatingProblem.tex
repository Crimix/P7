\subsection*{Initiating Problem} 
The existance of the filter bubble leads to a situation where users actively or
unknowingly limits their own exposure to news, opinions, and ideas.
Additionally, it seems like the most identifiable filter bubbles are based
around political opinion and ideas. As such, the analysis will be based around
identifying and breaking a users political filter bubble.
Based on this problem, the topic of this project, namely development of web
applications, services and agents, we can formulate the following initiating
problems,  which will be used to direct our analysis of the filter bubbles and
the options for solving them by developing web agents or services.


\begin{center}
\begin{minipage}{0.95\linewidth}

\begin{enumerate}
  \item \textbk{How can a filter bubble be identified by using social media and
  a web service or -agent?}
  
  \item \textbk{What information is available to be gathered from social media
  sites?}
  
  \item \textbk{Which technology, method, or approach is best suited
  to gather relevant information from social media sites?}
  
  \item \textbk{How can a users filter bubble be broken, and how would said
  approach affect a user's daily interaction with social media?}
  
%   \item \textbk{How can a web agent be developed to identify the user's filter
%   bubble from web activity, and how can it be designed to provide results from
%   outside that bubble?}
\end{enumerate}

\end{minipage}
\end{center}


