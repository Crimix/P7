\section{Initiating Problem} 
The implicit or explicit filtering of information that users are exposed to,
leads to a situation were users are not interacting with information outside of
their filter bubble. We believe that this is most problematic when it
comes to politics, as this is the subject where varied discussion is
most important. As such, this project will focus on identifying and breaking a
users political filter bubble.\nl

Based on this problem, the topic of this semester, namely development of web
applications, services, and agents: we formulate the following initiating
problems:

\begin{center}
\begin{minipage}{0.95\linewidth}

\begin{enumerate}
  \item \textbk{How can a filter bubble be identified by using social media and
  a web service or agent?}
  
  \item \textbk{What information is available to be gathered from social media
  sites?}
  
  \item \textbk{Which technology, method, or approach is suited to gather
  relevant information from social media sites?}
  
  \item \textbk{How can a user's be analyzed and used to break the user's filter
  bubble?}
  
%   \item \textbk{How can a web agent be developed to identify the user's filter
%   bubble from web activity, and how can it be designed to provide results from
%   outside that bubble?}
\end{enumerate}
\end{minipage}
\end{center}

\section{Target Audience}\label{sec:target}
Regarding the target audience, our observations seem to indicate that the market
best suited for this kind of software would be the United States. This choice is
due to the popularity of social media in the US, where up to 66\% of internet
users visists social media sites\citep{socialMediaActivity}.
In addition, americans make up the largest user base for social media such as
Twitter and Facebook \citep{socialMediaUsersTwitter, socialMediaUsersFacebook}.
As such, we define our target audience as: \say{Americans who are politically
active on social media.}
