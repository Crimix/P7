\subsection*{Initiating Problem} 
The existence of filter bubbles leads to a situation were users actively or
unknowingly limit their exposure to news, opinions, and ideas. Additionally,
filter bubbles based on political views and ideals are known to have opposing
angels. As such, the analysis will focus on identifying and breaking a users
political filter bubble. Based on this problem, the topic of this project,
namely development of web applications, services, and agents: we formulate the
following initiating problem.


\begin{center}
\begin{minipage}{0.95\linewidth}

\begin{enumerate}
  \item \textbk{How can a filter bubble be identified by using social media and
  a web service or agent?}
  
  \item \textbk{What information is available to be gathered from social media
  sites?}
  
  \item \textbk{Which technology, method, or approach is suited to gather
  relevant information from social media sites?}
  
  \item \textbk{How can a user's be analyzed and used to break the user's filter
  bubble?}
  
%   \item \textbk{How can a web agent be developed to identify the user's filter
%   bubble from web activity, and how can it be designed to provide results from
%   outside that bubble?}
\end{enumerate}

\end{minipage}
\end{center}

\nl \nl \nl 
The following chapter investigates where and how to get enough of a users
activity data. To find a suitable medium to gather the necessary user data and
identify a user's filter bubble.

\fix{}{Here, it would also be nice to give an outlook what will happen in this thesis, i.e., 
in each chapter. Doesn't have to super-long. However, the reader should get an overview how things are connected.
Also, (later, when you have it), what is the result of your thesis (brief)}
