\subsection*{Initiating Problem} The existence of filter bubbles leads to a
situation where users actively or unknowingly limit their own exposure to news,
opinions, and ideas. Additionally, filter bubbles based on political opinions
and ideas are known to have opposing views. As such, the analysis will be based
on identifying and breaking a users political filter bubble. Based on this
problem, the topic of this project, namely development of web applications,
services and agents, we formulate the following initiating problem, which is to
direct our analysis of filter bubbles and the options for solving them by
developing web agents or services.


\begin{center}
\begin{minipage}{0.95\linewidth}

\begin{enumerate}
  \item \textbk{How can a filter bubble be identified by using social media and
  a web service or agent?}
  
  \item \textbk{What information is available to be gathered from social media
  sites?}
  
  \item \textbk{Which technology, method, or approach is best suited
  to gather relevant information from social media sites?}
  
  \item \textbk{How can a user's filter bubble be broken?}
  
%   \item \textbk{How can a web agent be developed to identify the user's filter
%   bubble from web activity, and how can it be designed to provide results from
%   outside that bubble?}
\end{enumerate}

\end{minipage}
\end{center}

\nl \nl \nl 
The following chapter investigates how a users activity on social
platforms, containing a vast amount of user data. To find a suitable medium for
gathering enough user data to identify a filter bubble.
