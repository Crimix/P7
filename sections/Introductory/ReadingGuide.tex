\section*{Acknowledgement}
We would like to thank Klaus-Tycho Förster for his great supervision
and constant feedback during the semester.\nl

We would also like to thank Dr. Stefan Schmid for taking the time to meet with
us twice. He has been a great help in identifying the problems and making our
project more ambitious.\nl

Finally we would like to thank Helle Mouritzen, as she has been
baking and providing us with the best cookies for the entire month of
December.\newpage


\section{Reading Guide}\label{sec:readingGuide}
This section is used to provide an overview of the report's content and a list
of the various terms used throughout the report.

\subsection{Chapter Guide}
\autoref{ch:intro} starts by briefly describing the project subject, namely  the
filter bubble. It ends by introducing a set of initial problems for further
analysis.\nl

\autoref{cha:DG} starts by discussing various social media sites which are
deemed fit as a source of useful user data, and how this data can be
accessed and gathered. It ends by concluding that Twitter is best suited for our
needs and that we will use it as a data source.\nl

\autoref{cha:DA} starts by discussing how user data, in the form of tweets, can
be analysed to determine a user's filter bubble. It ends by concluding
that we want to make two models, one using a Bag-of-Words approach, and another
using a Naive Bayes Network.\nl

\autoref{ch:problem} continues from the analysis of the problem area and states
our intent to break a filter bubble, by determining the political bias of the
people a user follows on Twitter.\nl

\autoref{cha:req} presents a number of functional- and usability requirements,
deemed necessary to successfully develop the software solution. We conclude that
we want to build a web-application with a web page front-end.\nl

\autoref{ch:sysview} presents an overview of the developed system and describes
how its parts interact with each other. The system consists of a web-application
front-end, a database, and a back-end server.\nl

\autoref{cha:twitterAPI} describes how we interact with the Twitter \ac{API}to
gather the required data for analysis.\nl

\autoref{DatabaseAPI} describes the development of the \ac{API} which is used
for interacting with the database. This \ac{API} is developed using the Laravel
web application framework.\nl

\autoref{queueAPI} describes the development of the \ac{API} which is used to
interact with the back-end server. This \ac{API} is developed using the
Microsoft .NET framework in the C\# programming language.\nl

\autoref{workerLabel} describes the development of the Worker class, which is
the part of the back-end server responsible for determining a user's filter
bubble. This is done using both the Bag-of-Words model and the Naive Bayes
Network.\nl

\autoref{cha:classification} describes the development of the two different
classification algorithms, namely the Bag-of-Words, and the Naive Bayes
Network.\nl

\autoref{GUI} presents the \ac{GUI} of the web-application front-end. It also
describes the design choices involved in developing the visual presentation.\nl

\autoref{cha:usability} starts by presenting the usability testing of the
system. It then concludes with the quality of the system's front-end and
determines the number of improvements to be implemented. Finally, it presents
the improvements added based on the usability test.

\autoref{cha:testing} presents a number of tests which aim to ensure the quality
of the developed software solution. It ends by concluding the possible
bottlenecks in the execution process, namely the limited amounts of Tweets
retrievalble per second.\nl

\autoref{fwork} discusses which elements of the system are unfinished, or could
be a subject for future development.\nl

\autoref{cha:conclusion} ends the report by concluding on the project's success
in relation to the system requirements determined in \autoref{cha:req}.






\subsection{Style List}
The following font styles are used throughout the report, and has the following meaning:

\begin{table}[H]
\centering
\begin{tabular}{|l|p{6cm}|}
\hline
\textbf{Style} & \textbf{Meaning} \\ \hline
\textc{Code} &  Code related elements e.g. libraries, methods, and classes.\\ \hline
\textbf{line x-y} & Line reference in code. \\\hline
\end{tabular}
\caption{Notable font styles.}
\end{table}


% \subsection{Word List}
% The following words are used throughout the report and has the following
% meaning:
% 
% \begin{table}[H]
% \centering
% API
% \begin{tabular}{|l|p{6cm}|}
% \hline
% \textbf{Word} & \textbf{Meaning} \\ \hline
% 
% \end{tabular}
% \caption{Notable words}
% \end{table}

\subsection{Code repositories}
The code base of the developed software solution's parts can be found in the
following repositories:

\begin{table}[H]
\centering
\begin{tabular}{|l|p{6cm}|}
\hline
\textbf{Project Part} & \textbf{URL} \\ \hline
Worker \& Queue Server / \acs{API}  & \url{https://github.com/Crimix/BubbleBuster}.\\\hline 
Database \acs{API} / server & \url{https://github.com/Crimix/EnragedWindigo-DB}.\\\hline 
Front-end & \url{https://github.com/Crimix/EnragedWindigo}.
\\\hline
\end{tabular}
\caption{Solution repositories.}
\end{table}