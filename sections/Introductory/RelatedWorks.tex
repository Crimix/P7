\section{Related Works} There have been a number of papers written on
the subject of sentiment analysis using social media.
\citep{pennacchiotti2011machine} employs machine learning to analyze a users
feed on the social media site Twitter. They specifically looked to classify a
users political orientation or ethnicity by using observable information such as
user behavior, network structure, and the linguistic content of their
communications. Another approach comes from \citep{sarlan2014twitter} who
attempts to use a bag-of-words model for sentiment analysis, where a lexicon of
sentimental words are used to identify a general sentiment.
This approach aims to enable companies to assess what users think of their
products. This works by splitting their posts on social media into positive and
negative categories based on their use of emotional keywords. A different
approach is also utilized by \citep{go2009twitter}, who aim to determine a users
sentiment by using a Naive Bayes classifier trained using positive and negative
social media posts.