\chapter{Crawling}
%Small introduction to chapter?

A crawler, also known as a spider, crawls the web in order to gather
information from the internet by reading the content of a page. Each page
usually contain multiple links to other pages which can then be crawled and
indexed. During the crawl there are several things to consider, the crawler
must be able to avoid getting stuck fetching an infinite number of pages from
the same domain, as this is only acceptable if you want to crawl that specific
site. The crawler also needs to be 'polite' about how frequently it makes its
request, as too many requests to the same server can crash it. The crawler
should also respect what it is and is not allowed to access from a given
domain. Additionally, the crawler should also be able to refetch older sites,
in order to see if they have been updated. It should favor fetching pages that
are more likely to have quality content
\citep[Ch. 20.1]{manning2008introduction}.\nl

%General webcrawler structure
\section{Crawler design}
The crawler architecture, a basic version be seen in \autoref{BasicWC}, is
designed to be modular. The 'Fetch' module is designed to take a URL from the
frontier and request access to the corresponding site. The site content is then
passed to the 'Parse' module, where it is parsed, see \autoref{sec:parsing}.
The content is then verified to see if has been seen before. The URLs are then
filtered according to the domains ``/robots.txt'', which describes what the
crawler is allowed to access. Finally the duplicate URLs are eliminated and the
new URLs are added to the frontier.

\figx[0.8]{BasicWC}{The basic crawler architech \citep[Fig
20.1]{manning2008introduction}}

%Flesh out more
A more advanced version of the crawler involves managing multiple distributed
crawlers, such that they are able to work together efficiently.

%Should it be called parsing instead?
\section{Parsing} \label{sec:parsing}

Parsing is used to extract important information from a document and then
indexing it for later use. Indexing is an essential part of crawling, as it
speeds up the information retrieval process immensely. This is because it is a
lot faster to search through a list of indexes than to scan through every
single page every time a query is performed.

\subsection{Document Parsing}
The first part of parsing a document consists of removing the unnecessary
information such as HTML tags, while retaining the important information such
as links and the body text. Depending on the document, it may be necessary to
decode the document format and character encodings.

\subsubsection{Tokenization}
The second step is to tokenize the text, which means to split up the text in the
smallest meaningful entities, which in this context are words. The tokens are
then saved in a list. Depending on how advanced the tokenizer is  handle
abbreviations such as ``aren't'' or ``kg''. Another importent part of the
tokenizer is to remove stopwords. Stopwords are words that do not contain any
meaningful information, and should therefore not be indexed. Examples of this
are words such as``the'' and ``a''. These words do not refer to anything in
particular, a query should not match indexed pages based on them \citep[Ch.
2]{manning2008introduction}.

\subsubsection{Indexing}
There are different types of indexes which depends on the way that the
tokens and documents refer to each other. The two main types are the forward
index and inverted index and \autoref{iIndex}.

\begin{minipage}{.40\textwidth}
  \centering
  \begin{table}[H]
	\centering
    \label{fIndex}
    \begin{tabular}{|l|l|}
\hline
Doc1 & fresh, tomato, soup \\ \hline
Doc2 & fresh, potato, soup \\ \hline
Doc3 & fresh, tomato, sauce \\ \hline
	\end{tabular}
	\caption{A forward index}
  \end{table}
\end{minipage}
\begin{minipage}{0.5\textwidth}
  \centering
  \begin{table}[H]
	\centering
    \label{iIndex}
    \begin{tabular}{|l|l|}
\hline
fresh & Doc1, Doc2, Doc3 \\ \hline
tomato & Doc1, Doc3 \\ \hline
soup & Doc1, Doc2 \\ \hline
sauce & Doc3 \\ \hline
	\end{tabular}
	\caption{A simple inverted index}
  \end{table}  
\end{minipage}

In the forward index, the pages themselves are indexed with a reference to the
tokens they contain, see \autoref{fIndex}. An inverted index is made up of
tokens referencing the documents which contains them, this is useful for
querying for specific tokens \citep{Index3}.



%How could we use it for filter bubble



% Crawler must provide:
% Robustness: Avoid getting stuck fetching an infinite number of pages from a particular domain, either through traps or bad web design.
% Politeness: Web servers have both implicit and explicit policies regulating the rate at which a crawler can visit them. 
% 
% Crawler should provide:
% Distributed: Execute in a distributed fashion across multiple machines.
% Scalable: Can speed up scale crawl by adding more machines and bandwidth.
% Performance and efficiency: Make efficient use of system resources. 
% Quality: The crawler should be biased towards fetching “useful” pages first.
% Freshness: The crawler should operate in continuous mode: it should obtain fresh copies of previously fetched pages.
% Extensible: The crawler architecture be modular.
% 
% Basic properties any non-professional crawler should satisfy:
%  Only one connection should be open to any given host at a time.
%  A waiting time of a few seconds should occur between successive requests to a host.
% Politeness restrictions. We should not perform denial-of-service by repeated pings. Certain places of off limits, this is denoted by the ROBOTS EXCLUSION PROTOCOL.
% 
% Fig 1: Basic crawler architecture
% 
% Short description of a basic web crawler:
% A crawler thread begins by taking a URL from the frontier and fetching the web page at that URL, generally using the http protocol. The fetched page is then written into a temporary store, where a number of operations are performed on it. Next, the page is parsed and the text as well as the links in it are extracted. The text (with any tag information – e.g., terms in boldface) is passed on to the indexer. Link information including anchor text is also passed on to the indexer for use in ranking. In addition, each extracted link goes through a series of tests to determine whether the link should be added to the URL frontier
% Content Seen? module: Tests whether the content has been seen, try using a fingerprint such as the checksum or perform shingling.
% URL filter module: Can exclude domains and performs the Robots Exclusion Protocol, which is seen in a file called robots.txt which is at the root of the URL hierarchy. Robots filtering must be performed before attempting to fetch a page. URL’s should be normalized. 
% Duplicate URL Elimination module: Check to see whether the normalized URL is in the frontier. If it is not in the frontier it should receive a priority and be added.
% 
% 
% How do the various nodes of a distributed crawler communicate and share URLs? The idea is to replicate the flow of Fig 1 at each node, with one essential difference: following the URL filter, we use a host splitter to dispatch each surviving URL to the crawler node responsible for the URL; thus the set of hosts being crawled is partitioned among the nodes. This version can be seen in figure 2.
% 
% 
% Fig 2: Distributed webcrawler
% 
% During DNS resolution, the program that wishes to perform this translation (in our case, a component of the web crawler) contacts a DNS server that returns the translated IP address.
% Due to the distributed nature of the Domain Name Service, DNS resolution may entail multiple requests and round-trips across the internet, requiring seconds and sometimes even longer. There is another important difficulty in DNS resolution; the lookup implementations in standard libraries (likely to be used by anyone developing a crawler) are generally synchronous. This means that once a request is made to the Domain Name Service, other crawler threads at that node are blocked until the first request is completed.
% 
% The URL frontier at a node is given a URL by its crawl process (or by the host splitter of another crawl process). It maintains the URLs in the frontier and regurgitates them in some order whenever a crawler thread seeks a URL. The priority of a page should be a function of both its change rate and its quality.
