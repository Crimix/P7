\chapter{GUI} \label{GUI}
Whenever a user interacts with our system, it will be done by using our
web-application front-end. As such, it is important to provide an optimal user
experience, such that users will be interested in using and recommending our
system. As such, this chapter will be used to present the design and
implementation of our front-end, based on the usability requirements found in
\autoref{cha:req}. In \autoref{sec:GI} we describe how we have implemented the
GUI using the Laravel framework, specifically which models we use, how we
designed the presentation layers and queue system, and how we perform database
migrations. Following this, \autoref{sec:gui-visual-design} will be used to
present the design process and choices that went into making the graphical user
interface.


\section{Implementation}\label{sec:guiimplement}
We chose to build the \ac{GUI} on top of the Laravel Framework that we mentioned
in \autoref{sec:laravel}. Just as with the database \ac{API} this has allowed us
to work on our own implementation details and the usability aspects.\nl

Several aspects of our implementation are described below.

\subsection{Models}
We chose to use Laravel's \ac{ORM} class which meant that we could just define
the tables we needed and then tell Laravel which models to generate from
specific tables.\nl

The tables were created using Laravel's \ac{DB} migration functionality, a
declarative manner in which tables can be defined without having to consider
which \ac{DBS} is used. \autoref{PhpMigrationEx} shows the important parts of
the migration that creates the table for one of our models, as well as removing
it if the migration is ever undone (the \ttt{down()} method).\nl

The \ttt{Schema} class represents the database schema and allows for both
creation and modification of tables, including dropping tables. The methods
called on the \ttt{Blueprint} instance \ttt{\$table} are some of the
abstractions offered by Laravel, which are then converted to \ac{SQL} queries for the chosen
\ac{DB}. The convenience method \ttt{timestamps()} creates two columns;
\ttt{created\_at} and \ttt{updated\_at}. When these are present Laravel will
automatically keep these updated when an entry is created or modified.\nl

\begin{minipage}[H]{\linewidth}
\begin{lstlisting}[caption = Example of a migration in Laravel, label =
PhpMigrationEx, language = PHP, style = PHP]
public function up()
{
  Schema::create('twitter_requests', function (Blueprint $table) {
    $table->increments('id');
    $table->string('email');
    $table->string('request_ident')->unique();
    $table->string('twitter_username', 50);
    $table->string('access_token');
    $table->timestamps();

    $table->index('twitter_username');
  });
}

public function down()
{
  Schema::dropIfExists('twitter_requests');
}
\end{lstlisting}
\end{minipage}

In \autoref{GUImodels} the tables for our GUI project can be seen, of which
most are infrastructure tables. The table \ttt{migrations} tracks the applied
migrations, while \ttt{users} and \ttt{password\_resets} are used by Laravel's
auth system. \ttt{jobs} and \ttt{failed\_jobs} are used by the queue system
described in \autoref{sub:gui-queue}.\nl

The last table; \ttt{twitter\_requests} is used to keep track of the requests by
users to have Twitter accounts analysed by our system. The requests are kept in
the database after having been forwarded to the queue server (described in
\autoref{queueAPI}). This is done to ensure that notification emails can be
sent to the user who made the request once processing has completed.\nl

In order to clean up periodically a task is scheduled to run once a week,
removing \ttt{TwitterRequest} objects that are at least a week old.

\figxb[0.5]{GUImodels}{Tables for the GUI models.}

\subsection{Presentation layer}
The \ac{GUI} is primarily a multi--page application, with individual,
dynamic pages existing for the landing page (the first page the users encounter
when accessing the site), the confirmation page when a \ttt{TwitterRequest} has
been created successfully, as well as the results display.\nl

In order to streamline the experience for the user when creating a new request,
we created a small single--page application running on the front page, in
the form of a component written for the JavaScript framework called ``Vue''. The
component sends and retrieves information to and from the server using
\ac{AJAX} requests.

\subsection{Queue system} \label{sub:gui-queue}
In order to ensure a smooth experience for the users we decided to use Laravel's
queue system. To do this we created a \ttt{ForwardTwitterRequest} class that
implements an interface called \ttt{ShouldQueue}. When a user completes a
request creation, the \ttt{TwitterRequest} object is instantiated with the
necessary details and saved in the local database.\nl

Once the request has been stored a \ttt{ForwardTwitterRequest} is
created, providing the \ttt{TwitterRequest} instance. The request is enqueued and a
``success'' message is sent back to the Vue component, informing the user of the
creation.\nl

When Laravel's queue worker is executed next it will start processing the
requests, executing the \ttt{handle()} method shown in
\autoref{phpForwardTwitterRequest}. If it fails it will throw an exception which
will tell Laravel to retry the processing, and if it fails ten times the request
is marked as failed.\nl

\begin{minipage}[H]{\linewidth}
\begin{lstlisting}[caption = Handle method of the ForwardTwitterRequest class,
label = phpForwardTwitterRequest, language = PHP, style = PHP]
public function handle() {
  if (empty($this->twitterRequest)) {
    return;
  }

  $requestId   = $this->twitterRequest->requestId;
  $accessToken = json_decode($this->twitterRequest->access_token, true);
  $guzzle      = new GuzzleClient([
    'base_uri'    => config('ew.queue.url'),
    'http_errors' => false,
  ]);
  $response    = $guzzle->request(
    'POST',
    '/api/AnalyzeTwitterAccount',
    [
      'form_params' => [
        'Name'          => $this->twitterRequest->twitter_username,
        'RequesterName' => $accessToken['screen_name'],
        'Token'         => $accessToken['oauth_token'],
        'Secret'        => $accessToken['oauth_token_secret'],
        'RequestID'     => $requestId,
      ],
      'headers'     => [
        'Accept'        => 'application/json',
      ],
    ]
  );

  if ($response->getStatusCode() !== 200) {
    throw new Exception("Error Processing Request", $response->getStatusCode());
  }
}
\end{lstlisting}
\end{minipage}

\subsection{Charts generation}
In order to generate the charts we looked at a few different JavaScript
frameworks, settling on one called ``Chart JS'' which allowed us to create
both scatter plots and bar charts. An additional support library, written in
PHP, allowing for easy integration with Laravel was also chosen, enabling us to
define the data as PHP arrays and then ask the library to generate the
JavaScript code for Chart.js.

\section{Visual Design} \label{sec:gui-visual-design}
As described in \autoref{ch:intro}, the target audience of our application is
Americans who are politically active on Twitter. As such, the goal is to create
a GUI that reflects the expectations of this group of users. As such, this
chapter will be used to present and discuss the final design of the GUI. As only
select parts are presented in this chapter, additional screenshots of the
finished design can be found in \autoref{app:guifig}. \\
Overall, the design of our system is based on the fact that politics is a
serious topic, and as such, so we aim to create a simplistic and formal
interface that does not include unnecessary distractions.

\subsection{Front page}
The front page features the logo with the search bar below. Underneath the
search bar is the search button and a information button. A visual presentation
of the front page can be seen on \autoref{GUIfrontPage}.

% \usepackage{graphics} is needed for \includegraphics
\figxb[0.5]{GUIfrontPage}{The front page of the application.}

The color scheme on the front page is a white background with a dark gray text.
The white background promotes a feeling of cleanliness and simplicity, while the
gray text promotes neutrality and formality \citep[p. 63 \& 64]{WebUI}. 
The ``Check user'' button is a middle ground between light and dark blue, which
results in a promotion of calm, safety and reliability \citep[p. 61]{WebUI}.

\subsection{Authorization popup}

When searching for a user, a popup requiring a login pops up. The reason why
logging in is required is due to the fact that a limited amount of requests to
Twitter is allowed. By logging in as another user, we use their requests rather
than our own. A visual representation of the popup can be seen on
\autoref{GUIauthPopup}.

\figx[0.6]{GUIauthPopup}{The authorization popup.}

The Popup window has, like the front page, a blue button and a gray font on a
white background for the same reasons and to keep true to the theme.
Furthermore, there is a green button to acquire the authorization PIN from
Twitter. This button is green so that it is in contrast to everything else one
the screen. Green is a gap between calming colors like the ones used by our
application and energizing colors like orange and yellow, so it catches the
user's eye \citep[p. 60]{WebUI}. Additional pictures of the pages met during the
authorization PIN retrievement can be seen on \autoref{GUItwitterPin}.

\subsection{The E-mail}
After pressing the blue ``Verify" button, the application starts processing.
When the application is done processing the request, an E-mail is sent to the provided E-mail address. This E-mail consists of a short text telling the user
that the results are in for the requested Twitter account and provides a link to
the results.
A visual presentation of the E-mail can be seen on \autoref{GUImail}.

\figx[0.7]{GUImail}{The email received after the application is done
processing.}

The E-mail is Gray with a white box with the information and the blue button.
This is done to stay true to the color scheme. 

\subsection{The result screen}
After clicking on the blue button in the E-mail, a web page with the results
opens. On this page, a header shows some information about the retrievement
process itself, shown on \autoref{GUIresultHeader}. Beneath
the header is a bunch of graphs depicting different aspects of the analysis.
Each graph has a long and detailed description of what is on the graph and how
it was calculated. The graph on \autoref{GUIresult1} depicts political bias and
the average sentiment of everyone in the filter bubble and the user itself. 

\figx[0.5]{GUIresult1}{Graph showing the relation between bias}

Apart from this, there are two bar charts showing the bias of the
filter bubble according to the Naive Bayes algorithm and the Bag of words
algorithm respectively. The bar chart for the bag of words algorithm can be seen
on \autoref{GUIresult2}. The one for Naive Bayes can be seen in
\autoref{GUIresult3} together with the charts for
sentiment analysis and media bias described below which are both calculated by the Bag of Words algorithm.

\figx[0.5]{GUIresult2}{The bar chart depicting bias according to the Bag
of Words algorithm.}

The design of the result screen follows the gray theme. The bars are colorcoded,
such that the bars around the middle are gray and transparent, whereas the
farther the bars are from the middle, the more red/blue and opaque they become.
This is done to visually represent which side the users are placed, since the
left side is associated with blue, whereas the right side is associated with red
in American politics. There is an exception with the bar where the checked user
is located which is colored yellow. On the graph depicting bias and sentiment as
a scatter plot, all dots to the right are red, all dots between -1 and 1 bias are
gray and all dots to the left are blue. The dot representing the checked user
is green.

\subsection{Overall design choice}
The reason for the overall design choice is the power of simplicity. By having a
simplistic design with a lot of whitespace, all focus goes to what is
important; The text boxes and buttons. There are no distracting
pictures or unnecessary information.\citep[p. 26 \& 32]{WebUI}.\\