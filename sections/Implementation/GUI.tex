\chapter{GUI} \label{GUI}
Whenever a user interacts with our system, it is done by using our
web application front-end. It is important to provide an optimal user
experience, such that users are interested in using and recommending our
system. This chapter is used to present the design and implementation of our
front-end, based on the usability requirements found in \autoref{cha:req}.\nl

In \autoref{sec:guiimplement}, we describe how we have implemented the GUI using
the Laravel framework, how we designed the presentation layers and queue system
and how we perform database migrations. Following this,
\autoref{sec:gui-visual-design} is used to present the design process and
choices that went into making the \ac{GUI}.


\section{Implementation}\label{sec:guiimplement}
We chose to build the \ac{GUI} on top of the Laravel framework which we
mentioned in \autoref{sec:laravel}. Just as with the database \ac{API}, this has
allowed us to work on our own implementation details and the usability aspects.\nl

Several aspects of our implementation are described below.

\subsection{Models}
We chose to use Laravel's \ac{ORM} class which meant that we could just define
the tables we needed and then tell Laravel which models to generate from
specific tables.\nl

The tables were created using Laravel's \ac{DB} migration functionality, a
declarative manner in which tables can be defined without having to consider
which \ac{DBS} is used. \autoref{PhpMigrationEx} shows the important parts of
the migration that creates the table for one of our models, as well as removing
it if the migration is ever undone (the \ttt{down()} method).\nl

The \ttt{Schema} class represents the database schema and allows for both
creation and modification of tables, including dropping tables. The methods
called on the \ttt{Blueprint} instance \ttt{\$table} are some of the
abstractions offered by Laravel, which are then converted to \ac{SQL} queries
for the chosen \ac{DB}. The convenience method \ttt{timestamps()} creates two
columns; \ttt{created\_at} and \ttt{updated\_at}. When these are present Laravel
will automatically keep these updated when an entry is created or modified.\nl

\begin{minipage}[H]{\linewidth}
\begin{lstlisting}[caption = Example of a migration in Laravel., label =
PhpMigrationEx, language = PHP, style = PHP]
public function up()
{
  Schema::create('twitter_requests', function (Blueprint $table) {
    $table->increments('id');
    $table->string('email');
    $table->string('request_ident')->unique();
    $table->string('twitter_username', 50);
    $table->string('access_token');
    $table->timestamps();

    $table->index('twitter_username');
  });
}

public function down()
{
  Schema::dropIfExists('twitter_requests');
}
\end{lstlisting}
\end{minipage}

In \autoref{GUImodels}, the tables for our GUI project can be seen, of which
most are infrastructure tables. The table \ttt{migrations} tracks the applied
migrations while \ttt{users} and \ttt{password\_resets} are used by Laravel's
authentication system. \ttt{jobs} and \ttt{failed\_jobs} are used by the queue
system described in \autoref{sub:gui-queue}.\nl

The last table, \ttt{twitter\_requests}, is used to keep track of the requests
by users to have Twitter accounts analysed by our system. The requests are kept in
the database after having been forwarded to the queue server (described in
\autoref{queueAPI}). This is done to ensure that notification emails can be
sent to the user who made the request once processing has completed.\nl

To clean up periodically, a task is scheduled to run once a week,
removing \ttt{TwitterRequest} objects that are more than a week old.

\figxb[0.5]{GUImodels}{Tables for the GUI models.}

\subsection{Presentation Layer}
The \ac{GUI} is primarily a multi--page application with individual,
dynamic pages. The first page, the users encounter when accessing the site is
the landing page. The second page is the confirmation page when a
\ttt{TwitterRequest} has been created successfully. The third page is the
results display.\nl

To streamline the experience for the user when creating a new request,
we created a small single--page application running on the front page, in
the form of a component written for the JavaScript framework called
\textit{Vue}.
The component sends and retrieves information to and from the server using
\ac{AJAX} requests.

\subsection{Queue System} \label{sub:gui-queue}
To ensure a smooth experience for the users, we decided to use Laravel's
queue system. To do this, we created a \ttt{ForwardTwitterRequest} class that
implements an interface called \ttt{ShouldQueue}. When a user completes a
request creation, the \ttt{TwitterRequest} object is instantiated with the
necessary details and saved in the local database.\nl

Once the request has been stored, a \ttt{ForwardTwitterRequest} is
created, providing the \ttt{TwitterRequest} instance. The request is enqueued and a
success message is sent back to the Vue component, informing the user of the
creation.\nl

When Laravel's queue worker is executed next time, it will start processing the
requests, executing the \ttt{handle()} method shown in
\autoref{phpForwardTwitterRequest}. If it fails, it throws an exception
which tells Laravel to retry the processing. If it fails ten times the request is
marked as failed.\nl

\begin{minipage}[H]{\linewidth}
\begin{lstlisting}[caption = Handle method of the \textc{ForwardTwitterRequest}
class., label = phpForwardTwitterRequest, language = PHP, style = PHP]
public function handle() {
  if (empty($this->twitterRequest)) {
    return;
  }

  $requestId   = $this->twitterRequest->requestId;
  $accessToken = json_decode($this->twitterRequest->access_token, true);
  $guzzle      = new GuzzleClient([
    'base_uri'    => config('ew.queue.url'),
    'http_errors' => false,
  ]);
  $response    = $guzzle->request(
    'POST',
    '/api/AnalyzeTwitterAccount',
    [
      'form_params' => [
        'Name'          => $this->twitterRequest->twitter_username,
        'RequesterName' => $accessToken['screen_name'],
        'Token'         => $accessToken['oauth_token'],
        'Secret'        => $accessToken['oauth_token_secret'],
        'RequestID'     => $requestId,
      ],
      'headers'     => [
        'Accept'        => 'application/json',
      ],
    ]
  );

  if ($response->getStatusCode() !== 200) {
    throw new Exception("Error Processing Request", $response->getStatusCode());
  }
}
\end{lstlisting}
\end{minipage}

\subsection{Charts Generation}
To generate the charts, we looked at a few different JavaScript
frameworks, settling on one called \textit{Chart JS} which allowed us to
create both scatter plots and bar charts. An additional support library, written in
PHP, allowing for easy integration with Laravel was also chosen, enabling us to
define the data as PHP arrays and then ask the library to generate the
JavaScript code for \textc{Chart.js}.

\section{Visual Design} \label{sec:gui-visual-design}
As described in \autoref{cha:DG}, the target audience of our application is
Americans who are politically active on Twitter. The goal is to create a GUI
that reflects the expectations of this group of users. Therefore, this section 
presents and discusses the final design of the GUI. As only select parts are
presented in this section, additional screenshots of the finished design can be
found in \autoref{app:guifig}. \\
Overall, the design of our system is based on the fact that politics is a
serious topic so we aim to create a simplistic and formal interface that does
not include unnecessary distractions.

\subsection{Front Page}
The front page features the logo with the search bar below. Underneath the
search bar is the search button and a information button. A visual presentation
of the front page can be seen in \autoref{GUIfrontPage}.

% \usepackage{graphics} is needed for \includegraphics
\figxb[0.5]{GUIfrontPage}{The front page of the application.}

The colour scheme on the front page is a white background with a dark grey text.
The white background promotes a feeling of cleanliness and simplicity, while the
grey text promotes neutrality and formality \citep[p. 63 \& 64]{WebUI}. 
The \textit{Check user} button is a middle ground between light and dark blue,
which results in a promotion of calm, safety and reliability \citep[p. 61]{WebUI}.

\subsection{Authorisation Dialog}

When searching for a user, an authorisation dialog pops up. This is due to the
limitations described in \autoref{cha:twitterAPI}. By logging in as another
user, we use their requests rather than our own. A visual representation of the
dialog can be seen in \autoref{GUIauthPopup}.

\figx[0.6]{GUIauthPopup}{The authorisation dialog.}

The dialog window has, like the front page, a blue button and a grey font on a
white background for the same reasons and to keep true to the theme.
Furthermore, there is a green button to acquire the authorisation PIN from
Twitter. This button is green so that it is in contrast to everything else one
the screen. Green is a gap between calming colours like the ones used by our
application and energising colours like orange and yellow, so it catches the
user's eye \citep[p. 60]{WebUI}. Additional pictures of the pages met during the
authorisation PIN retrievement can be seen in \autoref{GUItwitterAuthTab}
and \autoref{GUItwitterPin}.

\subsection{The E-mail}
After pressing the blue \textit{Verify} button, the application starts
processing.
When the application is done processing the request, an E-mail is sent to the provided E-mail address. This E-mail consists of a short text telling the user
that the results are in for the requested Twitter account and provides a link to
the results.
A visual presentation of the E-mail can be seen in \autoref{GUImail}.

\figx[0.7]{GUImail}{The E-mail received after the application has finished
processing.}

The E-mail is grey with a white box with the information and the blue button.
This is done to stay true to the colour scheme. 

\subsection{The Result Screen}
After clicking on the blue button in the E-mail, a web page with the results
opens. On this page, a header shows some information about the retrievement
process itself, shown in \autoref{GUIresultHeader}. Beneath
the header is a bunch of graphs depicting different aspects of the analysis.
Each graph has a long and detailed description of what is on the graph and how
it was calculated. The graph in \autoref{GUIresult1} depicts political bias and
the average sentiment of everyone in the filter bubble and the user itself. 

\figx[0.5]{GUIresult1}{Graph showing the relationship between bias and
sentiment.}

Apart from this, there are two bar charts showing the bias of the filter bubble
according to the Naive Bayes model and the Bag-of-Words model
respectively. The bar chart for the Bag-of-Words model can be seen in
\autoref{GUIresult2}. The one for Naive Bayes can be seen in
\autoref{app:guifig} together with the charts for sentiment analysis and media
bias which are both calculated by the Bag-of-Words model.

\figx[0.5]{GUIresult2}{The bar chart depicting bias according to the 
Bag-of-Words model.}

The design of the result screen follows the grey theme. The bars are
colour-coded, such that the bars around the middle are grey and translucent,
whereas the further the bars are from the middle, the more red/blue they become.
This is done to visually represent which side the users are placed, since the
left side is associated with blue, whereas the right side is associated with red
in American politics. There is an exception with the bar where the checked user
is located which is coloured yellow. On the graph depicting bias and sentiment
as a scatter plot, all dots to the right are red, all dots between -1 and 1 bias
are grey and all dots to the left are blue. The dot representing the checked
user is currently green, as we have not yet changed its color to yellow.

\subsection{Overall Design Choice}
The reason for the overall design choice is the power of simplicity. By having a
simplistic design with a lot of whitespace, all focus goes to what is
important; The text boxes and buttons. There are no distracting
pictures or unnecessary information \citep[p. 26 \& 32]{WebUI}.\\