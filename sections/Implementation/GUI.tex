\chapter{GUI} \label{GUI} \fix{}{Could be a section with subsections instead. I
don't know.}
Since the target audience is americans who are politically active on twitter,
the goal is to create a GUI that reflects this. politics is a serious topic, so
we aim to create a simplistic and formal interface that does not include
unnecessary distractions.
\section{Front page}
The front page features the logo with the functionality below. For this
project, only searching for Twitter users and getting their filter bubble is
possible. More functionality can be added for future work. A visual
presentation of the front page can be seen on \autoref{frontPage}.

% \usepackage{graphics} is needed for \includegraphics
\figx[0.6]{frontPage}{The front page of the application.}

The colorscheme on the front page is a white background with a dark gray text.
The white background promotes a feeling of cleanliness and simplicity, while the
gray text promotes neutrality and formality\citep[p. 63 \& 64]{WebUI}. 

\section{Request page}
After clicking on the ``CHECK TWITTER USER'' button, it opens the interface for
the request page. It features an input text box for the Twitter username and a
search button. A visual presentation of the request page can be seen on \autoref{twitterRequest}.

\figx[0.6]{twitterRequest}{The ``Check twitter user'' page.}

The request page uses a gray background for the neutrality and formality that we
want to promote due to the fact that the whole purpose of the project is to
make Twitter users have a more politically neutral in their Twitter feed.
The button is a middle ground between light and dark blue, which results in
a promotion of calm, safety and reliability \citep[p. 61 \& 64]{WebUI}.

\section{Authorization popup}

When searching for a user, a popup requiring a login pops up. The reason why
logging in is required is due to the fact that a limited amount of requests to
Twitter is allowed. By logging in as another user, we use their requests rather
than our own. A visual representation of the popup can be seen on
\autoref{twitterAuth}.

\figx[0.6]{twitterAuth}{The authorization popup.}

The Popup window has a simple black text on a white background. The background
is white rather than gray, such that it stands out from the request page. The
button is still blue, such that it provides the user a feeling of calm, safety
and reliability, which is the perfect feeling to provide when asking them to
trust us with their authorization PIN\citep[p. 61]{WebUI}.

\section{The E-mail}
When the application is done processing the request, an E-mail is sent to the
provided E-mail address. This E-mail consists of a short text telling the user
that the results are in and a link to the results. A visual presentation of the
E-mail can be seen on \autoref{email}.

\figx[0.6]{email}{The email received after the application is done processing.}

The E-mail is Gray with a white box with the information and the blue button.
This is done to stay true to the color scheme.

\section{The result screen}
\fix{}{Make this when the screen is done.}

\section{Overall design choice}
The reason for the overall design choice is the power of simplicity. By having a
simplistic design with a lot of whitespace, all focus goes to what is
important; The text boxes and buttons. There are no distracting
pictures or unnecessary information.\citep[p. 26 \& 32]{WebUI}.\\