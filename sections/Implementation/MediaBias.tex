\section{Media bias} \label{sec:mediabias}
As the goal of the project is to break the filter bubble, it is essential to
identify the political agenda and bias of different news outlets. This way, it
is possible to determine the political orientation of the user, based on their
choice of sources, and provide relevant news stories from the other side of the
political spectrum.
\subsection{Classification}
To classify the news outlets, they are placed on a scale featuring the labels
´´left'', ´´left-leaning'', ´´center'', ´´right-leaning'' and ´´right''.
The classification of the news outlets are based on the research by AllSides
\ref{ref to this}, which obtains its data from any combination of the following
methods:
\subsubsection*{Blind survey}
Survey takers are asked to rate the bias news stories without knowing the source
of them. This data is then normalized, such that the data is equally represented
by all social and political backgrounds.
\subsubsection*{Third party data}
This method refers to gathering data from outside sources, which AllSides deems
reliable.
\subsubsection*{Community feedback}
Users of the site submit their rating of news outlets directly on the site. This
type of bias measurement is not, unlike the blind survey, free of prejudices and
is taken with a grain of salt.
\subsubsection*{Editorial review}
This refers to what AllSides themselves think the bias of a source is.
\subsubsection*{Independent research}
The opinion of other media outlets and/or other sources - political or
non-political - is used. This is mostly used as starting point before some of
the better methods are used.