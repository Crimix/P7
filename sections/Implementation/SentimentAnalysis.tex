\section{Sentiment Analysis}

Sentiment analysis is used to determine the opinions of people regarding
specific topics, such as a movie or a person. This has become especially useful
with the increasing usage of social media, as people increasingly discuss and
post their opinions. It is done by gathering and parsing the opinions which can
be found online into tokens. Extracting the features and using a classification
model to predict the sentiment.

\subsection{Feature extraction} 
Feature extraction is used after the tokenization to choose which words are to
be used to determine the sentiment. Depending on how simplistic the feature
extraction is, a lot of sentiment can be lost. For instance there is a big
difference between person 1 saying ``I love this movie'' and person 2 saying
``I LOVE this movie!''. In this example both the capitalization of ``love'' and
the exclamation mark help empathize that person 2 seems to have a stronger
positive sentiment towards the movie. But these features can easily be lost if
the text is simplified. It is also beneficial to consider emoticons such as
``:-)'' as valuable tokens, as it is used to show sentiment clues.




\subsection{Classification}


 By
gathering and parsing the opinions which can be found online, extracting
the features, and using a classification model

people display their opinions using social media
a lot of opinions are needed to get an accurate picture
can be handled by a sentiment analysis



Baseline algorithm:
tokens
stemming which can destroy some of the sentiment
feature extraction
classification:
- Naive Bayes
- Support Vector Machines
- Neural Networks

