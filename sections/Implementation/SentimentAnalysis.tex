\section{Sentiment Analysis}\label{SEC:SentimentAnalysis}

Sentiment analysis is used to determine the opinions of people regarding
specific topics, such as a movie or a person. This has become especially useful
with the increasing usage of social media, as people increasingly discuss and
post their opinions. It is done by gathering and parsing the opinions which can
be found online into tokens. Extracting the features and using a classification
model to predict the sentiment.

\subsection{Feature extraction} 
Feature extraction is used after the tokenization to choose which words are to
be used to determine the sentiment. A lot of sentiment can be lost if the
feature extracter is simplistic. There are several ways which sentiment can be
expressed in text, such as: \Source

\begin{itemize}
  \item Capitalization 
  \item Lengthening
  \item Emoticons
  \item Punctuation
  \item Stopwords
  \item Negation
\end{itemize}

\fix{}{We are stil in the analysis, where are all the sources}

For instance there is a big difference between person 1 saying ``I love this
movie'' and person 2 saying ``I LOVE this movie!!!!''. In this example both the
capitalization of ``love'' and the exclamation mark help empathize that person
2 seems to have a stronger positive sentiment towards the movie. But these
features can easily be lost if the text is simplified. It is also beneficial to
consider emoticons such as ``:-)'' as valuable tokens, as it is used to show
sentiment clues.\\

An example of lengthening is the difference in sentiment between ``huuuungry'',
``huuungry'' and ``hungry". The differences in sentiment between the first two
variations of ``hungry" are probably minimal, but there is a clear difference
between those and the last version. These features can be compressed by looking
for tokens with more than two of the same letters repeated and removing the
repeated letters. This will make the loss of sentiment minimal, while limiting
the varitions of the same word.\nl

The removal of stopwords, which are words that do not contain useful
information, is dependent upon which classification model is used. Some models
such as a Naive Bayes classifier will benefit from having the stopwords removed
as they are useless information. Neural Networks work very differently from
Naive Bayes classifiers, and are able to look past the noise given enough
training data \Source.\nl

Negation also needs to be considered, as it can completely change the sentiment
of the text. A simple way to handle negation is to add a tag such as ``NEG\_'' as
a prefix for tokens after a word such as ``not'' and ``arent''.\nl 

An example of how feature extraction is likely to look like can be seen in
\autoref{tab:feature}.

\begin{table}[H]
\centering
\begin{tabular}{|p{6cm}|p{8cm}|}
\hline
Text & Features \\ \hline
I LOVED the movie but I am sooooo hungry! & 
``I'' ``LOVED'' ``movie'' ``I'' ``soo'' ``hungry'' ``!''
\\ \hline 
I hate this song, why do they keep playing it? &
``I'' ``hate'' ``NEG\_song'' ``NEG\_why'' ``NEG\_keep'' ``NEG\_playing'' ``?''  
\\ \hline
\end{tabular}
\caption{An example of transforming text into features}
\label{tab:feature}
\end{table}


\subsection{Classification}

The final part of sentiment analysis is the classification task. This task
consist of predicting the sentiment based on the features which were extracted
earlier. The accuracy of the predictions are based on labeled data that was that
provided as training data, as well as the model that is used. We will not go
indepth about the models that are available but will describe the models that
are of interest, a cursory description of the different models can be seen at
\citep{Classification}.

\subsubsection{Training}
The training data is an important part of every model. Classification
performed best by using the training method called supervised learning.
There are two other types of learning methods, namely reinforcemet and
unsupervised learning, these are better suited for other types of tasks such as
clustering and robot behaviour \Source.

\begin{itemize}
  \item \textbf{Supervised learning} is when the training data is paired with a
  given result, for instance: ``This is a great movie'' can be paired with
  ``positive'' as a result. This allows the model to verify its prediction
  whether it guesses correct or not. This is useful as it provides an easy way
  to validate the accuracy of the model. The main problem is that most data is
  not labeled, which means that it has to be labeled by hand to be
  useful \Source.
  \item \textbf{Reinforcement learning} is similar to supervised learning with
  key difference being that the model is only given a ``reward'' when it
  performs the best action. This type of learning method among is useful when
  using robotic agent \citep{Reinforcement}.
  \item \textbf{Unsupervised learning} is different as the training data no
  longer includes a result, and as such can no longer evaluate the accuracy.
  Instead it useful for clustering data and discovering common elements \Source.
\end{itemize}

The training itself is dependent on the model, as different models train in
different ways. For instance training a Naive Bayes Classifier involves
adjusting the percentages of each token as they appear, while a Neural Network
adjusts the value of its weights. 

\subsubsection{Models}

There are a lot of different models which can be used for sentiment analysis,
see \autoref{sentiment}. Different models provide different levels of accuracy

\figx[1.0]{sentiment}{Accuracy of different sentiment analysis models on the
IMDB dataset \citep{Classification}}



There are two other methods, namely reinforcement and unsupervised learning,
whcih can be 

Reinforcement learning is similar to supervised learning with key difference
being that the model is only given a ``reward'' when it guesses the correct
action. 




Baseline algorithm:
tokens
stemming which can destroy some of the sentiment
feature extraction
classification:
- Naive Bayes
- Support Vector Machines
- Neural Networks

