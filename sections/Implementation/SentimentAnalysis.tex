\section{Sentiment Analysis}
Sentiment analysis is used to determine the opinions of people regarding
specific topics, such as movies or products. This has become especially useful
with the increasing usage of social media, as people increasingly discuss and
post their opinions online. Sentiment analysis is done by gathering opinions
online and extracting tokens which provide clues to the opinion. The tokens can
then be used with classification model to predict the sentiment.

\subsection{Feature extraction} 
Feature extraction is used after the tokenization, in order to choose which
words best determine the sentiment. A lot of the sentiment can be lost in the
feature extracter, as the opinions are much more apparent in text, than as a
series of tokens. There are several ways which sentiment can be expressed in
text, such as\citep[Ch.3-4]{Sentiment}:

\begin{itemize}
  \item Capitalization 
  \item Lengthening
  \item Emoticons
  \item Punctuation
  \item Stopwords
  \item Negation
\end{itemize}

For instance there is a big difference between person 1 saying ``I love this
movie'' and person 2 saying ``I LOVE this movie!!!!''. In this example both the
capitalization of ``love'' and the exclamation mark help empathize that person
2 seems to have a stronger positive sentiment towards the movie. But these
features can easily be lost if the text is simplified. It is also beneficial to
consider emoticons such as ``:-)'' as tokens, as they are used to show
emotions.\\

An example of lengthening is the difference in sentiment between ``huuuungry'',
``huuungry'' and ``hungry". The differences in sentiment between the first two
variations of ``hungry" are probably minimal, but there is a clear difference
between those and the last version. These features can be compressed by looking
for tokens with more than two of the same letters repeated and removing the
repeated letters. This will make the loss of sentiment minimal, while limiting
the varitions of the same word.\nl

It is useful to remove stopwords, which are words that do not contain useful
information. As they can appear in every type of sentence they can make
classification more difficult, though it also mainly depends on the model,
see \autoref{subsub:Models}.\nl

Negation also needs to be considered, as the sentiment can be completely changed
with a word ``not''. Since sentiment is extracted from tokens, it is difficult
to judge how each tokens relate to others, when see individually. A simple way
to handle negation is to prefix tokens with a tag such as ``NEG\_'' after a
word like ``not'' and ``arent''.\nl

An example of how feature extraction is can look like can be seen in
\autoref{tab:feature}.

\begin{table}[H]
\centering
\begin{tabular}{|p{6cm}|p{8cm}|}
\hline
Text & Features \\ \hline
I LOVED the movie but I am sooooo hungry! & 
``I'' ``LOVED'' ``movie'' ``I'' ``soo'' ``hungry'' ``!''
\\ \hline 
I don't like this song, why do they keep playing it? &
``I'' ``dont'' ``like'' ``NEG\_song'' ``NEG\_why'' ``NEG\_keep'' ``NEG\_playing''
``?'' \\ \hline
\end{tabular}
\caption{An example of transforming text into features}
\label{tab:feature}
\end{table}

There are a couple other methods that can affect the how much sentiment is
retained in the text, such as stemming and using n-grams.

\subsubsection{Stemming}
Stemming can be used when performing feature extraction. It is used to collapse
a word into its base, for instance ``finding'' can be stemmed to ``find''.
Stemming is both useful and destructive. It allows for reduction in the number
of terms as they can be collapsed. It is also destructive as sentiment clues
can be lost easily. A lot of a words meaning can be tied to the ending of the
word, such as ``captivating'' and ``captive''. By using the Porter Stemmer
Algorithm both words will collapse to ``captiv'' which removes what
differentiate them. There are other Stemmer algorithms but each provide similar
problems. Stemming can also be useful as it can reduce the vocabulary, instead
of remembering both ``captivating'' and ``captive'', it now only needs to
remember ``captiv''. This can be useful for small datasets where each token is
important, as it can reduce the vocabulary size and sharpen the result
\citep[Ch 3.b]{Sentiment}.

\subsubsection{N-Grams}
N-grams are sequence of 'n' items and determines how much context is stored
from a piece of text. An example of the 3 first n-grams can be seen in
\autoref{tab:ngram}. 

\begin{table}[H]
\centering
\begin{tabular}{|l|l|}
\hline
Text & ``I love horror movies'' \\ \hline
Unigrams (1) &
``I'' ``love'' ``horror'' ``movies''
\\ \hline 
Bigram (2) &
``I love'' ``love horror'' ``horror movies''
\\ \hline
Trigram (3) &
``I love horror'' ``love horror movies''
\\ \hline
\end{tabular}
\caption{An example of different N-Grams}
\label{tab:ngram}
\end{table}

It becomes obvious as sequence becomes longer that more context is stored in the
token, but it also becomes less likely that a similar token is found in another
text. This means that without a sizable sample it can be difficult to achieve
precision with longer n-grams.

\subsection{Classification}
The final part of sentiment analysis is the classification task. This task
consist of predicting the sentiment based on the features which were extracted
earlier. The accuracy of the predictions are based on labeled data that was that
provided as training data, as well as the model that is used. We will not go
indepth about the models that are available but will describe the models that
are of interest, a cursory description of the different models can be seen at
\citep{Classification}.

\subsubsection{Training}
The training data is an important part of every model. There are three different
training methods, each suitable for different tasks. 

\begin{itemize}
  \item \textbf{Supervised learning} is when the training data is paired with a
  given result, for instance: ``This is a great movie'' can be paired with
  ``positive'' as a result. This allows the model to verify its prediction
  whether it guesses correct or not \citep[Ch. 7.0]{MIBook}\KT. This is useful
  as it provides an easy way to validate the accuracy of the model. The main problem
  is that most data is not labeled, which means that it has to be labeled to be
  useful.
  \item \textbf{Reinforcement learning} is similar to supervised learning with
  key difference being that the model is only given a ``reward'' when it
  performs the best action. This type of learning method among is useful when
  using robotic agent \citep{Reinforcement}.
  \item \textbf{Unsupervised learning} is different as the training data no
  longer includes a result, and as such can no longer evaluate the accuracy.
  Instead it useful for clustering data and discovering common elements
  \citep[Ch. 11.1]{MIBook}\KT.
\end{itemize}

The model which is most used for sentiment analysis is the supervised learning
approach, as it is most useful for classification tasks.

\subsubsection{Models}\label{subsub:Models}

There are a lot of different models which can be used for sentiment analysis. 
Different models provide different levels of accuracy as can be seen in
\autoref{sentiment}, which shows a couple of different models on the same IMDB
dataset with 3 different outputs\citep{Classification}.

\figx[1.0]{sentiment}{Accuracy of different sentiment analysis models on the
IMDB dataset \citep{Classification}}

The different models which are shown in the graph are Naive Bayes using 





Baseline algorithm:
tokens
stemming which can destroy some of the sentiment
feature extraction
classification:
- Naive Bayes
- Support Vector Machines
- Neural Networks



%needs something about how they work, and how stopwords affect them
 is dependent upon which classification model is used. Some models
such as a Naive Bayes classifier will benefit from having the stopwords removed
as they are useless information. Neural Networks work very differently from
Naive Bayes classifiers, and are able to look past the noise given enough
training data \Source.\nl