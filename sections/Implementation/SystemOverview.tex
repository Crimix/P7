\chapter{System Overview} \fix{}{the chapter (Development in general) should
also explain your design decisions} The developed solution consists of multiple different parts, including a Queue
Server, a webpage front end, a database, and a set of classification algorithms.
As such, this chapter will be used to give an overview of these different parts,
and how they interact with each other.

\section{Overview}
The developed system consists of four individual parts, the front end web
service, the database, the classification system, and a task queue which handles
users requests. These elements are developed using three different tools or
frameworks: Laravel, C\# and MySQL. The overall architecture of the system can
be seen in \autoref{P7_SystemOverview} where the arrows imply that different parts
are communicating directly with each other.
 
\figx[0.62]{P7_SystemOverview}{System architecture of the BubbleBuster system}

\subsection{GUI} %API, Interfaces, Data Transfer, System Logic The
The part of the system developed in Laravel makes up the front end of the
system, and is used as the user's entry-point to the system. From here, the user can
make requests to analyze their filter bubble, and view the data derived from
the analysis. As the solution is centered around a web application, it is
scaleable such that it can support input from multiple users at a time.\nl

\subsection{Database and API}
The database API is developed using the Laravel framework, is used to execute
commands on the tables on the MySQL database. This is further described in
\autoref{DatabaseAPI}.

\subsection{Queue Server}
The queue server is used to handle and schedule requests from all the users
using the web application. In order to handle these requests we have developed
an api in C\# which is used as the entry point. This is further described in
\autoref{queueAPI}. 

\subsection{Worker}
The core functionality of the system is to determine users political filter
bubble. This functionality is contained in the Worker. Whenever the queue server
receives a request to classify a user, it calls the methods contained in the
worker. This is further described in \autoref{workerLabel}.
