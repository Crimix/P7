\chapter{System Overview}\label{ch:sysview}
The developed solution consists of multiple different parts, including a Queue
Server, a webpage front-end, a database, and a set of classification models. As
such, this chapter will be used to give an overview of these different parts,
and how they interact with each other.

\section{Overview}
The developed system consists of four individual parts, the front end web
service, the database, the classification system, and a task queue which handles
users requests. These elements are developed using three different tools or
frameworks: Laravel, C\# and MySQL. The overall architecture of the system can
be seen in \autoref{P7_SystemOverview} where the arrows imply that different
parts are communicating directly with each other.
 
\figx[0.62]{P7_SystemOverview}{System architecture of the BubbleBuster system.}

\subsection{GUI} %API, Interfaces, Data Transfer, System Logic The
The part of the system developed in Laravel makes up the front-end and is the
user's entry-point to the system. From here, the user can make requests to
analyse a filter bubble using a Twitter username, and view the data derived from
the analysis. The solution is centred around a web application; thus it is
scalable as it can support input from multiple users at a time, it is also
possible to acquire more servers if needed.

\subsection{Database and API}
The database \ac{API} is developed using the Laravel framework described in
\autoref{sec:laravel} and is used to execute commands in the tables on the MySQL
database. As a safeguard, we have implemented authorisation using Laravel
Passports, which requires the request to include an authorisation header with a
\textc{bearer} token. We have chosen this approach, as we want to ensure that
only authorised sources can edit the stored data. The implementation of the
database \ac{API} is described in \autoref{DatabaseAPI}.

\subsection{Queue Server and API}
The queue server is used to handle and schedule requests from all the users
using the web application. To handle these requests we have developed an API
using C\# it serves as the server entry point. As this API is used to start
tasks unconditionally, it requires a safeguard in the form of an authorization
system, as hostile users would otherwise be able to overload the system with
fake requests. While we deem this to be necessary, the authorization system is
not yet implemented, as is discussed in \autoref{disc:auth}. The implementation
of the queue server and its API is further described in \autoref{queueAPI}.

\subsection{Worker}
The core functionality of the system is to determine a user's, and those they
follow, political filter bubbles. The Worker contains this functionality.
Whenever the queue server receives a request to classify a user, it calls the
methods contained in the worker. This is further described in
\autoref{workerLabel}.

\subsection{Multithreading}\label{subs:multithread}
To optimise the system performance, we have chosen to use multiple threads for
actions which can easily be split into parts. These actions are the retrieval of
tweets and the following classification of tweets. For tweet retrieval, we make
use of three threads, where each thread is working on retrieving the tweets of a
single user. As such, we can process three users concurrently. Based on our
general testing, this was chosen to be the optimal amount, as Twitter
occasionally terminated our requests if we had too many running concurrently.\\
For tweet classification, we make use of five threads, working together to
analyse the tweets from a single person at a time. This method means that a
single person's tweets get split into five parts, analysed, and the results
combined into a single classification. Splitting the classification into five
threads greatly improved performance, documented in \autoref{test:multithread}.

\subsection{Singleton Class Structure}
Many of the classes in the developed system are defined as singletons, meaning
that only a single instance is ever created. We have chosen this approach, as it
offers slight improvements over static classes, as we ensure that classes are
only stored on the stack if they are ever used. A potential problem with this
approach is that we use a multithreaded method for analysing tweets, which
causes problems with accidentally trying to create multiple instances of the
classes. However, this is solved by using double-checked locking to ensure that
only one instance of a singleton exists.

\subsection{Web Communication}
When retrieving data from the Twitter API, we have chosen to make use of
\textc{HttpWebRequest} instead of \textc{WebClient} as it allows us to have more
control over the requests. As an example the \textc{HttpWebRequest} have a
\textc{Timeout} property which we can modify to make the request wait longer
before failing. Otherwise, this would only be possible if we extended
\textc{WebClient} and made our web client. But it was decided that this would be
unnecessary, as it would imply extending a class only to change a single
property.

\subsection{Classification Models}
Two different models are developed to help determine a Twitter user's political
leaning; this is explained in \autoref{cha:classification}. The first model uses
a Bag-of-Words approach where a text get analysed for emotional, political, and
news-related keywords. This approach is based partly on the methodology
described by \cited{sarlan2014twitter}.
The second approach uses a Naive Bayesian Network to determine a user's
political affiliation based on a set of training data built from the Twitter
data from American politicians.




