\chapter{Database \acs{API}}\label{DatabaseAPI}
After analysing users, we want to store that data such that we do not have to
retrieve or analyse the tweets again. The system makes use of a
database, which the workers can use to store and access the saved data. This
design choice is especially important, as we make use of the user's Twitter
account to retrieve data from the Twitter \ac{API}, which have a limited amount
of API requests, as described in \autoref{cha:twitterAPI}. This section
is used to describe how we access and store this data and how we make sure that
it is only accessible to authorised users.\nl

In \autoref{sec:laravel}, we start by describing the Laravel framework, which
has been used to develop the front-end and the database \ac{API}. In
\autoref{DBDesign}, we continue by describing the design of our database and the
data it is used to store. Then in \autoref{sec:APIDesign}, we end by describing
the implementation of the Database \ac{API} and the different end-points we
have defined for accessing data.

\section{Laravel Framework}\label{sec:laravel}
Laravel is a framework written in PHP that enables the creation of websites and
\ac{REST} web services, abstracting many underlying implementation details
including database access, user authentication and routing.\nl

Laravel's structure resembles that of the \ac{MVC} design pattern, with models
representing and managing database entities, views handling the presentation
layer without implementation logic and controllers handling the main parts of
the logic. When needed, other types of layers can be added. This
includes helper classes and packages that extend the functionality of
Laravel.\nl

The main reason for choosing Laravel is that the framework offer all the
necessary tools in a single framework and that the group has existing
experience with the framework.
% 
% 
% The main reason for choosing Laravel was group experience with various
% frameworks and a preference for how Laravel does things.

\section{Database Design}\label{DBDesign}
Laravel supports multiple \acp{DBS} \citep{LaravelDBS} and due to the use of
Laravel's abstraction, our system should be able to run on any of the supported
database systems without any changes on our part. As such, we have opted to use
MySQL while developing and testing, as it is the default for Laravel, and
because the group has existing experience with the system.\nl

The system uses ten tables, of which seven are used for authentication purposes,
and one tracks database migrations. The remaining two tables are
\textc{twitters} and \\\textc{twitter\_twitter}. This is shown partially with
example data in \autoref{TwittersDBTable} and \autoref{TwitterTwitterDBTable}.\nl

The columns \textc{created\_at} and \textc{updated\_at} have been left out to
fit within the width of the page, while still remaining readable.

\begin{table}[H]
\centering
\cc{
\resizebox{1.2\textwidth}{!}{%
\begin{tabular}{| l | l | l | l | l | l | l | l	| l | l |}
\hline
\textbf{\underline{id}} & \textbf{twitter\_name} & \textbf{twitter\_id} &
\textbf{analysis\_val} & \textbf{mi\_val} & \textbf{sentiment\_val} &
\textbf{media\_val} & \textbf{tweet\_count} & \textbf{protect} &
\textbf{processed} \\
\hline
\underline{224} & theTonaldDrump & 1234567 & 9.7323990 & 8.9295123 & -9.1618950
& -3.1275883 & 472 & 0 & 1 \\
\hline
\underline{225} & TayaDickinson & 914292 & 6.3012218 & 2.3406686 & -7.4462646 &
-6.2828786 & 414 & 0 & 0 \\
\hline
\underline{226} & BrodyHahn & 200519 & 1.9456733 & -7.2432034 & 4.1452475 &
-8.1468677 & 325 & 0 & 0 \\
\hline
\end{tabular}}
}
\caption{\textc{twitters} table containing Twitter users and their determined
values.}
\label{TwittersDBTable}
\end{table}

The table \textc{twitters} is used for storing data about analysed Twitter
users, such that we can reference this data if we should receive a request. This
reduces overhead, as we can reuse data. One important aspect is that we need to
respect a user's privacy. We have introduced the \textc{protect}
boolean to indicate if only the Twitter user themself, or a follower, is allowed
to retrieve it. Although this data is present in the database, the actual 
protection is not implemented. Details about this can be found in
\autoref{sec:twitterProtect}.\nl

The \textc{processed} column is used to distinguish between users who have been
fully analysed, along with all the users they follow, versus those who have been
processed as followed accounts. In the example data in
\autoref{TwittersDBTable}, this distinction can be seen where ``theTonaldDrump''
is the only processed entry, while ``TayaDickinson'' and ``BrodyHahn''
are only analysed because they are followed by ``theTonaldDrump''.


\begin{table}[H]
\centering
\begin{tabular}{| l | l |}
\hline
\textbf{\underline{twitter\_id}} & \textbf{\underline{follows\_id}} \\
\hline
\underline{224} & \underline{225} \\
\hline
\underline{224} & \underline{226} \\
\hline
\end{tabular}
\caption{\textc{twitter\_twitters} table which keeps track of followers.}
\label{TwitterTwitterDBTable}
\end{table}

The table \textc{twitter\_twitters} is used as a link table and keeps track
of the followers. In the example data in \autoref{TwitterTwitterDBTable}, it is 
shown how we keep track of the users followed by ``theTonaldDrump''.

\section{\acs{API} Design}\label{sec:APIDesign}
To access the database described in \autoref{DBDesign}, we have
developed a \ac{REST} \ac{API} using Laravel. This
\ac{API} functions as a layer on top of the database, to which we can make
requests and retrieve data through the use of authorised \ac{HTTP} requests.

\subsection{\acs{API} Endpoints}
To enable external applications to interact with the \ac{API} we must
define a number of endpoints that covers the data and the functions that can be
performed. These are based on the following base \ac{URL}:\nl

\say{http://localhost:8000/api/}\nl

In this example, the domain for the \ac{API} is \textc{localhost} on port 8000,
as this is the default for locally hosted Laravel applications. If the application was to
be hosted on an external server, it would have a domain name and would
preferably use the \textc{https://} scheme and run on port 443.\nl

The defined endpoints are described in \autoref{DBAPIEndpointTable}. It is worth
noting that \textc{boolean} is either \ttt{0} or \ttt{1}, so \ttt{true} and
\ttt{false} are not permitted as input. 

\begin{table}[H]
\centering
\begin{tabular}{| l | l | l |}
\hline
\textbf{METHOD} & \textbf{Endpoint} & \textbf{Required Data} \\
\hline
POST & /twitter                       &  \ttt{string} twitter\_name \\
~    & ~                              &  \ttt{int} twitter\_id \\
~    & ~                              &  \ttt{double} analysis\_val \\
~    & ~                              &  \ttt{double} mi\_val \\
~    & ~                              &  \ttt{double} media\_val \\
~    & ~                              &  \ttt{double} sentiment\_val \\
~    & ~                              &  \ttt{int} tweet\_count \\
~    & ~                              &  \ttt{boolean} protect \\
\hline
\multicolumn{3}{| l |}{\textit{%
Adds a single user's results to the database and returns the}} \\
\multicolumn{3}{| l |}{\textit{%
record ID of the created user, if creation was successful.}} \\
\multicolumn{3}{| l |}{} \\
\hline
PUT  & /twitter/add\_follower         &  \ttt{int} record\_id \\
~    & ~                              &  \ttt{int} follows\_id \\
\hline
\multicolumn{3}{| l |}{\textit{%
Creates a link between two record IDs specifying that }\ttt{record\_id}} \\
\multicolumn{3}{| l |}{\textit{%
follows }\ttt{follows\_id} \textit{on Twitter.}} \\
\multicolumn{3}{| l |}{} \\
\hline
PUT  & /twitter/finalize              &  \ttt{int} record\_id \\
~    & ~                              &  \ttt{string} request\_id \\
\hline
\multicolumn{3}{| l |}{\textit{%
Marks the user as processed and then forwards the} \ttt{request\_id}} \\
\multicolumn{3}{| l |}{\textit{%
to the \acs{GUI} server to notify it of the completion.}} \\
\multicolumn{3}{| l |}{} \\
\hline
GET  & /twitter/has/\{twitterName\}   &  \\
\hline
\multicolumn{3}{| l |}{\textit{%
Looks for a recently processed result for the given Twitter name.}} \\
\multicolumn{3}{| l |}{\textit{%
If found the record ID is returned.}} \\
\multicolumn{3}{| l |}{} \\
\hline
GET  & /twitter/has-id/\{twitterId\}  &  \\
\hline
\multicolumn{3}{| l |}{\textit{%
Looks for a recent result for the given Twitter ID.}} \\
\multicolumn{3}{| l |}{\textit{%
If found the record ID is returned.}} \\
\multicolumn{3}{| l |}{} \\
\hline
GET  & /twitter/\{recordId\}          &  \\
\hline
\multicolumn{3}{| l |}{\textit{%
Returns the full result for the given record ID if it exists.}} \\
\multicolumn{3}{| l |}{} \\
\hline
\end{tabular}
\caption{\ac{API} endpoints for the database back-end.}
\label{DBAPIEndpointTable}
\end{table}
 

\subsection{Authorisation}
To make sure that data is only available to authorised users, we make
use of Laravel Passport, which is a token-based authorisation system based on
the OAuth2 standard \citep{LaravelPassport}.\nl

To make a request to the database, the user has to supply a token, which is an
encrypted string. This token is then verified by comparing it to a list of
active tokens on the database. The token verification is handled by Laravel
Passport.\nl

Among the options that Laravel Passport provides for authentication, we opted
for the one called ``client credentials'' which is meant for machine--to--machine
communication, as opposed to user--to--machine authentication
\citep{LaravelPassportClientCredentials}.

The base \ac{URL} for authentication requests is the same as for the \ac{API}
except not prefixed with \ttt{/api}.\nl

To retrieve a token, a worker must make a request to the
``/oauth/token'' endpoint, while supplying a valid client ID and client
secret, along with the grant type, which is always ``client\_credentials''. The
request specification is documented in \autoref{DBAPIOAuthEndpointTable}.\nl

If the supplied credentials are valid as per the information stored in the
database, the request responds with a long-lived token that uniquely identifies
the client, be it a worker or the \ac{GUI}. This token must then be sent in the
header of every request as a \textc{bearer} token.\nl

\begin{table}[H]
\centering
\begin{tabular}{| l | l | l |}
\hline
\textbf{METHOD} & \textbf{Endpoint} & \textbf{Required Data} \\
\hline
POST & /oauth/token  & \ttt{string} grant\_type \\
~    & ~             & \ttt{int} client\_id \\
~    & ~             & \ttt{string} client\_secret \\
\hline
\end{tabular}
\caption{\ac{API} endpoints for authentication.}
\label{DBAPIOAuthEndpointTable}
\end{table}


