\chapter{Database \ac{API}}
As a part of the overall application, we need a database to store information
used by the different workers. This is important, as we make use of the
user's Twitter account to retrieve data from the Twitter \ac{API}. As such, this
section is used to describe how we access and store this data, and how we
make sure that it is only accessible to authorized users.

\section{Database Design}\label{DBDesign}
We have opted to use MySQL as our \ac{DBS}, structuring our data in a
standardized relational database. In order to store the general data, we have
two tables; \textc{Users} and \textc{Twitters}, as seen in
\autoref{TwitterDBTable} and \autoref{UserDBTable}.

\begin{table}[H]
\begin{tabular}{l | l | l | l | l | l | l | l	}
\textbf{id} & \textbf{name} & \textbf{twitterID} & \textbf{pol\_var} &
\textbf{lib\_var} & \textbf{protect} & \textbf{created\_at} & \textbf{updated\_at}
\\\hline 1 & John123 & 4893758 & 4.1 & 3.5 & 1 & 03-11-2017 & 03-11-2017 
\\\hline 2 & MMan & 9302847 & 2.9 & 1.7 & 0 & 03-11-2017 & 03-11-2017 
\\\hline
\end{tabular}
\caption{Table containing twitter users and their determined values.}
\label{TwitterDBTable}
\end{table}

\textc{Twitters} is used for storing data about analyzed
twitter users, such that we can reference this data if we should receive
a request. This allows to reduce overhead, as we can reuse data. One important
aspect is that we need to respect a user's privacy. As such, we have introduced
the ``protect'' boolean to indicate if only the Twitter user, which the result
belong to, is allowed to retrieve it.

\begin{table}[H]
\begin{tabular}{l | l | l | l | l }
\textbf{id} & \textbf{name} & \textbf{password} &
\textbf{created\_at} & \textbf{updated\_at}\\\hline
1 & Worker\_1 & xyaAngeaG93AFk & 03-11-2017 & 03-11-2017
\\\hline
2 & Worker\_2 & oLj7Ft89AFSA & 03-11-2017 & 03-11-2017 \\\hline
\end{tabular}
\caption{Table containing twitter users and their determined values.}
\label{UserDBTable}
\end{table}

\textc{Users} is used for authenticating workers. This means that each time a
user uses our application, that instance is saved as a worker. This worker has
an automatically generated name and password.

\section{\ac{API} Design}
In order to access the database described in \autoref{DBDesign}, we have
developed an \ac{API} using the Laravel web development framework. This \ac{API} functions
as a layer on top of the database, to which we can make requests and retrieve
data through the use of \ac{HTTP} requests.

\subsection{\ac{API} Endpoints}
As the \ac{API} uses \ac{HTTP} requests, we have to define a number of
endpoints which are available to be used by external applications. These are based on the following
request:\nl

\say{http://localhost:8000/api/}\nl

In this example, the domain for the \ac{API} is localhost on port 8000, as this is
the default for locally hosted Laravel applications. In order to access the
data, we have the following endpoints:

\begin{table}[H]
\begin{tabular}{l | l | l}
\textbf{TYPE} & \textbf{Endpoint} & \textbf{Required Data} \\\hline
POST & api/twitter   & 	\textc{string} name\\
~    & ~			 &  \textc{int} twitterID\\
~    & ~			 &  \textc{double} pol\_var\\
~    & ~			 &  \textc{double} lib\_var\\
~    & ~			 &  \textc{boolean} protect				      
\\\hline
POST & api/user      & \textc{string} name\\
~    & ~			 &  \textc{string(Encrypted)} password					      
\\\hline
POST & oauth/token 	& \textc{grant\_type} password					  \\
~    & ~			& \textc{client\_id} 2					\\
~    & ~			& \textc{client\_secret} XIpLV3Jl\ldots					\\
~    & ~			& \textc{username} Worker					\\
~    & ~			& \textc{password(Encrypted)} ERExjM0rVvcC7zfJvSbU					\\
~    & ~			& \textc{scope} *					     
\\\hline
GET & api/user/\{id\} &
\\\hline
GET & api/twitter/\{twitterID\} &
\\\hline


\end{tabular}
\caption{Table containing twitter users and their determined values. Note that
the variables in the token request are example values}
\label{APIEndpointTable}
\end{table}
 

\subsection{Authorization}
In order to make sure that data is only available to authorized users, we make
use of Laravel Passports, which is a token-based authorization system. As such,
in order to make a request to the database, the user has to supply a token,
which is an encrypted string. This token is then verified by comparing it to a
list of active tokens on the database.\nl

In order to retrieve a token, a worker must make a request to the
``oauth/token'' endpoint, while supplying a username, password, and client
secret. The username and password are compared to those stored in the user's
table on the database, and the client\_secret is a static string.\nl

If the supplied credentials are found on the database, the request responds with
a token, which uniquely identifies the worker which made the request. When
making requests for data on the database, the worker must supply this token,
which is used to authenticate it.\nl

\fix{}{This chapter was written while I was half blind, and very annoyed with my
contact lenses. As such, it needs some more info in some parts (more tables,
better description of requests and responses). Also proof read (half-blind,
remember?). Also, add references to tables.}


















