\chapter{Database \acs{API}}\label{DatabaseAPI}
After analysing a user, we want to store that data, such that we do not have to
retrieve or analyze the tweets again. As such, the sytem makes use of a
database, which the Workers can use to store and access the saved data. This
design choice is especially important, as we make use of the user's Twitter
account to retrieve data from the Twitter \ac{API}, which have a limited amount
of API requests, as described in \autoref{cha:twitterAPI}.
As such, this section is used to describe how we access and store this data, and
how we make sure that it is only accessible to authorized users.

\section{Laravel Framework}\label{sec:laravel}
Laravel is a framework written in PHP that enables the creation of websites and
\ac{REST} web services, abstracting many underlying implementation details
including database access, user authentication and routing.\nl

Laravel's structure resembles that of the \ac{MVC} design pattern, with models
representing and managing database entities, views handling the presentation
layer without implementation logic, and controllers handling the main parts of
the logic. When needed, however, other types of layers can be added. This
includes helper classes and packages that extend the functionality of
Laravel.\nl

The main reason for choosing Laravel was previous knowledge with various
frameworks and a preference for how Laravel does things.

\section{Database Design}\label{DBDesign}
While Laravel supports multiple \acp{DBS}\citep{LaravelDBS} we have opted to use
MySQL while developing and testing. Due to the use of Laravel's abstraction our
system should be able to run on any of the supported database systems without
any changes on our part.\nl

The system uses 10 tables of which 7 are used for authentication purposes and
one tracks database migrations. The remaining two tables are \textc{twitters}
and \textc{twitter\_twitter}, shown in part with example data in
\autoref{TwittersDBTable} and \autoref{TwitterTwitterDBTable} respectively.\nl

The columns \textc{created\_at} and \textc{updated\_at} have been left out here
due to size constraints on the pages.

\begin{table}[H]
\centering
\resizebox{1.1\textwidth}{!}{%
\begin{tabular}{| l | l | l | l | l | l | l | l	| l | l |}
\hline
\textbf{\underline{id}} & \textbf{twitter\_name} & \textbf{twitter\_id} &
\textbf{analysis\_val} & \textbf{mi\_val} & \textbf{sentiment\_val} &
\textbf{media\_val} & \textbf{tweet\_count} & \textbf{protect} &
\textbf{processed} \\
\hline
\underline{224} & theDonaldDrumpf & 1234567 & 9.7323990 & 8.9295123 & -9.1618950
& -3.1275883 & 472 & 0 & 1 \\
\hline
\underline{225} & TayaDickinson & 914292 & 6.3012218 & 2.3406686 & -7.4462646 &
-6.2828786 & 414 & 0 & 0 \\
\hline
\underline{226} & BrodyHahn & 200519 & 1.9456733 & -7.2432034 & 4.1452475 &
-8.1468677 & 325 & 0 & 0 \\
\hline
\end{tabular}}
\caption{Table containing twitter users and their determined values.}
\label{TwittersDBTable}
\end{table}

The table \textc{twitters} is used for storing data about analysed
twitter users, such that we can reference this data if we should receive
a request. This allows to reduce overhead, as we can reuse data. \fix{One
important aspect is that we need to respect a user's privacy. As such, we have
introduced the ``protect'' boolean to indicate if only the Twitter user, which the result
belong to, is allowed to retrieve it.}{We need to make sure we've reflected on
this elsewhere since it wasn't actually implemented.}\nl

The \textc{processed} column is used to distinguish between users who have been
fully analysed along with all the users they follow, versus those who have been
processed as followed accounts. In the example data in \autoref{TwittersDBTable}
this distinction can be seen as ``theDonaldDrumpf'' is the only processed entry,
while ``TayaDickinson'' and ``BrodyHahn'' are analysed because they are followed
by ``theDonaldDrumpf''.

\begin{table}[H]
\centering
\begin{tabular}{| l | l |}
\hline
\textbf{\underline{twitter\_id}} & \textbf{\underline{follows\_id}} \\
\hline
\underline{224} & \underline{225} \\
\hline
\underline{224} & \underline{226} \\
\hline
\end{tabular}
\caption{Table that keeps track of followers.}
\label{TwitterTwitterDBTable}
\end{table}

The table \textc{twitter\_twitter} is used as a link table, named as it is due
to convention more so than any requirements from the framework. It keeps track
of the followers and in the example data it is shown how we keep track of the
users followed by ``theDonaldDrumpf''.

\section{\acs{API} Design}
In order to access the database described in \autoref{DBDesign}, we have
developed a \ac{REST} \ac{API} using Laravel. This
\ac{API} functions as a layer on top of the database, to which we can make
requests and retrieve data through the use of authorised \ac{HTTP} requests.

\subsection{\acs{API} Endpoints}
To enable external applications to interact with the \ac{API} we must
define a number of endpoints that covers the data and functions that can be
performed on it. These are based on the following base \ac{URL}:\nl

\say{http://localhost:8000/api/}\nl

In this example, the domain for the \ac{API} is localhost on port 8000, as this is
the default for locally hosted Laravel applications. If the application was to
be hosted somewhere, it would have a domain and would likely run on port 443,
just as the scheme would be \textc{https://}.

We defined the following endpoints:

\begin{table}[H]
\begin{tabular}{| l | l | l |}
\hline
\textbf{TYPE} & \textbf{Endpoint} & \textbf{Required Data} \\\hline
POST & api/twitter   & 	\textc{string} name\\
~    & ~			 &  \textc{int} twitterID\\
~    & ~			 &  \textc{double} pol\_var\\
~    & ~			 &  \textc{double} lib\_var\\
~    & ~			 &  \textc{boolean} protect				      
\\\hline
POST & api/user      & \textc{string} name\\
~    & ~			 &  \textc{string(Encrypted)} password					      
\\\hline
POST & oauth/token 	& \textc{grant\_type} password					  \\
~    & ~			& \textc{client\_id} 2					\\
~    & ~			& \textc{client\_secret} XIpLV3Jl\ldots					\\
~    & ~			& \textc{username} Worker					\\
~    & ~			& \textc{password(Encrypted)} ERExjM0rVvcC7zfJvSbU					\\
~    & ~			& \textc{scope} *					     
\\\hline
GET & api/user/\{id\} &
\\\hline
GET & api/twitter/\{twitterID\} &
\\\hline


\end{tabular}
\caption{Table containing twitter users and their determined values. Note that
the variables in the token request are example values}
\label{APIEndpointTable}
\end{table}
 

\subsection{Authorization}
In order to make sure that data is only available to authorized users, we make
use of Laravel Passports, which is a token-based authorization system. As such,
in order to make a request to the database, the user has to supply a token,
which is an encrypted string. This token is then verified by comparing it to a
list of active tokens on the database.\nl

In order to retrieve a token, a worker must make a request to the
``oauth/token'' endpoint, while supplying a username, password, and client
secret. The username and password are compared to those stored in the user's
table on the database, and the client\_secret is a static string.\nl

If the supplied credentials are found on the database, the request responds with
a token, which uniquely identifies the worker which made the request. When
making requests for data on the database, the worker must supply this token,
which is used to authenticate it.\nl
