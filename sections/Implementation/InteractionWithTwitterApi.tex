\chapter{Twitter API Interaction}\label{cha:twitterAPI}
Following the decision to use Twitter as a source of data in
\autoref{sec:social-media-conclusion}, this chapter will be used to document how we interact
with the Twitter API, and how we handle the authorization required to access the
data. Whenever a user wants to request data, Twitter requires that request to
conform with the OAuth 1.0 authorization protocol \citep{OAuth1}.

\section{OAuth}
OAuth is an authorization protocol required by Twitter in order to be allowed to
use their \ac{API}.

\subsection{Request header}
In order for an \ac{HTTP} request to be valid, it has to
contain seven key/pair values in the header \citep{TwitterAPIAuth}:
\begin{itemize}
  \item oauth\_consumer\_key
  \item oauth\_nonce
  \item oauth\_signature
  \item oauth\_signature\_method
  \item oauth\_timestamp
  \item oauth\_token
  \item oauth\_version
\end{itemize}

\subsubsection*{Consumer key}
The consumer key is a unique token that identifies who makes the request.
This key is bound to the specific application.

\subsubsection*{Nonce}
The nonce is a unique token generated for each request. This way, it can be
determined if a request has been submitted multiple times. 

\subsubsection*{Signature}
The signature is used by Twitter to assure that the request has not been
modified after being sent. This works by encoding all information from the
initial request into a signature, and if this signature is not representative of
the request Twitter recieves, it is considered as invalid
\citep[sec 3.4]{OAuth1}.

% It is also used to verify the consumer key and the
% token, which is described below.
% This is done by generating a token using a cryptographic hash function on, among
% others, the request parameters and secret values.
% 
% \fix{}{This subsub is maybe a bit badly described. maybe help rewrite or
% something, maybe.}
 
\subsubsection*{Signature method}
The signature method for Twitter is always HMAC-SHA1. Any request without this
signature method is invalid. HMAC-SHA1 means that the cryptography hash function
used is \ac{SHA-1}.

\subsubsection*{Timestamp}
The timestamp is a string created together with the request. The string is the
number of seconds since the Unix epoch, which is 1/1/1970. It is used to
determine when the request was made.
If a request is marked as created before the Unix epoch, it is invalid.

\subsubsection*{Authorization token}
The token is used to identify the user who is making the request. Each user has
a unique token, which is kept secret. This token can be recieved by submitting a
users login details with twitter. As such, a user must be authorized before
making requests.

% that the application is allowed to
% interact with the user's account. It is generated by Twitter by going to
% apps.twitter.com and clicking on the application 
% \fix{}{maybe go in-depth on how this is done}

\subsubsection*{Version}
The version always has to be 1.0 for the API to accept the
request. This is because the Twitter \ac{API} only uses OAuth 1.0, as opposed to
OAuth 2.0.

\subsubsection{Creating the string}
When all the necessary tokens have been generated, the header string itself can
be created \citep{TwitterAPIAuth}. The finished header looks like this, where
\ldots\_val is replaced with the corresponding token:\nl

OAuth oauth\_consumer\_key="oauth\_consumer\_key\_val",\\
oauth\_nonce="oauth\_nonce\_val", \\
oauth\_signature="oauth\_signature\_val", \\
oauth\_signature\_method="HMAC-SHA1", \\
oauth\_timestamp="oauth\_timestamp\_val", \\
oauth\_token="oauth\_token\_val", \\
oauth\_version="1.0" \\