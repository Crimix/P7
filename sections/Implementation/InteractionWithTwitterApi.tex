\chapter{Interaction with the Twitter \ac{API}}
\fix{}{insert introduction}

\section{OAuth}
Oauth is an authorization protocol required by Twitter in order to be allowed to
use their \ac{API}. 

\subsection{Request header}
In order for an \ac{HTTP} request to be valid, it has to
contain seven key/pair values in the header:
\begin{itemize}
  \item oauth\_consumer\_key
  \item oauth\_nonce
  \item oauth\_signature
  \item oauth\_signature\_method
  \item oauth\_timestamp
  \item oauth\_token
  \item oauth\_version
\end{itemize}

\subsubsection*{Consumer key}
The consumer key is a unique token that identifies who makes the request.
This key is bound to the specific application.

\subsubsection*{Nonce}
The nonce is a unique token generated for each request. This way, it can be
determined if a request has been submitted multiple times. 

\subsubsection*{Signature}
The signature is used by Twitter to assure that the request has not been
modified after being sent. It is also used to verify the consumer key and the
token, which is described below.
This is done by generating a token from the request parameters and secret values.
\fix{}{maybe go in-depth on how this is done exactly.}
 
\subsubsection*{Signature method}
The signature method for Twitter is always HMAC-SHA1. 
, any request
without this signature method is invalid.

\subsubsection*{Timestamp}
The timestamp is a string created together with the request. The string is the
number of seconds since the Unix epoch, which is 1/1/1970. It is used to
determine when the request was made.
If a request is marked as created before the Unix epoch, it is invalid.

\subsubsection*{Token}
The token is used to assure that the application is allowed to
interact with the user's account. \fix{}{maybe go in-depth on how this is done}

\subsubsection*{Version}
The version always has to be 1.0 for the API to accept the
request.\cite{TwitterAPIAuth}

