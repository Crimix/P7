\chapter{Interaction with the Twitter \ac{API}}
\fix{}{This whole chapter should maybe be moved into something else. And it
shouldn't be in the development part as it is  not about how we used it but
just what it is.}
As described in \autoref{sub:twitterapi}, to interact with the Twitter
\ac{REST} \ac{API}, we have to use \ac{HTTP} GET requests. In order for
Twitter to be able to verify the authentity of the requests, it has to conform
to the OAuth 1.0 authorization protocol.
\fix{}{The introductions in this chapter are bad. sorry. please help fix}
\section{OAuth}
OAuth is an authorization protocol required by Twitter in order to be allowed to
use their \ac{API}.

\subsection{Request header}
In order for an \ac{HTTP} request to be valid, it has to
contain seven key/pair values in the header:
\begin{itemize}
  \item oauth\_consumer\_key
  \item oauth\_nonce
  \item oauth\_signature
  \item oauth\_signature\_method
  \item oauth\_timestamp
  \item oauth\_token
  \item oauth\_version
\end{itemize}

\subsubsection*{Consumer key}
The consumer key is a unique token that identifies who makes the request.
This key is bound to the specific application.

\subsubsection*{Nonce}
The nonce is a unique token generated for each request. This way, it can be
determined if a request has been submitted multiple times. 

\subsubsection*{Signature}
The signature is used by Twitter to assure that the request has not been
modified after being sent. It is also used to verify the consumer key and the
token, which is described below.
This is done by generating a token using a cryptographic hash function on, among
others, the request parameters and secret values.

\fix{}{This subsub is maybe a bit badly described. maybe help rewrite or
something, maybe.}
 
\subsubsection*{Signature method}
The signature method for Twitter is always HMAC-SHA1. Any request without this
signature method is invalid. HMAC-SHA1 means that the cryptography hash function
used is \ac{SHA-1}.

\subsubsection*{Timestamp}
The timestamp is a string created together with the request. The string is the
number of seconds since the Unix epoch, which is 1/1/1970. It is used to
determine when the request was made.
If a request is marked as created before the Unix epoch, it is invalid.

\subsubsection*{Authorization token}
The token is used to assure that the application is allowed to
interact with the user's account. It is generated by Twitter by going to
apps.twitter.com and clicking on the application 
\fix{}{maybe go in-depth on how this is done}

\subsubsection*{Version}
The version always has to be 1.0 for the API to accept the
request. This is because the Twitter \ac{API} only uses OAuth 1.0, as opposed to
OAuth 2.0.

\subsubsection{Creating the string}
When all the necessary tokens have been generated, the header string itself can
be created. The finished header looks like this, where \ldots\_val is replaced
with the corresponding token:\nl

OAuth oauth\_consumer\_key="oauth\_consumer\_key\_val",\\
oauth\_nonce="oauth\_nonce\_val", \\
oauth\_signature="oauth\_signature\_val", \\
oauth\_signature\_method="HMAC-SHA1", \\
oauth\_timestamp="oauth\_timestamp\_val", \\
oauth\_token="oauth\_token\_val", \\
oauth\_version="1.0" \\
\citep{TwitterAPIAuth}