\chapter{Queue \ac{API}}\label{queueAPI}
As a part of the overall application, we have to be able to handle requests
and queue them for execution as tasks, such that the first request gets handled
first.

\section{Implementation}
The \ac{API} is implemented in a C\# program, such that it uses the same
language as the QueueServer program. This way, the \ac{API} can use already
written code from the QueueServer library and the Worker class.

\subsection{\ac{REST} \ac{API}}
The \ac{REST} \ac{API} utilises the ASP.net framework with an added controller
class called AnalyzeTwitterAccountController. This controller handles the post
request by the GUI by running the AddTask function from the QueueServerInstance
described in \autoref{sub:queueserver}.

\subsection{Queue server} \label{sub:queueserver}
The Queue server is a singleton class in a library that contains various
queues and functions used to help execute tasks. When the GUI makes a request,
the function AddTask is executed. This function adds the requested Twitter
account to a queue called nonAddedRequests. TaskQueue is an asynchronous
function that is executed on its own thread when the \ac{API} is started and runs for
the entirety of it. In each iteration, it checks if there are any entries in
nonAddedRequests. If there are, it makes a task for each of them and adds them
to a queue of tasks called taskQueue. Afterwards, it starts each task
asynchronously so that at most \fix{5}{if TASK\_LIMIT is changed, change this}
are running at the same time. Afterwards, it checks the list of
running tasks and removes the finished ones. \nl

A code example of how requests are converted to tasks can be seen on
\autoref{taskqueue}. \\

\begin{minipage}[H]{\linewidth}
\begin{lstlisting}[caption = Adding tasks to the queue, label = taskqueue] 
public async void TaskQueue()
{
...
  while (ThereIsNewTask()) // Adds all requests to the queue as tasks.
  {
	TwitterAcc input = nonAddedRequests.Peek();
	
	Task newTask = new Task(() =>
	{
	    ServerTask st = new ServerTask(input);
	    st.Run();
	});
	taskQueue.Enqueue(newTask);
	nonAddedRequests.Dequeue();
  }
...  
}

\end{lstlisting}
\end{minipage}

On \textbf{line 4}, it runs ensures that it runs once for every request. \\
On \textbf{lines 6}, it recieves the next request in line.\\
On \textbf{lines 8-12}, it adds a servertask corresponding to the request.\\
On \textbf{line 13 and 14}, it adds the task to the task queue and removes the
request from the queue of requests.\\

\fix{}{Need more implementation details about how it was done in the code, what
was needed and used}

\subsection{Twitter account}
The class TwitterAcc is used for every instance of a twitter account. It
contains all necessary information about a specific twitter account, such as its
token, Twitter name and token secret. What these are, can be seen on 
\fix{}{Insert reference to section about twitter account.}. 

\subsection{Server task}
An instance of the class ServerTask is created every time a task is created. It
is used to parse the twitter api key and twitter name of the twitter account
being processed to the worker.


\section{Requests}
The \ac{API} only has one post request, which is: \nl

\say{http://localhost:62020/api/TwitterAcc}\nl

Here, the Content-Type header must be ``application/x-www-form-urlencoded'',
where the body contains a value for TwitterApiKey and TwitterName
respectively.
The request returns a response of whether the job was successfully added to the
queue or not.



