\chapter{Queue \ac{API}}
As a part of the overall application, we need to be able to queue jobs to be
executed. A job is a request to our application to evaluate a specific Twitter
users political side.

\section{Implementation}
The \ac{API} is implemented in a C\# program, such that it can interact with the
QueueServer program which also is in C\#. This is the exect reason for why the
\ac{API} has been implemented in C\# such that it can use the already written
code for both the QueueServer which interacts with the Workers which downloads
Tweets and calculates the political values.

\fix{}{Need more implementation details about how it was done in the code, what
was needed and used}

\section{Requests}
The \ac{API} has only one post request which is: \nl

\say{http://localhost:62020/api/TwitterAcc}\nl

where the Content-Type header must be ``application/x-www-form-urlencoded''
then the two values a request needs can be sent in the body. The two values are
TwitterName and TwitterApiKey.
Then the request returns a response weather the job was added to the queue or
not.

\section{Authorization}
\fix{}{Might need authorization in the api}



