\chapter{Queue \acs{API}}\label{queueAPI}
As we use a web application front-end, which should be accessible to many
simultaneous users, we need a way to queue and schedule the order in which
requests from the different users are handled. This chapter describes our queue
server and its \ac{API}.\nl

In \autoref{sec:queueimpl}, we describe how the queue server is implemented, and
how it handles the queueing and scheduling of threads. In
\autoref{sec:queueReq}, we shortly describe the queue server \ac{API} and its endpoints. 

\section{Implementation}\label{sec:queueimpl}
To make a request to the queue server, the front-end must make a
request to our queue server's \ac{API}. This \ac{API} is implemented using the 
C\# language, such that it uses the same language as the queue server.
Using the same language for these system parts allows the \ac{API} to use
existing code from the queue server and the worker class. When the
\ac{API} receives an incoming request, it calls a number of methods in the queue
server library \textc{QSLib}, which handles scheduling and queueing.

\subsection{Queue Server \ac{API}}
The queue server \ac{API} is based on the \ac{REST} principles
\citep{principlesREST} and utilises the ASP.NET framework
\citep{aspNEToverview}. Most of the \ac{API} has been automatically generated as
an \textc{ASP.NET Web Application} template from Visual Studio 2017
\citep{vsMainPage}. We have added a single controller class
called \textc{AnalyzeTwitterAccountController}. This controller handles the
\textc{POST} requests by the \ac{GUI} by running the \textc{AddTask} function
from the \\\textc{QueueServerInstance} described in \autoref{sub:queueserver}.

\subsection{Queue server} \label{sub:queueserver}
At the core of the queue server library is the \textc{QueueServerInstance}
singleton class, which contains the logic needed to queue requests. When the
front-end makes a request, the \textc{AnalyzeTwitterAccountController} calls the
\textc{AddTask} method in the \textc{QueueServerInstance}. This method adds the
requested Twitter account to a \textc{Queue} called \textc{nonAddedRequests}.
Whenever the \textc{QueueServerInstance} class is first instantiated, it calls
an asynchronous method named \textc{TaskQueue}, which runs a \textc{while(true)}
loop and is executed on its own thread. In each iteration of the loop, it
checks if there are any entries in \textc{nonAddedRequests}. If there are, it
makes a task for each of them and adds them to a \textc{Queue} of tasks called
\textc{taskQueue}. Afterwards, it starts each task asynchronously such that at
most five are running at the same time.
Afterwards, it checks the list of running tasks and removes the finished ones. A
code example of how requests are converted to tasks can be seen in
\autoref{taskqueue}. \\

\begin{minipage}[H]{\linewidth}
\begin{lstlisting}[caption = Adding tasks to the queue., label = taskqueue] 
public async void TaskQueue()
{
...
  while (ThereIsNewTask()) // Adds all requests to the queue as tasks.
  {
	TwitterAcc input = nonAddedRequests.Peek();
	
	Task newTask = new Task(() =>
	{
	    ServerTask st = new ServerTask(input);
	    st.Run();
	});
	taskQueue.Enqueue(newTask);
	nonAddedRequests.Dequeue();
  }
...  
}

\end{lstlisting}
\end{minipage}

On \textbf{line 4}, it ensures that it runs once for every request. \\
On \textbf{line 6}, it receives the next request in line.\\
On \textbf{lines 8-12}, it adds a \textc{ServerTask} corresponding to the
request.\\
On \textbf{lines 13 and 14}, it adds the task to the task queue and removes the
request from the request queue.\\

\subsection{Twitter account}
The class \textc{TwitterAcc} is used for every instance of a Twitter account. It
contains all necessary information about a specific Twitter account, such as its
\textc{Access Token}, Twitter name and \textc{Access Token Secret}. These 
concepts are covered in \autoref{cha:twitterAPI}.

\subsection{Server task}
An instance of the class \textc{ServerTask} is created when a task is created.
It is used to pass the Twitter authorisation values and the screen name of the
Twitter account being processed to the worker.

\section{Requests}\label{sec:queueReq}
The \ac{API} only has one post request, which is: \nl

\say{http://localhost:62020/api/AnalyzeTwitterAccount}\nl

Here, the \textc{Content-Type} header must be
``application/x-www-form-urlencoded'', where the body contains a value for
\textc{Name}, \textc{Token} and \textc{Secret}. This header describes that the
content must be encoded using an url encoding where non-alphanumeric characters are replaced with '\%' and
hexadecimal values \citep[sec. 17.13.4]{urlencoded}. The request returns a
response of whether the job was successfully added to the queue or not. 


