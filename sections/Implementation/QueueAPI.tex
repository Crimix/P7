\chapter{Queue \ac{API}}
As a part of the overall application, we need to be able to queue jobs to be
executed, such that the first request gets handled first. A job is a request to
our application to evaluate a specific Twitter user's political orientation.

\section{Implementation}
The \ac{API} is implemented in a C\# program, such that it uses the same
language as the QueueServer program. This way, the \ac{API} can use already
written code from the QueueServer and the Worker classes.

\subsection{Queue Server}
The Queue server is a singleton class in a library that contains various
queues and functions used to help execute tasks. When the GUI makes a request,
the function AddTask is executed. This function adds the requested Twitter
account to a queue called nonAddedRequests. TaskQueue is an asynchronous
function that is executed on its own thread when the \ac{API} is started and runs for
the entirety of it. In each iteration, it checks if there are any entries in
nonAddedRequests. If there are, it makes a task for each of them and adds them
to a queue of tasks called taskQueue. Afterwards, it starts each task
asynchronously so that at most \fix{5}{if TASK\_LIMIT is changed, change this}
are running at the same time. Afterwards, it checks the list of
running tasks and removes the finished ones.

\fix{}{Need more implementation details about how it was done in the code, what
was needed and used}

\section{Requests}
The \ac{API} only has one post request, which is: \nl

\say{http://localhost:62020/api/TwitterAcc}\nl

Here, the Content-Type header must be ``application/x-www-form-urlencoded'',
where the body contains a value for TwitterApiKey and TwitterName
respectively.
The request returns a response of whether the job was successfully added to the
queue or not.



