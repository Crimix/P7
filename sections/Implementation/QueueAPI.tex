\chapter{Queue \ac{API}}
As a part of the overall application, we need to be able to queue jobs to be
executed, such that the first request gets handled first. A job is a request to
our application to evaluate a specific Twitter users political side.

\section{Implementation}
The \ac{API} is implemented in a C\# program, such that it uses the same
language as the QueueServer program. This way, the \ac{API} can use already
written code from the QueueServer and the Worker programs.

\subsection{Queue Server}
The Queue server is a class in a libray that is executed on its own thread when
the \ac{API} is started. When a request is received, it adds it to the queue
server for execution. The queue server then uses the worker class from our library which
downloads the tweets that the requested user's friends tweet. 

\fix{}{Need more implementation details about how it was done in the code, what
was needed and used}

\section{Requests}
The \ac{API} has only one post request which is: \nl

\say{http://localhost:62020/api/TwitterAcc}\nl

Here, the Content-Type header must be ``application/x-www-form-urlencoded'',
where the body contains a value for TwitterApiKey and TwitterName
respectively.
The request returns a response of whether the job was successfully added to the
queue or not.



