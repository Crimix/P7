\chapter{Queue \acs{API}}\label{queueAPI}
As we use a web-application front-end, which should be accessible to many
simultanious users, we need a way to queue and schedule the order in which
requests from the different users are handled. As such, this chapter will be
used to describe our Queue Server Library and its \ac{API}.

\section{Implementation}
To make a request to the Queue Server, the front-end must make a
request to our Queue Server \ac{API}. This \ac{API} is implemented using the 
C\# language, such that it uses the same language as the Queue Server Library.
Using the same language for these system parts allows the \ac{API} to use
existing code from the Queue Server Library and the Worker class. When the
\ac{API} recieves an incoming request, it calls a number of methods in the Queue
Server Library, which handles scheduling and queueing.

\subsection{Queue Server \ac{API}}
The Queue Server \ac{API} is based on the \ac{REST} principles
\citep{principlesREST}, and utilizes the ASP.net framework
\citep{aspNEToverview}. Most of the \ac{API} has been automatically generated as
a \textc{APS.NET Web Application} template from Visual Studio 2017
\citep{vsMainPage}. As such, we have added a single controller class
called \textc{AnalyzeTwitterAccountController}. This controller handles the post
request by the \ac{GUI} by running the AddTask function from the
\textc{QueueServerInstance} described in \autoref{sub:queueserver}.

\subsection{Queue server} \label{sub:queueserver}
At the core of the Queue Server Library is the \textc{QueueServerInstance}
singleton class, which contains the logic needed to queue requests. When the
front-end makes a request, the \textc{AnalyzeTwitterAccountController} calls the
\textc{AddTask} method in the \textc{QueueServerInstance}. This method adds the
requested Twitter account to a \textc{Queue} called \textc{nonAddedRequests}.
Whenever the \textc{QueueServerInstance} class is first instantiated, it calls
an asynchronous method named \textc{TaskQueue}, which runs a \textc{while(true)}
loop, and is executed on its own thread. In each iteration of the loop, it
checks if there are any entries in \textc{nonAddedRequests}. If there are, it
makes a task for each of them and adds them to a \textc{Queue} of tasks called
\textc{taskQueue}. Afterwards, it starts each task asynchronously so that at
most five are running at the same time.
Afterwards, it checks the list of running tasks and removes the finished ones. A
code example of how requests are converted to tasks can be seen in
\autoref{taskqueue}. \\

\begin{minipage}[H]{\linewidth}
\begin{lstlisting}[caption = Adding tasks to the queue., label = taskqueue] 
public async void TaskQueue()
{
...
  while (ThereIsNewTask()) // Adds all requests to the queue as tasks.
  {
	TwitterAcc input = nonAddedRequests.Peek();
	
	Task newTask = new Task(() =>
	{
	    ServerTask st = new ServerTask(input);
	    st.Run();
	});
	taskQueue.Enqueue(newTask);
	nonAddedRequests.Dequeue();
  }
...  
}

\end{lstlisting}
\end{minipage}

On \textbf{line 4}, it ensures that it runs once for every request. \\
On \textbf{lines 6}, it recieves the next request in line.\\
On \textbf{lines 8-12}, it adds a servertask corresponding to the request.\\
On \textbf{line 13 and 14}, it adds the task to the task queue and removes the
request from the queue of requests.\\

\subsection{Twitter account}
The class \textc{TwitterAcc} is used for every instance of a Twitter account. It
contains all necessary information about a specific Twitter account, such as its
\textc{Access Token}, Twitter name and \textc{Access Token secret}. These
concepts relating to access tokens and secrets are covered in
\autoref{cha:twitterAPI}.

\subsection{Server task}
An instance of the class \textc{ServerTask} is created when a task is created.
It is used to parse the Twitter authorisation values and the screen name of the
Twitter account being processed to the worker.

\section{Requests}
The \ac{API} only has one post request, which is: \nl

\say{http://localhost:62020/api/AnalyzeTwitterAccount}\nl

Here, the Content-Type header must be ``application/x-www-form-urlencoded'',
where the body contains a value for Name, Token and Secret.
The request returns a response of whether the job was successfully added to the
queue or not.


