\chapter{Worker}\label{workerLabel}
The worker is the part of the system which collects the tweets and analyzes
them. Each task in the queue corresponds to a worker. 

\section{Implementation}
The worker first retrieves the user which the screen name corresponds to. Then
it retrieves the list of tweets by the user. The next step for the
worker is to retrieve the list of users that the user is following, retrieve
their tweets and analyse the tweets. The retrievement of tweets and the
analysis are done in parallel by splitting them into tasks.

\subsection{Tweet retriever}
There are two versions of the whole tweet retrievement process. The first and
oldest version uses the wrapper function \textc{GetTweetsFromUser}, which wraps
around the helper function \textc{GetTweetsFromUserHelper}. The reason for using
a helper function is that the helper function is used by both versions, with the
only difference being the supplied parameters.


To speed up the retrievement of tweets, it is split up in tasks that are run in
parallel. It is split such that each user, the user is following, corresponds
to a task. This way, the tasks are somewhat equally split. 
When the worker analyzes the tweets, it also needs to split the list of tweets
into tasks, such that it is possible to work on a smaller subset of the
problem. Here, the splitting is done such that each task gets an equal amount of
tweets.

The main difference between how the task splits are done in the retrievement
part and the analysis part is that in the retrievement part, a predefined number
of tasks can run at the same time, even though the total number of tasks may be
higher, whereas the analysis part always has all its predefined number of tasks
running at the same time. The predefined number of the latter may be 1 higher if
it can not split the tasks equally. In this case, the excess tasks are
put in their own small task.

\subsection{}
\fix{}{Needs more}
