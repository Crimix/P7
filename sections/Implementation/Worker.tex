\chapter{Worker}\label{workerLabel}
When the Queue Server begins executing a queued request, it instantiates a
\textc{Worker}, which begins to retrieve and classify Tweets. Each task in the
queue corresponds to a single \textc{Worker}.

\section{Implementation}
The \textc{Worker} starts by requesting a \textc{User} object for the user for
which we want to determine a filter bubble. Then, it retrieves a list of up to
3200 most recent tweets created by the user. The next step for the worker is to
retrieve the list of users that the user is following, retrieve their tweets and
analyse the tweets. The retrieval and analysis of tweets are done in parallel by
splitting them into tasks. To retrieve and analyze tweets, it uses the functions
\textc{GetTweetsFromUserAndAnalyse} and \textc{GetTweetsFromFriendsAndAnalyse}
to receive and analyse tweets from the user and the users it follows
respectively. These methods are found in the class called
\textc{TweetRetriever}.\nl

After analyzing all tweets for necessary for determining a user's filter bubble,
the worker sends the result to the Database API, which then stored the resulting
value on the Database. During this, the Database API also sends a message to the
front-end, telling it that a result is ready to be presented.

\subsection{Splitting up in tasks}\label{sec:taskSplit}
To speed up the retrieval of tweets, the retrieval is split up in tasks that are
run in parallel. It is split such that each user, the user is following,
corresponds to a task. This way, the tasks are somewhat equally split.
When the worker analyzes the tweets, be it while retrieving, from a file or just
a list of tweets, it also needs to split the list of tweets into tasks, such
that it is possible to work on a smaller subset of the problem. Here, the
splitting is done such that each task gets an equal amount of tweets.

The main difference between how the task splits are done in the retrieval
part and the analysis part is that in the retrieval part, a predefined number
of tasks can run at the same time, even though the total number of tasks may be
higher, whereas the analysis part always has all its predefined number of tasks
running at the same time. The predefined number of the latter may be 1 higher if
it can not split the tasks equally. In this case, the excess tasks are put in
their own small task.

\subsection{Tweet retriever} \label{sub:tweetretriever}
There are two versions of the whole tweet retrieval process. The first version
is a legacy version that retrieves tweets and returns them as one list such that
they can be saved in files locally or to be analyzed at a later point.
The new implementation retrieves and analyzes each Twitter account on their own
thread to save time, as described in \autoref{sec:taskSplit}. The code from the
legacy version is moved to the helper function \textc{GetTweetsFromUserHelper},
which both of the versions wrap around. To make this work, the helper method,
among others, receives the \textc{retrieveMethod} as a parameter. Since this
parameter is of the type \textc{Func<List<Tweet>>}, the helper method can take
any method as parameter as long as it returns a list of tweets. This can be
seen on the three methods in the three code examples \autoref{legacyUserTweet},
\autoref{NewUserTweet} and \autoref{UserTweetHelper} where the calls on
\textbf{line 3} in the two wrapper methods give the last parameter as a lambda
expression / anonymous function, which takes no input as seen by \textc{()=>}.\\

\begin{minipage}[H]{\linewidth}
\begin{lstlisting}[caption = Legacy method call , label = legacyUserTweet ] 
public List<Tweet> GetTweetsFromUser(User user, AuthObj auth)
{	
    return GetTweetsFromUserHelper(user, auth, (() => TweetThreadMethod(user, auth)));
}
\end{lstlisting}
\end{minipage}

\begin{minipage}[H]{\linewidth}
\begin{lstlisting}[caption = Current method call to speed up execution , label =
NewUserTweet ] 
public void GetTweetsFromUserAndAnalyse(User user, AuthObj auth, Func<User, bool> checkDB, Func<List<Tweet>, AnalysisResultObj> classifyMethod, Func<AnalysisResultObj, User, bool> postToDB)
{
    GetTweetsFromUserHelper(user, auth, (() => TweetThreadMethod(user, auth, checkDB, classifyMethod, postToDB)));
}
\end{lstlisting}
\end{minipage}

\begin{minipage}[H]{\linewidth}
\begin{lstlisting}[caption = The GetTweetsFromUserHelper, label =
UserTweetHelper]
private List<Tweet> GetTweetsFromUserHelper(User user, AuthObj auth, Func<List<Tweet>> retrieveMethod)
{
    List<Tweet> tweetList = new List<Tweet>();
    Task<List<Tweet>> task = new Task<List<Tweet>>(() => retrieveMethod());
    task.Start();
    task.Wait();
    tweetList.AddRange(task.Result);

    return tweetList;
}
\end{lstlisting}
\end{minipage}

The helper was made to reuse code, because the only change would be the method
to be called in the task on \textbf{line 4}.\\

\subsubsection{Retrieve method}
As mentioned in \autoref{sub:tweetretriever}, the \textc{retrieveMethod}
parameter can be replaced with any method that returns a list of tweets. Both
versions send the method \textc{TweetThreadMethod}, shown on
\autoref{tweetthreadmethod}. \\

\begin{minipage}[H]{\linewidth}
\begin{lstlisting}[caption = The TweetThreadMethod function, label =
tweetthreadmethod]
private List<Tweet> TweetThreadMethod(User user, AuthObj auth, Func<User, bool> checkDB = null, Func<List<Tweet>, AnalysisResultObj> classifyMethod = null, Func<AnalysisResultObj, User, bool> postToDB = null)
{
    bool alreadyExist = false;
    AnalysisResultObj tempResult = new AnalysisResultObj();
    List<Tweet> temp = new List<Tweet>();
    if (checkDB != null)
    {
        alreadyExist = checkDB(user);
		...
    }

    if (!alreadyExist)
    {
        temp = RetrieveTweets(user, auth);
		...	
        if (classifyMethod != null && postToDB != null)
        {
            tempResult = classifyMethod(temp); 
            postToDB(tempResult, user); 
            temp = new List<Tweet>();
        }
    }

    return temp;
}
\end{lstlisting}

\end{minipage}
The difference is that this method has a bunch of optional parameters. The
legacy version only parses a user and an authorization object, while the new
version also parses the methods \textc{checkDB}, \textc{classifyMethod} and
\textc{postToDB}.
\textc{checkDB} is used to check if the user is already in the database. If it
is, it skips the tweet retrievement and analysis of the particular user. If it
is not, it retrieves the tweets and analyzes them if it is the new version.\\

\subsection{Parameter Methods}
The three methods mentioned in \autoref{sub:tweetretriever} -
\textc{checkDB}, \textc{classifyMethod} \\and \textc{postToDB} are defined in
the worker class under the names of \\\textc{CheckIfResultExistOnDB},
\textc{ClassifyTweet} and \textc{PostResultToDB} respectively, where
\textc{CheckIfResultExistOnDB} and \textc{PostResultToDB} also have a version
that links the user being analysed to the overall user. The methods are defined
in \textc{Worker}, rather than \textc{TweetRetriever}, because they are
different for each worker, since the \\\textc{userRecordId} is different for
each request. Such a thing cannot be defined in the singleton class without the need
to pass the id to each method. This way, it is still the worker's methods that
are used in a context where they have access to what is defined in the worker.