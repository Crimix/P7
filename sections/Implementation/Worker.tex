\chapter{Worker}
The worker is the part of the system which collects the tweets and analyzes
them. Each task in the queue corresponds to a worker. 

\section{Implementation}
The worker first retrieves the user which the twitter name corresponds to. Then
it retrieves a list of tweets that the user has posted. The next step for the
worker is to retrieve the list fo users that the user is following and then
retrieve the tweets they have posted. All the retrieving of tweets is split
into task such that the process can be in parallel. Then the tweets can be
analyzed fro the political value. This is also done in a parallelised way.

\subsection{Splitting into task}
The retrieving of tweets needs to be split into different task which can run at
the same time, to speed up th collection of tweets. The way that it split is
that each user that a user is following, corresponds to a task, because for each
of them a maximum of \fix{3200 tweets can be retrieved see \ldots}{We need
to say it either in the analysis or here}. Therefor the tasks are equally split.
When the worker analyzes the tweets it also need to split the list of tweets
into task such that it is possible to work on a smaller subset of the problem, here
the splitting is done such that each task gets an equal amount of tweets. The
main difference between the two parts where the work load is split into task is,
for the retrieving only a maximum number of tasks is running at any given time,
even though a user follows more than the numbers of running task. When in the
analyze part, the split is performed such that the number of tasks specified as
a constant is at most the constant+1. This is because it first tries to split
the tweets evenly between each task, the rest of the tweets is then in its own
task.

\fix{}{Needs more}
