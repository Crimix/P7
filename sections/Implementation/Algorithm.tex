\chapter{Algorithm}
In order to classify the political leanings of a twitter user, we have developed
two different algorithms. One makes use of a neural network, and the other uses
an algorithmic approach using keywords and sentiment analysis. This chapter will be
used to describe the algorithmic approach, and the machine intelligence model
will be described in \autoref{}.

\section{Basic Idea}
The basic idea behind this algorithm is a three-pronged approach, where we
analyze the general sentiment of a tweet, what keywords are present, and what
news media the user is sharing. Using this approach, we can first determine the
sentiment of the tweet, that is, if the user is happy, angry, or neutral. Based
on this, we can apply this sentiment to certain keywords which are present in
the tweet e.g. if the sentiment is positive, and the keyword is the name of a
politician, we have a good indicator as to the users political affiliation.
Additionally, we can analyze what news media the user shares. This can be
useful, as news media often have some inherent political bias, which can be
reflected in the people sharing their material.

\section{Sentiment Analysis}
As described in \autoref{SentiAnal}, the sentiment analysis consists of
identifying words which are considered emotional, and using them to determine
either the general emotional value of a text, or the emotional reaction to
specific nouns in the text.\nl

The algorithm makes use of a simplified sentimental analysis model, where we
determine the general sentimental value of a tweet by counting the occurrences
of positive and negative words. \fix{This approach is also known as a \ldots,
and is generally used for \ldots}{}. These emotional words used for this
analysis have been compiled by Minqing Hu and Bing Liu who are assosiated with
the University of Illinois at Chicago \citep{Hu:2004:MSC:1014052.1014073}.\nl

The implementation of the sentiment analysis can be seen in
\autoref{sentiAnalCode}.\\

\begin{minipage}[H]{\linewidth}
\begin{lstlisting}[caption = Determining the sentiment of a tweet , label =
sentiAnalCode] 
public AnalysisResultObj AnalyzeAndDecorateTweets(List<Tweet> tweetList){
...
	var puncturation = tweet.Text.Where(Char.IsPunctuation).Distinct().ToArray(); 
    List<String> wordList = tweet.Text.Split(' ').Select(x => x.Trim(puncturation)).ToList<String>();

	foreach (string word in wordList){
		int wordValue;
    	if (analysisWords.ContainsKey(word) && analysisWords.TryGetValue(word, out wordValue)){
           	
            if (wordValue == 1){
               	tweet.posList.Add(word);
                tweet.positiveValue++;
            }
            else if (wordValue == -1){
              	tweet.negList.Add(word);
               	tweet.negativeValue++;
          	} ...
		} ...
	} ...
}
\end{lstlisting}
\end{minipage}

On \textbf{lines 3-4} we start by removing the puncturation in the tweet, and
continue by splitting the text on spaces, such that we split it into individual words. On
\textbf{lines 6-8} we iterate over all words in the tweet, and check if our
dictionary of emotional words contain an element with the scanned word as the key. If it does,
we output the emotional value of that word to the variable \textc{wordValue}.
Then on \textbf{lines 10-16} count and add the positive and negative words as
information on the tweet object.

\section{Keyword Analysis}
The original idea was to analyze the hashtags contained in a tweet, and use them
to determine the users political beliefs. As such, we started by compiling a
list of hashtags which were often used in a political context. These hashtags
was found by using the site Top-Hashtags, which allows users to view what
hashtags are frequently used in conjunction. By searching for the hashtag
``\#politics'' we were able to find the top ~200 hashtags most used together
with ``\#politics''. 
Based on this initial list, we determined that the words used as hashtags were
often general enough to also be present in the tweet text itself. As an example,
people would often create hashtags from politicians' names or the name of a
piece of legislature. 
As such, the analysis was expanded into a ``keyword analysis" which also
searched the text body for ``hashtags''. In order to determine the user's
opinion towards a specific keyword, the keyword analysis hase been combined with
the sentiment analysis, such that we can determine the user's sentiment towards
that keyword. The code used for this an

\begin{minipage}[H]{\linewidth}
\begin{lstlisting}[caption = , label = ] 
public AnalysisResultObj AnalyzeAndDecorateTweets(List<Tweet> tweetList){
...
	foreach (string word in wordList){
		...
		if (hashtags.ContainsKey(word) && 
			!tweet.tagList.Contains(word, StringComparer.InvariantCultureIgnoreCase)){
        	HashtagObj hashtagObj;
			
			if (hashtags.TryGetValue(word, out hashtagObj)){
            	tweet.tagList.Add(word);
				int sentiment = tweet.getSentiment();
                if (sentiment > 1)
	                tweet.hashtagBias += hashtagObj.Pos;
                else if (sentiment < -1)
    	            tweet.hashtagBias += hashtagObj.Neg;
                else
        	        tweet.hashtagBias += hashtagObj.Bas;
            } ...
    	} ...
	} ...
}
\end{lstlisting}
\end{minipage}

\section{Special Cases}
Quotes, Sarcasm

\section{Media Analysis}


\section{Tests}