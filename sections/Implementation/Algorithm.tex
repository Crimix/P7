\section{Bag-Of-Words}\label{sec:BoW}
\ldots

\subsection{Basic Idea}
The basic idea behind this algorithm is a three-pronged approach, where we
analyze the general sentiment of a tweet, what keywords are present, and what
news media the user is sharing. Using this approach, we can first determine the
sentiment of the tweet, that is, if the user is happy, angry, or neutral. Based
on this, we can apply this sentiment to certain keywords which are present in
the tweet e.g. if the sentiment is positive, and the keyword is the name of a
politician, we have a good indicator as to the users political affiliation.
Additionally, we can analyze what news media the user shares. This can be
useful, as news media often have some inherent political bias, which can be
reflected in the people sharing their material. The general structure of the
algorithm can be seen in \autoref{AlgorithmStructure}.\\

\begin{minipage}[H]{\linewidth}
\begin{lstlisting}[caption = Main method for handling the algorithm, label = AlgorithmStructure]
public AnalysisResultObj AnalyzeAndDecorateTweets(List<Tweet> tweetList){
	AnalysisResultObj output = new AnalysisResultObj();
   	
   	foreach (Tweet tweet in tweetList){
       	tweet.hasQuotes = CheckForQuotationMarks(tweet);
        
        if (!tweet.hasQuotes){
           	var puncturation = tweet.Text.Where(Char.IsPunctuation).Distinct().ToArray(); 
           	List<String> wordList = 
           		tweet.Text.Split(' ').Select(x => x.Trim(puncturation)).ToList<String>();
                       
            SentimentAnalysis(word, tweet);
            KeywordAnalysis(word, tweet);
            MediaAnalysis(tweet);

            output.KeywordBias += tweet.keywordBias;
            output.MediaBias += tweet.mediaBias;
            output.NegativeSentiment += tweet.negativeValue;
            output.PositiveSentiment += tweet.positiveValue;
        }
    }
    output.Count = tweetList.Count;
    return output;
}
\end{lstlisting}
\end{minipage}

On \textbf{lines 1-4} we start by creating the output object, and begin
iterating over the tweets in the input list. On \textbf{lines 5} we
determine if a tweet containes a quotation. This is necessary as quotes can be
problematic for the analysis. This is further described in
\autoref{AlgorithmCases}. Then on \textbf{lines 7-10}, if a tweet does not
contain quotes, we remove the punction from the tweet's text, and split it into
individual words. On \textbf{lines 12-14} we call the methods for each
individual analysis part. These methods decorate the tweet object with the
information derived from the analysis. Finally on \textbf{lines 16-19} we add
the derived information to the output object, and on \textbf{lines 22-23} we add
the tweet count and return the output object.

\subsection{Special Cases}\label{AlgorithmCases}
When analyzing a tweet, we have a number of special cases which we have so far
been unable to adequately handle. The first special case is ``quotes'', which refers
to whenever a user places a piece of text within a set of quotes. The problem
whith handling this case is that we have found that users often don't agree
with whatever statement they quote, but instead use it to present a view point
which they wish to mock or discuss. This is problematic, as the quote would
originally be attributed to the creator of the tweet, which could skew our
analysis. As such, we have chosen to entirely ignore tweets containing quotes.
In order to determine if a tweet contains quotes, we use a simple regular
expression.

\subsection{Sentiment Analysis}
As described in \autoref{SentiAnal}, the sentiment analysis consists of
identifying words which are considered emotional, and using them to determine
either the general emotional value of a text, or the emotional reaction to
specific words in the text. The algorithm makes use of a simplified sentimental
analysis model, where we determine the general sentimental value of a tweet by
counting the occurrences of positive and negative words. This approach to is
also known as a bag-of-words sentiment analysis\citep{bagSentiment}. The
compiled list of emotional words used for this analysis have been produced by
PhD Minqing Hu and Bing Liu in their paper ``Mining and Summarizing Customer
Reviews"\citep{Hu:2004:MSC:1014052.1014073}. The implementation of the sentiment
analysis can be seen in \autoref{sentiAnalCode}.\\

\begin{minipage}[H]{\linewidth}
\begin{lstlisting}[caption = Determining the sentiment of a tweet , label = sentiAnalCode] 
private void SentimentAnalysis(string word, Tweet tweet){
	int wordValue;

    if (analysisWords.ContainsKey(word) && analysisWords.TryGetValue(word, out wordValue)){
    	if (wordValue == 1){
        	tweet.posList.Add(word);
            tweet.positiveValue++;
        }

        else if (wordValue == -1){
        	tweet.negList.Add(word);
            tweet.negativeValue++;
        }
    }
}
\end{lstlisting}
\end{minipage}

On \textbf{line 4} we start by checking if our dictionary of emotional words
contain an element with the scanned word as the key. If it does, we output the
emotional value of that word to the variable \textc{wordValue}. This value is
either -1 if the word is negative, or 1 if the word is positive. Then on
\textbf{lines 5-13} we increment the tweet's counter for either positive or
negative words, and add the word to the tweet's list of emotional words.

\subsection{Keyword Analysis}
In order to determine the user's opinion towards a specific keyword, the keyword
analysis hase been combined with the sentiment analysis, such that we can
determine the user's sentiment towards that keyword.

The initial list of keywords consists of the most popular hashtags used together
with the hashtag ``\#Politics''. This approach to compiling a word-list was
chosen as it gave us a good starting point. These hashtags were found by using
the site \fix{Top-Hashtags.com,}{Skal det være en kilde?} which allows users to
view what hashtags are frequently used in conjunction. This resulted in the 200 hashtags most used
together with ``\#politics''.\nl

Through testing on a multitude of users, we have manually appended the word-list
with new keywords whenever we identified one that had a clear meaning in a
political context. The format of the word list can be seen in
\autoref{keywordsAnalList}.\\

\begin{table}[H]
\begin{tabular}{|l|l|l|l|}
\textbf{Keyword} & \textbf{Negative Context} & \textbf{Neutral Context} & \textbf{Positive Context} \\\hline
Communism & 10 & 0 & -10 \\\hline
Capitalism & -10 & 0 & 10 \\\hline
\end{tabular}
\caption{Format of the keywords list}
\label{keywordsAnalList}
\end{table} 

In \autoref{keywordsAnalList}, the context values imply either a left or
right leaning conclusion based on the negative/neutral/positive, where negative
values are left leaning and positive values are right leaning. The code used for
the keyword analysis can be seen in \autoref{keywordAnalysisExample}.\\

\begin{minipage}[H]{\linewidth}
\begin{lstlisting}[caption = Method for handling the keyword analysis, label = keywordAnalysisExample] 
private void KeywordAnalysis(List<string> wordList, Tweet tweet){
	foreach (string word in wordList){
    	if (keywords.ContainsKey(word) && 
            !tweet.TagList.Contains(word, StringComparer.InvariantCultureIgnoreCase) && 
            keywords.TryGetValue(word, out KeywordObj keywordObj)){
                        
        	tweet.TagList.Add(word);
            int sentiment = tweet.GetSentiment();

            if (sentiment > 1)
            	tweet.KeywordBias += keywordObj.Pos;
            else if (sentiment < -1)
            	tweet.KeywordBias += keywordObj.Neg;
            else
            	tweet.KeywordBias += keywordObj.Bas;
		}
	}
}
\end{lstlisting}
\end{minipage}

On \textbf{line 2} we start by iterating over all of the words in the tweet. On
\textbf{lines 3-5} we check if our keywords dictionary has a key equal to the
word, and if we have already found that keyword in the text. Finally we output
the \textc{KeywordObj} object related to the value in the dictionary. Then on
\textbf{lines 7-8} we get tweet's sentiment, and  add the current word to the
tweet's keyword-list. On \textbf{lines 10-15} we determine the political value
of the word given the sentimental context as shown in
\autoref{keywordsAnalList}. This value is incremented onto the tweet's existing
\textc{keywordBias} value.

\subsection{Media Analysis}
As a final step in classifying a user's political leaning we determine if the
user is sharing any news media. This is useful as we find that actively sharing
news media often implies an endorsement of that media. And as news media often
has some inherent political leaning and bias, we can assume that users are going
to share those attributes. In order to perform this analysis, we use an index of
american news media with their determined political bias. In order to create
this index, we have made use of the site AllSides.com, which is a website that
attempts to determine news media bias based on a combination of statistical
research and community feedback\citep{allSidesMedia}. When rating a news
media, AllSides applies a rating among 5 values, going from very left
leaning, to neutral, to very right leaning. Internally, we have chosen to
represent this rating as values between -10 and 10, as shown in
\autoref{AllsidesRepres}.\\

\begin{table}[H]
\begin{tabular}{|l|l|l|l|l|l|}
\textbf{AllSides Rating:} & Very Left & Left & Neutral & Right & Very Right \\\hline
\textbf{Internal Representation:} & -10 & -3 & 0 & 3 & 10 \\\hline
\end{tabular}
\caption{Our internal representation of a news media's bias}
\label{AllsidesRepres}
\end{table}

Whenever a user links to an external site in a tweet, the link will be presented
in the tweet's json-object. As such, we have created our index dictionary as a
combination of the site's base URL and our representation of their political
bias value. In order to determine if a user is referencing a news media, we make a
string comparison between the base URL and the URL found in the tweet. The
implementation of the media analysis can be seen in
\autoref{mediaAnalysisExample}.\\


\begin{minipage}[H]{\linewidth}
\begin{lstlisting}[caption = Method for handling the media analysis, label = mediaAnalysisExample] 
private void MediaAnalysis(Tweet tweet){
	foreach (Url link in tweet.Entities.Urls){
		string shortenedUrl = UrlHelper.Instance.ShortenUrl(link.ExpandedUrl);
        if (newsHyperlinks.ContainsKey(shortenedUrl)){
        	tweet.mediaBias += newsHyperlinks[shortenedUrl];
        }
	}
}
\end{lstlisting}
\end{minipage}

On \textbf{line 2} we start by iterating over the URL's present in the
\textc{Tweet} objects URL list. Then on \textbf{line 3} we call the
\textc{ShortenUrl} method in order to trim the Tweet's URL of everything but the
base URL. On \textbf{4-5} we check if our dictionary has an element
corresponding to the URL, and if it does, we increment the tweet's
\textc{mediaBias} value with the determined political bias. 

\subsection{Word Updater}
A problem with the algorithmic approach to determining the user's political
leaning is that the model is unable to adjust itself or to react to a shifting
political landscape. This is problematic, as some indicators of political
affiliation, like names of current politicians, will likely change over time. In
order to alleviate this problem, we have begun implementation of a system that
will be able to identify patterns in keywords used by users.\\ 
The general idea is that whenever the algorithm determines a clear political
affiliation for a user, we can analyze that user to find more keywords which we
can add to our keywords list. This would be done by finding uncommon words in
the user's tweets' text. These words would then be added to a list of currently
monitored words, and whenever a person with a clear political affiliation uses
these words, we increment a counter for either left-wing or right-wing use of
that word. \fix{}{Then what?}

% The unfinished implementation of this system can be seen in
% \autoref{} and \autoref{}.\\
% 
% \begin{minipage}[H]{\linewidth}
% \begin{lstlisting}[caption = , label = ] 
% 
% \end{lstlisting}
% \end{minipage}
% 














