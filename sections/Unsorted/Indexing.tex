\section{Indexing} \label{sec:indexing}
Indexing is an essential part of crawling, as it speeds up the crawling
immensely. This is because it is a lot faster to search through a list of
indexes than to scan through every single page every time a query is performed.
\subsection{Document parsing}
The first part of indexing is called document parsing and consists of scanning
the page and processing it. The processing part consists of tokenizing the page
and saving the tokens in a list with a reference to which pages it is used in
and its location in said pages. This way of doing it is called inverted indexing
and is the most used index data structure in modern information retrieval
systems. This way, it is possible to find the most relevant pages by ranking and
returning the pages referenced the most times by the query tokens. Another way
to index is forward indexing, where the pages themselves are indexed with a
reference to the tokens they contain.
\subsection{Tokenization}
Before data can be indexed, it has to be tokenized. This means that the input
text has to be split up in the smallest meaningful entities, which in this
context are words.
\subsection{Stop words}
Certain words do not contain any meaningful information, and should therefore
not be indexed. These words are referred to as stop words. some good examples of
this are the words ``the'' and ``a''. As these words do not refer to anything in
particular, a query should not match indexed pages based on them.


