\chapter{Facebook Analysis}\label{chap:facebook-analysis}
In the context of social networks, Facebook is difficult to miss
\citep{FacebookPopularity} and as such it had to be looked at to see if it could be used in the project.
The initial thoughts were to look at a specific user and then perform some sort of analysis on the user and their
influences from friends, pages and various forms of media.
It could also involve looking at which posts they agreed with either by sharing them or reacting to them positively.\nl

Facebook has different entities that can interact with one another in different ways.
First and foremost there are the users, which are a central part of the platform, and these are connected via
``friendships''.
A similar entity are ``pages'' which are basically profiles for companies as well as individuals who are or want to be
famous in some way.
Users can mark these pages as ``liked'' which can be used both to get updates from the page, as well as to mark an
interest that others can see on the user's profile.\nl

Facebook prohibits crawling/scraping of their site with a few exceptions such as Google's search crawler.
To access data on their site they have created the Facebook Graph API as part of their developer platform, which
requires a registration of ones application, along with an accompanying API key used when performing queries.
In addition to this requirement there's also a strict requirement of consent from the user of an application, which
needs to specify all the different types of data that the application wants to access.\nl

An analysis of Facebook's Graph API revealed that while the platform had previously allowed for a lot of freedom in
gathering information about a user and their social circles, this was no longer the case since version 2.0 of the API.
Older versions of the API are phased out over time, as new versions are released, meaning it is not possible to access
the needed information for the initial plan.
Version 1.0 of the API was phased out on April 30th, 2015.
At least not using the Graph API.\nl

The current API is limited to only listing those of the user's friends that are using the same app and have given
consent to let the app use their details.
Access to the user's ``likes'' is limited to the interests listed on their profile, such as their favourite movies and
pages on Facebook that they have marked as ``liked''.\nl

It is possible to access the user's posts and details regarding these, including whether or not they were originally
posted by another user or entity and then shared by the user in question.
These posts can then be traced back one or more steps depending on the privacy settings of each link in the
``share--chain'', but only limited information can be gathered about the users on the chain.\nl

Based on this information it would be possible to use Facebook's API to find further connections between existing users
of the system.
Knowledge of these users, based on what the system has found out via other means, could then be used to either confirm
or update the system's knowledge of these individuals.
In addition a lot of news outlets and other companies have pages on Facebook, which could also be used by looking at
what the user has marked as ``liked''.
