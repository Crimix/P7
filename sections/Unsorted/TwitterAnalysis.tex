\section{Twitter Analysis}\label{sec:twitter-analysis}
Twitter is a global social media site with more than 328 million active users
\citep{aboutTwitter}. Twitter's users can generally be split into two groups,
private citizens and businesses. While businesses might use twitter for
announcements and marketing, private citizens are more likely to use twitter for
general communication, and to discuss global and local events. In general
twitter defines itself as being used for the following purposes
\citep{StartingTwitter}:

\begin{enumerate}    
  \item News and Politics
  \item Sports 
  \item Pop Culture 
  \item Influencers 
  \item Utility 
\end{enumerate}

As this project attempts to widen a given users input of news information, we
are going to focus on the first topic, namely ``News and Politics''.

\subsection{Using Twitter}
In order to use twitter, a user will make use of the following elements
\citep{StartingTwitter} in order to make tweets and communicate with other
users:

\subsubsection{Tweet}
A tweet is a general message, which a user can post from their account. A tweet
can contain text, links, pictures and similar media sources. A tweet can be seen
by all other users, and users who ``follow'' the account will automatically be
presented with the tweet on their twitter home page.

\subsubsection{Follow}
A user can choose to follow other peoples accounts. This allows them to be
notified whenever that account makes a tweets or retweets an existing tweet.

\subsubsection{Retweet}
Instead of making a tweet, a user can choose to forward a tweet made by another
account. By doing this, a user can either retweet the tweet as is, or embed the
tweet into a tweet of their own. As such, the user can make comments or remarks
in relation to the original tweet.

\subsubsection{Hashtag}
Whenever a users tweets or retweets they can add a ``hashtag'', which can be
used to group the tweet with others using the same hashtag. A hashtag can be
used a general tag such as \#Politics or \#Sports, or in order to specify the
users stance on a subject e.g. \#Idiotic. In general hashtags are used for
searching and filtering.

\subsection{Twitter API's}
In order for developers to retrieve data about users, tweets and trending
hashtags, Twitter has three main API, namely the REST API's, the Streaming API's
and the ADs API \citep{TwitterDevDocs}. For the purpose of this project,
we will make use of the REST api, as it allows us to read historical twitter
data.

\subsubsection{REST API}
The REST api makes use of http request in order to allow developers to access
Twitter's data \citep{TwitterREST}. In addition, by using the OAuth protocol
\citep{TwitterOAuth} Twitter ensures that only authorized applications can
access the data.\nl

As Twitter makes use of a REST API, all requests must be made by using the HTTP
protocol. These requests are structured differently based on the requested
data. An example of a HTTP request can be seen in \autoref{httpReq}, where each
of the requests parts have been denoted with a number, and is further described
in \autoref{httpElaboration}.

\figx{httpReq}{Structure of Twitter HTTP Request}

\begin{table}[H] 
\begin{centering}
\begin{tabular}{|l|p{9cm}|l|}
Nr.			&	Description 		\\\hline
1			&	Denotes the type of request: GET, POST, DELETE. 					\\\hline
2			&	Address for the Twitter API					\\\hline
3			&	Name of the requested resource					\\\hline
4			&	Format of the requested resource					\\\hline
5			&	Index of the page of the requested data. In case all data can't be
retrieved in a single page \\\hline 
6			&	Extra parameters such as name of the requested user					\\\hline
7			&	Number of data points requested. Maximum of 5000 per page					\\\hline
\end{tabular}
\caption{Elaboration on the Twitter HTTP requests}
\label{httpElaboration}
\end{centering}
\end{table}











\subsection{Crawling Twitter}
