\chapter{Usability Evaluation}\label{cha:usability}

In order to determine the quality of our graphical user interface, we have
performed a usability test with a potential user of our system. The usability
test in this chapter is largely based on \citep[p. 65-72 and 269-275]{UXBook},
as it provides a well structured method of performing usability evaluations.\nl

In \autoref{sec:UXTD} we discuss how the test was designed, and what the purpose
of each part of the test is. In \autoref{sec:UXUT} we discuss the test
conditions and describe the events that occurred during the test.
Finally in \autoref{sec:UXUR} we rank the problems that the user encountered in
the test, and provide solutions to these problems.

\section{Test Design}\label{sec:UXTD}
The test has been designed using a series of research questions as the basis
for two exercises. The goal of the exercises is to have the user use the
system in a simulated real world example. In addition to the test, we have two
questionnaires, one for the demographics and one for opinions.

\subsection{Research Questions}\label{subsec:UXRQ}
The research questions are used to guide the direction of the tasks, as well as
the analysis. We used the research questions on \citep[p. 70-71]{UXBook} as the
basis, with a focus on user interaction and understanding the information.
The following research questions are used as the basis for the tasks:

\begin{enumerate}
  \item \textbf{Web Sites:} 
	\begin{itemize}
      \item How easily do users understand what is clickable?
      \item How easily and successfully do users find the information
      explaining what they are at?
      \item How easily and successfully do users understand the authorization
      steps for the site?
    \end{itemize}
  \item \textbf{Documentation:}
    \begin{itemize}
      \item How well do they understand the content of the topics they find?
    \end{itemize}
  \item \textbf{General:}
    \begin{itemize}
      \item What are the major usability flaws that prevent users from
      completing the most common tasks?
    \end{itemize}
\end{enumerate}

\subsection{Exercises}
Based on the research questions in \autoref{subsec:UXRQ} we created the tasks
seen in \autoref{app:UXTasks}. There are two exercises, each trying to simulate
how a user would use the system.\nl

\subsubsection{Exercise 1}\label{subsub:UXE1}
The first exercise is divided into 3 tasks. The premise of the exercise is that
the that the user has not used the system before, and has heard of by
word-of-mouth.\nl

The first task therefore requires the user to log into the system, using a
pre-selected twitter account. This tasks is used to see how well the user
responds to the steps that are necessary before the user can see his/hers
bias.\nl

The second task wants the user to find his/her bias. This information is shown
several places, as we use and show different methods to determine, and
is intended to somewhat confuse new users. The reason for this is the third
task.\nl

The third task wants the user to locate the information that explains why bias
is shown multiple times. This task is present to see how well the information
is present and explained to the user.

\subsubsection{Exercise 2}
The second exercise is divided into 2 tasks. The premise of this exercise is
that user, now more familiar with the system, wants to find the bias of a
friend. These tasks are used to see how well the user considers the possibility
of the system to check other Twitter users.\nl

The first task wants the user to find the sentiment of a given users
friends, those he is following. This task is in hindsight problematic as we
previously asked the user to log in to check their bias. By using the words log
in, in the test we might have planted the idea that it is not possible to use
other peoples Twitter names to see their bias. This is not the case, as no code
is required to request and see the bias of a Twitter user. The goal of this task
is to see how easily the user considers finding another persons bias using
their Twitter name.\nl

The second and final task is similar to the second task in exercise 1. The goal
of this task is to see how easily another users information is found, it should
also show if it is easier to find information when you know what to look for.


\section{Usability Test}\label{sec:UXUT}
The test is structured to proceed the following way:\nl

First the user is given a questionnaire to see how close he/she is to the target
audience. The user is then given a very brief overview of what the program
does, and asked to read the exercises and make sure he/she understands them. The
user then proceeds through the exercises, as see in \autoref{subsec:UXTest}.
Finally the user is given another questionnaire which asks about the experience.

\subsection{Test Conditions}
The usability test took place on the 12/12/2017. Due to time constraint, the
test and evaluation is performed with only a single user, using a prototype
with some of the interface features not present. It is decided that the test is
to proceed regardless of the missing features as it is considered necessary to
get some outside feedback.\nl

The test is performed on a computer using real data, collected real time. The
test is captured using a web camera for the users expressions and voice, and
screen capture is used to track mouse movement. Present during the test is the
representative user, a test moderator and data logger. The test moderator
ensures that the user understands the task and helps if the user has been stuck
on a task for a lengthy amount of time. The data logger observers the test,
and writes down observations and moments of interest for later analysis.

\subsection{The Test}\label{subsec:UXTest}
The test is conducted by a test moderator and data logger. After the test was
completed, several key issues has become apparent with the execution of the
test. One of the main issues is that the test moderator and data logger is
inexperienced and ended talking with the tester too much, occasionally trying
to help her with suggestions after only minor issues. There is also the problem
of some missing features which were not properly taken into account when the
test was created. Though there were many problems, the outcome is still
relevant.

\subsubsection{Exercise 1 - Duration: 11 minutes}
The first exercise starts poorly as the tester spent close to a minute on the
first screen, trying to get started, finally proceeding after receiving help.
The issue which the tester encountered was that the way to proceed was a link,
with no indication of being a link, at the center of screen, underneath the
logo. The tester proceeded to input the Twitter username and read what is
involved in verification. She finds the link that proceeds to what permissions
are needed, but expresses that she would not do this, were it not for the test.
The rest of the task proceeds smoothly. Though she looks confused as to why the
text is suddenly in the upper left corner. She then proceeds to wait for the
mail to arrive and makes small talk with the test moderator.\nl

The second task start when the mail with the result arrive, after 4 minutes
of small talk. The user proceeds to look at the graphs, and becomes quiet. She
then proceeds to look at the graphs for a frustration for two minutes, hovering
over the bars. After finding one of the correct graphs she proceeds to the final
task.\nl

The final task is problematic, as we due to miscommunication do not have any
information other than graph names and what type of model is used to create it.
The tester proceeds to look for the information above each of the graphs. The
tester the proceeds to scroll to the top of the page to see if there is a
description there. The task is ended shortly after.

\subsubsection{Exercise 2 - Duration: 7 minutes}
Exercise 2 begins with the user still looking at the graphs from the previous
exercise. The tester is told to find the sentiment of another user. She proceeds
to look at the Sentiment/Bias graph, a 2D graph where each user is a point. As
she finds no information about specific users from that she proceeds to look
around on the same page, until she asks for help about 2 minutes. After being
asked to consider how she found her current information, she proceeds quickly to
front page to verify that she wants information for the specific user. An issue
is that she proceeds to click several times on button, without receiving a
response. The rest of the task is quickly finished after she receives the mail,
the only problem is she has trouble identifying sentiment with
positivity/negativity. This is likely to be a issue with people with English as
the non-primary language.\nl

In the final task the tester proceeded to quickly find one of bias graphs, and
quickly found the bias of the other user.

\section{Usability Result}\label{sec:UXUR}
Based on the experiences and difficulties which we experienced during the test,
we have compiled a list of problems. The list has been rated using
\autoref{tab:UXCrit}.

\begin{table}[H]
\centering
\begin{tabular}{|l|l|l|l|}
\hline
 & Delay & \begin{tabular}[c]{@{}l@{}}Irritation/\\ irrationality\end{tabular}
 & \begin{tabular}[c]{@{}l@{}}Expectation\\ vs. actual\end{tabular} \\ \hline
Cosmetic & \textless 1 minute & Low & Small diff.\\\hline 
Serious & \begin{tabular}[c]{@{}l@{}}Several\\ minutes\end{tabular} & Medium & \begin{tabular}[c]{@{}l@{}}Significant\\ diff.\end{tabular} \\ \hline
Critical & \begin{tabular}[c]{@{}l@{}}Total\\ (User stops)\end{tabular} & Strong & Critical diff. \\ \hline
\end{tabular}
\caption{Guidelines for classifying usability problems}
\label{tab:UXCrit}
\end{table}

The following problems were identified and rated:\nl

\textbf{Critical:}
\begin{itemize}
  \item Not enough explanations anywhere
  \item When asking for permissions we ask to be allowed to send tweets on the
  users behalf
\end{itemize}

\textbf{Serious:}
\begin{itemize}
  \item The ``Check Twitter user'' button to begin at the front is difficult to
  find as it looks like text
  \item There is too much information at once, when the graphs are shown
  \item Expects to find information on friends in graphs or somewhere close
\end{itemize}

\textbf{Cosmetic:}
\begin{itemize}
  \item Long waiting times before a reaction when clicking on the ``Check
  Twitter user'' button on the start screen
  \item Confirmation text is presented in the corner
  \item Bars provide unexpected information
  \item Nearly misses the small link on the ``Twitter Verification'' page 
\end{itemize}

\section{Solutions}
Based on the results from \autoref{sec:UXUR} we have the following suggestions
for improving the program and user experience. These have aggregated under the
part of program they pertain to:

\subsection*{Home page}
To fix the issue of the tester not being able to find the ``Check Twitter
user'' button, the button should either look more like a button or be replaced
with the request page.

\subsection*{Authentication}
Since the test user had a minor problem finding the link to acquire the
authentication PIN, an actual button would solve this problem.\\
When asked to accept the permissions required by the application, she thought
they were too invasive. This can be fixed by reevaluating which permissions
are absolutely necessary.

\subsection*{Solutions to overall issues}
When waiting for the result to arrive, the test user said that there are way too
few explanations. This is fixed by adding as much information about each element
on each page. An information button could be added to the home page, explaining
all the overall ideas and concepts in the application.

\subsection*{Result screen}
The test subject had no idea what any of the graphs meant, and all the graphs
seemed overwhelming. To solve this, a lot more information has to be
associated with each graph. For example, there was no way of knowing that green
bar was the where user is located. 
