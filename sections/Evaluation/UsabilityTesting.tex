\chapter{Usability Evaluation}\label{cha:usability}
\fix{}{Write me}


\section{Test Design}\label{sec:UXTD}
The test is designed with a numbe \fix{}{Write me}



\subsection{Research Questions}\label{subsec:UXRQ}
The research questions are used to guide the direction of the tasks, as well as
the analysis. We used the research questions on \citep[p. 70-71]{UXBook} as the
basis, with a focus is on user interaction and understanding the information.
The following research questions are used as the basis for the tasks:

\begin{enumerate}
  \item \textbf{Web Sites:} 
	\begin{itemize}
      \item How easily do users understand what is clickable?
      \item How easily and successfully do users find the information they are
      looking for?
      \item How easily and successfully do users register for the site?
    \end{itemize}
  \item \textbf{Documentation:}
    \begin{itemize}
      \item How well do they understand the content of the topics they find?
    \end{itemize}
  \item \textbf{General:}
    \begin{itemize}
      \item What are the major usability flaws that prevent users from
      completing the most common tasks?
    \end{itemize}
\end{enumerate}

\subsection{Exercises}
Based on the research questions in \autoref{subsec:UXRQ} we created the tasks
seen in \autoref{app:UXTasks}. There are two exercises, each trying to simulate
how a user would use the system.\nl

\subsubsection{Exercise 1}\label{subsub:UXE1}
The first exercise is divided into 3 tasks. The premise of the exercise is that
the that the user has not used the system before, and has heard of by
word-of-mouth.\nl

The first task therefore requires the user to log into the system, using a
pre-selected twitter account. This tasks is used to see how well the user
responds to the steps that are necessary before the user can see his/hers
bias.\nl

The second task wants the user to find his/her bias. This information is shown
several several places, as we use and show different methods to determine, and
is intended to somewhat confuse new users. The reason for this is the third
task.\nl

The third task wants the user to locate the information that explains why bias
is shown multiple times. This task is present to see how well the information
is present and explained to the user.

\subsubsection{Exercise 2}
The second exercise is divided into 2 tasks. The premise of this exercise is
that user, now more familiar with the system, wants to find the bias of a
friend. These tasks are used to see how well the user considers the possibility
of the system.\nl

The first task wants the user to find the sentiment of a given users
friends, those he is following. This task is in hindsight problematic as we in
\autoref{subsub:UXE1} asked the user to log in to check their bias. By using
the words log in, in the test we might have planted the idea that it is not
possible to use other peoples twitter names to see their bias. This is not the
case, as no code is required to request and see the bias of a twitter user.\\
The goal of this task is to see how easily the user considers finding
another persons bias using their twitter name.\nl

The second and final task is similar to the second task in exercise 1. The goal
of this task is to see how easily another users information is found.

\subsection{Test Conditions}
Due to time constraint, the test is performed with only a single user,
using a prototype with some of the interface features not present. It is
decided that the test is to proceed regardless of the missing features as it
is considered necessary to get some outside feedback.\nl

The test is performed on a computer using real data, collected real time. The
test is captured using a web camera for the users expressions and voice, and
screen capture is used to track mouse movement. Present during the test is the
representative user, a test moderator and data logger. The test moderator
ensures that the user understands the task and helps if the user has been stuck
on a task for a lengthy amount of time. The data logger observeres the test,
and writes down observations and moments of interest for later analysis.

\section{Test Description}

\section{Usability Problems}

\section{Solutions}
\fix{}{blabla intro to solutions}

\subsection{Home page}
To fix the issue of the test user not being able to find the ``Check Twitter
user'' button, the button either look more like a button or completely remove
the button and move the request page to the front page, since it is the only
feature of the application.

\subsection{Authentication}
Since the test user had a minor problem finding the link to aquire the
authtication PIN, an actual button would solve this problem.\\
When asked to accept the permissions required by the application, she thought
they were too invasive. This can be fixed to by reevalutating which permissions
are absolutely necessary.

\subsection{Solutions to overall issues}
When waiting for the result to arrive, the test user said that there are way too
few explanations. This is fixed by adding as much information about each element
on each page. An information button could be added to the home page, explaining
all the overall ideas and concepts in the application.

\subsection{E-mail}




% The chapter starts by presenting the usability testing of the
% system. It then concludes on the quality of the system's frontend, and
% determines a number of improvements to be implemented. Finally it presents the
% improvements added based on the usability test.


















