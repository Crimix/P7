\chapter{Future Work}\label{fwork}
This chapter highlights different parts of the system which could
be improved for a future version of the application.

\section{Limit System}
The limit on the amount of requests to Twitter we are allowed to make works fine
on a small scale, as long as the user of the application does not use other
programs that uses requests from the same request pool as our application. This
is due to the fact that our application does not check how many request there
are remaining, which would risk exceeding the request limit. \nl
%The limit rate system that exist in the program works on a small scale, when
%only our application is used because we do not update the rate limits by
%using a specific request or using the response from a
%request.
%The headers of a response contains the remaining request in the pool for that
%category of requests. \nl 
%Dette er erstattet af det over. Er du ikke er enig i ændringen siger du bare til :)

Before a potential release, a better implementation of requests would have to be
made. This could be done by first sending a request to get the remaining amount
of requests available and then each time a request is made, the response
header, which contains the remaining amount of request, updates the
limits in the system.\nl


%For future use of the Twitter \ac{API} a better implementation for the limits
%system would be to make a request first to get the amount of requests the user
%has for each category, then each time a response is received from Twitter the
%header with the remaining amount of requests available. \nl
\fix{}{check if anything of value is lost from the outcommented sections}

\section{Queue API Authorization}
In the final version, the Queue API does not use any kind of authorization for
the requests. This means that it is possible for anyone to send requests to the
queue server. The problem is that without authorization, we cannot guarantee
that it is only our application that tries to analyse Twitter accounts. This
also means that our queue server needs to be robust enough to not crash when it
receives something that is not a proper request. \nl

While this is not a big problem for the current version of the application,
since it uses a local server, it would be a huge problem on a large scale. 

\section{Word updater for Bag-Of-Words Implementation}
Right now the Bag-Of-Words classifier, can only classify using already known
words which has been premade and stored in a list. This results in the
classifier being able to classify tweets with a high accuracy right now, but it
could in a few months lose some of that accuracy because of how the key words 
change. This is why we started to implement a system that should identify new
keywords and put them on a watch list using the Naive Bayes algorithm. This
system has some functionality implemented, but it was not possible to complete it within the time span.

For a system that should be able to be used without the maintenance required by
the Bag-Of-Words classifier, the system using Naive Bayes should be developed in
full.

\section{More MI  Models}
\fix{}{write}

\section{Larger training samples}
\fix{}{write}

\section{Less rushed / more usability tests for better feedback}
\fix{}{write}

\section{Result Person Identification}
Currently, we cannot identify who each person is in the filter bubble on the
result screen. This would be a nice feature to have, as the user could be
curious about where specific people they follow are placed on the scale without
having to look each of them up individually. In addition to this feature, a
shortcut to looking up the filter bubble of another user could be added by
allowing the user to click on a specific user in their filter bubble.


\section{Recommendation System}
In our original design of the web application, after displaying the result of
the analysis of the filter bubble, the application should list some different
twitter accounts that you could follow such that you will get more different
political views that what you currently get on Twitter. \nl

This feature would need to be developed for the actual busting of the filter
bubble to exist. One way of implementing this feature would be to use data from
our database where we have analysed Twitter users and selecting twitter users to
show which have a different political learning than you. Another way would be to
do the same with regards to present users with different political learning, but
also users with many followers, this then requires that we store such
information about a user in our database.\nl

To call our web application finished this would need to be implemented.

\section{HTTP not HTTPS}
\fix{}{write}

\section{Database Twitter Protect Boolean}\label{sec:twitterProtect}
Although in the database there is a boolean used to represent if you need to
follow the user on Twitter to view their tweets. We do not 
