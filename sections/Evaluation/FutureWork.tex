\chapter{Future Work}\label{fwork}
This chapter highlights different parts of the system which could
be improved for a future version of the application.

\section{Limit System}
The limit on the amount of requests to Twitter we are allowed to make works fine
on a small scale, as long as the user of the application does not use other
programs that uses requests from the same request pool as our application. This
is due to the fact that our application does not check how many request there
are remaining, which would risk exceeding the request limit. \nl
%The limit rate system that exist in the program works on a small scale, when
%only our application is used because we do not update the rate limits by
%using a specific request or using the response from a
%request.
%The headers of a response contains the remaining request in the pool for that
%category of requests. \nl 
%Dette er erstattet af det over. Er du ikke er enig i ændringen siger du bare til :)

Before a potential release, a better implementation of requests would have to be
made. This could be done by first sending a request to get the remaining amount
of requests available and then for each time a request is made, the response
header which contains the remaining amount of request and then update the limits
in the system.\nl


%For future use of the Twitter \ac{API} a better implementation for the limits
%system would be to make a request first to get the amount of requests the user
%has for each category, then each time a response is received from Twitter the
%header with the remaining amount of requests available. \nl
\fix{}{check if anything of value is lost from the outcommented sections}

\section{Queue API Authorization}
In the final version, the Queue API does not use any kind of authorization for
the requests. This means that it is possible for anyone to send requests to the
queue server. The problem is that without authorization, we cannot guarantee
that it is only our application that tries to analyse Twitter accounts. This
also means that our queue server needs to be robust enough to not crash when it
receives something that is not a proper request. \nl

While this is not a big problem for the current version of the application,
since it uses a local server, it would be a huge problem on a large scale. 

\section{Word updater for Bag-Of-Words Implementation}
\fix{}{write}

\section{More MI  Models}
\fix{}{write}

\section{Larger training samples}
\fix{}{write}

\section{Less rushed / more usability tests for better feedback}
\fix{}{write}

\section{Result Person Identification}
Currently we cannot indetify who each person is in on the resulting graphs.
\fix{\ldots}{Moooorreee}
