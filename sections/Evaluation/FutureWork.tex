\chapter{Discussions and future work}\label{fwork}
This chapter highlights different parts of the system which could
be improved for a future version of the application.

\section*{Limit System}
The limit on the amount of requests to Twitter we are allowed to make works fine
on a small scale, as long as the user of the application does not use other
programs that uses requests from the same request pool as our application. This
is due to the fact that our application does not check how many request there
are remaining, which would risk exceeding the request limit. \nl
%The limit rate system that exist in the program works on a small scale, when
%only our application is used because we do not update the rate limits by
%using a specific request or using the response from a
%request.
%The headers of a response contains the remaining request in the pool for that
%category of requests. \nl 
%Dette er erstattet af det over. Er du ikke er enig i ændringen siger du bare til :)

Before a potential release, a better implementation of requests would have to be
made. This could be done by first sending a request to get the remaining amount
of requests available and then each time a request is made, the response
header, which contains the remaining amount of request, updates the
limits in the system.\nl


%For future use of the Twitter \ac{API} a better implementation for the limits
%system would be to make a request first to get the amount of requests the user
%has for each category, then each time a response is received from Twitter the
%header with the remaining amount of requests available. \nl
\fix{}{check if anything of value is lost from the outcommented sections}

\section*{Queue API Authorization}
In the final version, the Queue API does not use any kind of authorization for
the requests. This means that it is possible for anyone to send requests to the
queue server. The problem is that without authorization, we cannot guarantee
that it is only our application that tries to analyse Twitter accounts. This
also means that our queue server needs to be robust enough to not crash when it
receives something that is not a proper request. \nl

While this is not a big problem for the current version of the application,
since it uses a local server, it would be a huge problem on a large scale. 

\section*{Word updater for Bag-Of-Words Implementation}
Right now the Bag-Of-Words classifier, can only classify using already known
words which has been premade and stored in a list. This results in the
classifier being able to classify tweets with a high accuracy right now, but it
could in a few months lose some of that accuracy because of how the key words
change. This is why we started to implement a system that should identify new
keywords and put them on a watch list using the Naive Bayes algorithm. This
system has some functionality implemented, but it was not possible to complete
it within the time span.

For a system that should be able to be used without the maintenance required by
the Bag-Of-Words classifier, the system using Naive Bayes should be developed in
full. While this approach is not perfect, it would allow us to dynamically
update our model with a shifting political climate.
% 
% A problem with the Bag-of-Words approach to determining the user's political
% leaning is that the model is unable to adjust itself or to react to a shifting
% political landscape. This is problematic, as some indicators of political
% affiliation, like names of current politicians, will likely change over time. In
% order to alleviate this problem, we have begun implementation of a system that
% will be able to identify patterns in keywords used by users.\\ 
% The general idea is that whenever the algorithm determines a clear political
% affiliation for a user, we can analyze that user to find more keywords which we
% can add to our keywords list. This would be done by finding uncommon words in
% the user's tweets' text. These words would then be added to a list of currently
% monitored words, and whenever a person with a clear political affiliation uses
% these words, we increment a counter for either left-wing or right-wing use of
% that word. Whenever we determine that we have enough data, we can add this word
% to our list of political keywords.\nl

\section*{More MI  Models}
\fix{}{write}

\section*{Larger training samples}
One of the problems with the Naive Bayes model is that the amount of training
samples are limited to 120, with actual training only using 90. In more ideal
circumstances, the training set should increase by 10 or 100 times. If using a
sizable, varied and balanced training set, it should be possible to achieve a
better prediction rate at the cost of slower predictions. This can also be
extended to other models.\nl

By using larger training samples, there will be more features to consider, which
should increase accuracy a lot. But there will also be a larger requirement on
the memory usage. From what we experienced while working on Naive
Bayes, this can be quite a problem for other models. To work around
this, we will need to consider reducing the amount of features we use. A
suggestion from \autoref{subsec:Exp2} was to use Principal Component Analysis.
This idea, along with other methods of reducing the dimensionality of the
features, would need to be explored further.

\section*{Less rushed / more usability tests for better feedback}
\fix{}{write}

\section*{Result Person Identification}
Currently, we cannot identify who each person is in the filter bubble on the
result screen. This would be a nice feature to have, as the user could be
curious about where specific people they follow are placed on the scale without
having to look each of them up individually. In addition to this feature, a
shortcut to looking up the filter bubble of another user could be added by
allowing the user to click on a specific user in their filter bubble.



\section{Recommendation System}
In the initial design idea of the web application, the application listed
different Twitter accounts that would help the user's Twitter feed get closer to
the middle of the poltical spectrum, thus breaking their filterbubble. \nl

This feature would have to be developed before actual bubble busting would take
place. One way of implementing this feature would be to make a seperate
database of recommended users with data generated from earlier searches. This
database would of course have to start out with some hardcoded accounts.
Otherwise, the first few users of the program would not get any
recommendations. There would have to be a filter on which users gets added to
the database. This could be a requirement of at least 10000 followers.
Otherwise, people would get irrelevant Twitter accounts recommended. If the
application had a lot of traffic, a filter on what kind of people they want to
be recommended could be implemented. For example, some people could prefer to
only get well educated people recommended rather than celebrities who happen to
tweet politically a lot.


\section*{HTTP not HTTPS}
\fix{}{write}

\section{Database Twitter Protect Boolean}\label{sec:twitterProtect}
Some Twitter accounts are privatised, thus requiring the user to follow that
specific account before being able to access their tweets. Therefore, if a
privatised account is in the database by a user who follows that account and
another user, who does not follow the account, searches for the account, they
will acquire information they would otherwise not be allowed to get. While this
is not a violation with the Twitter API, it is bad practice. While a boolean
determining whether a Twitter account is private or not is stored in the
database, it is not currently used. This would have to be implemented before a
public release.

\section{Worker Unrecoverable Error}
If the worker encounters an error for which it can not recover, it does not send
any information to either the Database Server or GUI Server, as such a user does
not ever receive the email or any information. \nl

This is not acceptable in a finished system and need to be addressed before it
can be launched.

