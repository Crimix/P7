\chapter{Future Work}\label{fwork}
This chapter will be used to highlight different parts of the system which could
be improved for a future version of our application.

\section{Limit System}
The limit rate system that exist in the program works on a small scale, when the
user only uses our application. Because we do not update the rate limits
by using a specific request or using the response from a
request, during analysis.
The headers of a response contains the remaining request in the pool for that
category of requests. \nl

For future use of the Twitter \ac{API} a better implementation for the limits
system would be to make a request first to get the amount of requests the user
has for each category, then each time a response is received from Twitter the
header with the remaining amount of requests available should be used to update
the counters for the different category pools.
\nl

The problem with the current system is that the user can not use any other
application that uses the same category of requests as our application without
the risk of going over the limit that Twitter has specified.

\section{Queue API Authorization}
Right now the Queue API does not use any form for authorization for the
requests, this means that is possible for any one to send requests to the
Queue server, the problem is that without authorization we cannot guaranty that
it is only our application that tries to analyse Twitter accounts, this also
means that our Queue server needs to be robust enough to not crash when it
receives something that is not a proper request. \nl

This is not a big problem for the state that the application is in, because it
is not on any server which can be reached from the internet, but only on
localhost on our computers. 

\section{Word updater for Bag-Of-Words Implementation}
\fix{}{write}

\section{More MI  Models}
\fix{}{write}

\section{Larger training samples}
\fix{}{write}

\section{Less rushed / more usability tests for better feedback}
\fix{}{write}
