\chapter{Conclusion}\label{cha:conclusion}
This chapter will be used to conclude on the success of the project based on the
requirements in \autoref{cha:req}, and the final functional testing done in
\autoref{cha:testing}. Throughout this project the group has been designing and
developing a web-application with the purpose of allowing Twitter users to identify their
political filter bubble. The system was developed with a web-application
front-end using the Laravel framework, a MySQL database, and a server back-end using the .NET C\# framework. 
To make an initial conclusion on the success of the project: We have been able
to complete all of the requirements, except for a number of those relating to
the usability of the graphical user interface. While we have been unable to
complete all requirements, we still deem that the project has been a success, as
it is capable of performing its main function, namely giving a user the
information necessary to identifying their political filter bubble. All of
the functional requirements and their level of completion can be seen below in
\autoref{sec:concFunc}, and the usability requirements can be seen in
\autoref{sec:concUsa}.

\section{Functional Requirements}\label{sec:concFunc}
The requirements in \autoref{table:concFuncReq} describe system
features and parts which should be implemented.

\begin{longtable}{|p{6cm}|p{8cm}|}\hline 
\textbf{Requirement:}	&	\textbf{Evaluation:} \\\hline
The system should be able to retrieve a user's tweets, and a list of who they
follow.
&	This requirement has been completed, and has been documented in
\autoref{sub:tweetretriever}.\\\hline 
The system should adhere to the Twitter guidelines regarding limited requests
&	This requirement has been completed. In \autoref{DatabaseAPI} we
conclude that we need to store data, such that we limit the need for
Twitter requests. Furthermore in \autoref{sec:rateLimit} we describe how we
monitor our number of available requests. \\\hline
The program should be able to determine a user's political leaning.
&	This requirement has been completed. In \autoref{cha:classification} we
show our approaches to classifying users, and in \autoref{sec:accTest} we show
that when analyzing a set of politically active and inactive users, we get an
accuracy of up to 96\%.\\\hline
The application should have a web page front-end.
&	This requirement has been completed. The front-end and it's implementation can
be seen in \autoref{}. \\\hline
The system logic should be hostable on as a server backend.
& This requirement has been completed. The implementation of the back-end can be
seen in \autoref{queueAPI}, \autoref{workerLabel}, and
\autoref{cha:classification}. \\\hline
The should be able to store analyzed user data on a database.
&	This requirement has been completed. The description of the database and its
API can be seen in \autoref{DatabaseAPI}. \\\hline
The system should be scalable in order to handle many simultanious users.
&	This requirement has been completed. \autoref{queueAPI} describes how we queue
user requests, and \autoref{cha:twitterAPI} describes how we allow the
application to make more requests by making requests on users behalf.\\\hline
The front-end should allow a user to authorize our application with their
Twitter account.
&	This requirement has been completed. In \autoref{GUI} we describe the
implementation of our front-end, and how we handle User authrization. \\\hline
The system should use APIs to communicate between the different system parts.
&	This requirement has been completed. The different APIs are described in
\autoref{DatabaseAPI}, and \autoref{queueAPI}.\\\hline
\caption{Functional requirements and their level of completion.}
\label{table:concFuncReq}
\end{longtable}

\section{Usability Requirements}\label{sec:concUsa}
The requirements in \autoref{table:concUsaReq} describe aspects of the Graphical
User Interface which should be adhered to.

\begin{longtable}{|p{6cm}|p{8cm}|}\hline
\textbf{Requirement:}	&	\textbf{Evaluation:} \\\hline
The system use of color should be suited for color blindness.
&	This requirement has been mostly completed. Most of the GUI is black and
white. Whenever color is used, it is with a white background, except in the
result graphs. Due to our intent on showing graphs as shades of blue and red
based on political leaning.\\\hline 
When presenting colorful elements next to each other, the system should use
background patterns to allow for easier distinguishing.
& This requirement has not been completed. Currently the resulting graphs only
use color when presenting data. \\\hline
Text and images should be resizable.
&	This requirement has been partly completed. Currently only text is scalable,
while it is not possible to enlarge the graphs.\\\hline
The system should minimize the amount of user actions necessary.
&	This requirement has been completed. Currently, all of the system's
functionality can be reached with less than 5 clicks.\\\hline
The GUI should be simplistic, and easily recognizable.
&	This requirement has been completed. In order to make our system easy to
understand, the front page consists of a search bar similar to the one used by
search engines such as Google and Yahoo.\\\hline
There should be no doubts about what information is
needed and which button(s) to press. 
&	This requirement has been partly completed. Based on our usability test, there
was some confusion as to what elements to interact with, but since then the GUI
has been entirely reworked in order to fix those elements. \\\hline
The user should have easy access to assisting information on how to use the
system. 
&	This requirement has been completed. Currently the system has a large Info
button on the main page, and all cases of user actions have been described in
plain text next to the element in question. \\\hline
When presenting data it should be clearly labeled, and explained in
helping text what the data represents.
& This requirement has been completed. All system elements is supplied with
helpting text next to it. Additionally, all data graphs has clealy explained and
labeled data. \\\hline
It should be clearly stated what user data will be used for.
&	This requirement has been completed. The Info section clealy states what we
will use the data for, and we have added an explaining text before the user
presses the button to authorize their account with our application \\\hline
The system should use neutral language and imply no political bias when
presenting data in a political context.
&	This requirement has been mostly completed. Currently, the text in the system
imply no political leaning. A possible problem could be, that left-wing
is labeled as a negative value, and right wing as a positive.\\\hline
\caption{Usability requirements and their level of completion.}
\label{table:concUsaReq}
\end{longtable}

\section{Final Conclusion}
Based on the entirely or partly completion of all of the functional
requirements, we conclude that the system is a succes. While we have failed to
complete all of the usability requirements, the current state of the system is
still usable, and capable of performing its indended function. Finally, if we
look at our problem statement:\nl

\say{How can a web-application be developed as to assist a user in
identifying and breaking out of a filter bubble? And how can this be done by
using information gathered from Twitter in the form of tweets?}\nl

We can conclude that we have managed to develop a system which is capable of
identifying a user's filter bubble based on tweets from Twitter, and visualize
it such that they should be able to understand the inherent bias inside their
specific bubble.




















