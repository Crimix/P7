\chapter{Testing}\label{cha:testing}
In order to determine the quality of the developed software solution, we have
performed a number of tests. These tests aim to assure the quality of the
determined results, and to ensure that the software is operating as expected.\nl

The tests were performed on a PC running Windows 10, and having a 3.1GHz
processor, 256GB SSD harddisk, and 8GB DDR4 RAM.

\section{Speed Test}\label{speedtestlavel}
The purpose of this test is to document to effeciency of the two different
approaches to classification discussed in \autoref{sec:BoW} and
\autoref{sec:NBImp}. The effeciency will be evaluated based on the time
necessary to process a given number of tweets. As such, each approach will
process 819.200 tweets in batches of 3200. These numbers have been chosen as we
need a high enough amount to accurately determine the average time necessary,
and because each tweet analysis will process a maximum of 3200 tweets, as this
is the maximum amount of tweets retrievable from a person.

\subsection*{Results}
A subset of the results from the speed test can be seen in \autoref{table:stest}
together with the average tweets per second in \autoref{AvgTweetsPs}. The
full dataset can be found in \autoref{app:speedTest}.

\begin{table}[H]\centering
\begin{tabular}{|l|l|l|l|l|}
\hline
Try Nr.:	&	BoW (ms)	&	BoW Tweets/Sek	&	Bayes (ms)	&	Bayes Tweets/Sek	\\\hline
1	&	55	&	58181	&	493	&	6490 \\\hline
2	&	22	&	145454	&	353	&	9065 \\\hline
3	&	30	&	106666	&	334	&	9580 \\\hline
4	&	23	&	139130	&	328	&	9756 \\\hline		
\end{tabular}
\caption{Subset of the results from the speed test}
\label{table:stest}
\end{table}

\begin{table}[H]\centering
\begin{tabular}{|l|l|l|}\hline
					&	\textbf{BoW Tw/sek}	&	\textbf{Bayes Tw/sek}	\\\hline
\textbf{Average:}	&	124007				&	9410 					\\\hline	
\end{tabular}
\caption{Average number of tweets processed per second}
\label{AvgTweetsPs}
\end{table}

The results from this test show that the bag-of-words model is the fastest
approach, being capable of processing around 120.000 tweets per second, and the
naive bayes approach processing around 9500 tweets per second. To put this into
perspective, if a user were to follow 500 accounts, it would result in a maximum
of 16.000.000 tweets, which would take around 2 minutes to process for the
bag-of-words model, and 26 minutes for the naive network. Considering that this
amount of followed users is above average, we consider this processing time as
being acceptable.








\section{Accuracy Test}
The purpose of this test is to document to accuracy of the results determined
by the two different approaches to classification discussed in \autoref{sec:BoW}
and \autoref{sec:NBImp}. The accuracy will be determined by comparing the
results from the classification with a set of manually pre-classified
twitter users. The chosen users consists of politicians, activists, and a set of
average twitter users.

\subsection*{Results}
A subset of the results from the accuracy test can be seen in
\autoref{table:acctest} together with the percentage success in
\autoref{AvgTweetsPs}. The full dataset can be found in
\autoref{app:accuracyTest}.\nl

A subset of the dataset from the accuracy test can be seen below in
\autoref{speedTestReslabel}. Each test shows one pre-classified user, and the
results determined from the bag-of-words model and the Naive Bayes model. The
result consist of a value between -10 and 10, where a low value implies
left-leaing views, and a high value implies right-leaning views. The
bag-of-words (BoW) model also has a conclusion, which is dependent on the
percentage of tweets considered political, and the size of the determined value.
For the BoW model, a user with less than 1\% political tweets, or a determined
value below 1, is automatically considered as neutral. The percentage accuracy
of each model can be seen in \autoref{AccPercent}.
The columns represent the following, from left to right:\nl

\begin{itemize}
  \item The percentage of tweets the BoW model
consider political in nature.
\item Numeric result from BoW model.
\item Conclusion from BoW model.
\item Numeric result from the Bayes model.
\item Conclusion from the Bayes model.
\item Actual political leaning of the analyzed user.
\end{itemize}

\begin{table}[H]\centering
\begin{tabular}{|l|l|l|l|l|l|}
\hline
\textbf{Pol\%:}	&	\textbf{Alg:}	&	\textbf{Alg Conc:}							&	\textbf{Bayes:}	&
\textbf{Bayes Conc:}
& \textbf{Actual:}
\\\hline 0.64	&	-2.42	&	Left/Neutral (Pol\% too low)		&	5.82	&	Right		&	Left	\\\hline
1.82	&	-0.865	&	Right								&	3.16	&	Right		&	Right	\\\hline
8.25	&	-3.219	&	Left								&	2.55	&	Right		&	Left	\\\hline
1.39	&	0.5		&	Right/Neutral (Result is 0 +/- 1)	&	3.51	&	Right		&	Right	\\\hline
\end{tabular}
\caption{Results from the speed test}
\label{speedTestReslabel}
\end{table}

\begin{table}[H]\centering
\begin{tabular}{|l|l|}\hline
\textbf{Accuracy BoW}	&	\textbf{Accuracy Bayes}	\\\hline
96.67\%					&	43.33\%					\\\hline	
\end{tabular}
\caption{Average number of tweets processed per second}
\label{AccPercent}
\end{table}

The results from this test show that the bag-of-words model is the most
approach, being capable of accurately classifying 96\% of the test users, and
the naive bayes approach classifying 46\%.













\section{Tweet Retrieval Test}
The purpose of this test is to document the limitations of retrieving tweets
through the Twitter API. This is important as the speed at which we can retrieve
tweets sets an upper limit to how fast we can present an output to the user.
The speed will be determined by calling the twitter API from the worker class,
and documenting the amount of tweets we can recieve over a set amount of time.
The Worker class and tweet retrieval is discussed in \autoref{workerLabel}.

\subsection*{Results}
A subset of the dataset from the tweet retrieval test can be seen below in
\autoref{tweetRetReslabel}, and the full dataset can be found in
\autoref{app:tweetRet}.
Each test shows the retrieval of tweets from 5 different users, and the result is the total tweets retrieval per second in
total from all five concurrent threads. The average tweets per second can be
seen in \autoref{RetNumberAvg}. 

\begin{table}[H]\centering
\begin{tabular}{|l|l|l|}
\hline
\textbf{Nr.:}	&	\textbf{Time (ms):}	&	\textbf{Tweets/Sek:} \\\hline
1	&	8603	&	332.6\\\hline
2	&	9385	&	304.8\\\hline
3	&	8229	&	347.7\\\hline
4	&	9454	&	302.6\\\hline
\end{tabular}
\caption{Subset of the results from the tweet retrieval test}
\label{tweetRetReslabel}
\end{table}

\begin{table}[H]\centering
\begin{tabular}{|l|}\hline
\textbf{Average Tweets/sek}	\\\hline
310.0 Tweets \\\hline
\end{tabular}
\caption{Average number of tweets retrieved per second}
\label{RetNumberAvg}
\end{table}

The results from this test show that we can retrieve a maximum of 310 tweets per
second. Considering the speed of processing these tweets, as shown in
\autoref{speedtestlavel}, this represents the greatest bottleneck for the
system.
 




\section{Stress Testing}
The purpose of this test is to ensure the system's ability to queue requests
from multiple users. This is important, as the system should be scalable such
that it can handle an undefined amount of users at a time. The test consists of
calling the code in \autoref{stressTestCode} 25 times. This code sends a request
to our Queue API at the address \textc{localhost:62020}.\nl

\begin{minipage}[H]{\linewidth}
\begin{lstlisting}[caption = Code initiating the stress test, label = stressTestCode] 
bool DatabaseSendDataRequest(params string[] parameters){
...
	HttpWebRequest request = (HttpWebRequest)WebRequest.Create("localhost:62020/API/AnalyzeTwitterAccount");
...
}
\end{lstlisting}
\end{minipage}

The results of this test was that the system was indeed capable of handling the
requests. This shows that the system is scalable in that it can handle an
undefined amount of requests from different people at a time.

\section{Black Box Testing}
\fix{}{Make the black box test}


