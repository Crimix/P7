\chapter{Testing}\label{cha:testing}
To determine the quality of the developed software solution, we have
performed a number of tests. These tests aim to assure the quality of the
determined results and to ensure that the software is operating as
expected.\nl

The tests were performed on a PC running Windows 10, an i5-7200U
3.1GHz processor, 256GB SSD, 8GB DDR4 RAM and a 100/100Mbps internet
connection.\nl

In \autoref{speedtestlavel}, we present a test which documents how many tweets
the Bag-of-Words and Naive Bayes Classifiers can classify per second. In
\autoref{sec:accTest}, we determine how accurate the two models are when
classifying a user's political leaning. In \autoref{test:tweetRetr}, we
determine how many tweets we can receive from the Twitter API per second. In
\autoref{test:stress}, we perform a stress test to determine if the system can
handle a large number of system users simultaneously.  

\section{Speed Tests}\label{speedtestlavel}
The purpose of these tests is to document the speed of the
classification models, discussed in \autoref{sec:BoW} and
\autoref{sec:NBImp}, using prefetched tweets. The first test is to test the
efficiency of multithreading the Bag-of-Words model while the second test is
used to compare the speed of the two models.

\subsection{Bag-of-Words Speed Test}\label{test:multithread}
The purpose of this test is to document the reason behind implementing a
multithreaded version of the classification method, which was described in
\autoref{subs:multithread}. 
The result of running both the multithreaded Bag-of-Words model and the
singlethreaded version on the same input can be seen in
\autoref{speedTestResSingleThread} and \autoref{speedTestResMultiThread}.

\begin{table}[H]\centering
\begin{tabular}{|l|l|}
\hline
3060	&	Tweets per batch 	\\\hline
500		&	Batches				\\\hline
17394	&	Milliseconds		\\\hline
34.7	&	ms/batch 			\\\hline
\end{tabular}
\caption{Results from the singlethreaded speed test.}
\label{speedTestResSingleThread}
\end{table}

\begin{table}[H]\centering
\begin{tabular}{|l|l|}
\hline
3060	&	Tweets per batch 	\\\hline
500		&	Batches				\\\hline
9928 	&	Milliseconds		\\\hline
19.8	&	ms/batch 			\\\hline
\end{tabular}
\caption{Results from the multithreaded speed test.}
\label{speedTestResMultiThread}
\end{table}



The results of the test shows that the time it takes to analyse tweets has been
cut 43\% by using the multi threaded version of the model. 

\subsection{Speed Test Comparison}
In this test the classification speed of both models will be compared. The
efficiency is evaluated based on the time necessary to process a given
number of tweets. Each approach will process 819.200 tweets in batches
of 3200. These numbers have been chosen as we need a high enough amount to
accurately determine the average time necessary and because each tweet
analysis will process a maximum of 3200 tweets, as this is the maximum amount
of tweets retrievable from a person.\nl

A subset of the results from the speed test can be seen in \autoref{table:stest}
together with the average tweets per second in \autoref{AvgTweetsPs}. The
full dataset can be found in \autoref{app:speedTest}.

\begin{table}[H]\centering
\begin{tabular}{|l|l|l|l|l|}
\hline
\textbf{Try No.:}	&	\textbf{BoW (ms):}	&	\textbf{BoW Tweet/s:}	&	\textbf{Bayes
(ms):}	& \textbf{Bayes Tweet/s:} \\\hline 
1	&	55	&	58181	&	493	&	6490 \\\hline
2	&	22	&	145454	&	353	&	9065 \\\hline
3	&	30	&	106666	&	334	&	9580 \\\hline
4	&	23	&	139130	&	328	&	9756 \\\hline		
\end{tabular}
\caption{Subset of the results from the speed test.}
\label{table:stest}
\end{table}

\begin{table}[H]\centering
\begin{tabular}{|l|l|l|}\hline
					&	\textbf{BoW Tweet/s:}	&	\textbf{Bayes Tweet/s:}	\\\hline
\textbf{Average:}	&	124007				&	9410 					\\\hline	
\end{tabular}
\caption{Average number of tweets processed per second.}
\label{AvgTweetsPs}
\end{table}

The results from this test show that the Bag-of-Words model is the fastest
approach, being capable of processing around 120.000 tweets per second, while
the Naive Bayes model is capable of processing around 9500 tweets per second.
To put this into perspective, if a user were to follow 500 accounts, it would
result in a maximum of 16.000.000 tweets, which would take around two minutes to
process for the Bag-Of-Words model and 26 minutes for the Naive Bayes model.
Considering that this amount of followed users is above average, we consider
this processing time as being acceptable.

\section{Accuracy Tests}\label{sec:accTest}
The purpose of these tests is to document the accuracy of both the
classification models. The first test is used to determine the accuracy of the
Naive Bayes model, while the second is used test the Bag-of-Words model, and
compare the predictions from two models.


\subsection{Naive Bayes Cross-Validation}\label{subsec:NBCV}
The purpose of this test is to determine the accuracy of the Naive Bayes model
by performing cross-validation on the training data. The idea is to split the
training sets into two parts, one for training and one for testing \citep[Ch.
7.5.2]{MIBook}. Due to time constraints, the cross-validation is done four times
with 90 labelled training tweets and 30 labelled testing tweets. These tweets have
been shuffled to avoid being in any particular order and the sets are not
balanced.\nl

Combining the results of the cross-validation, the following confusion
matrix is constructed, see \autoref{tab:ConfMat}. The matrix shows the
hit rate of the prediction compared to the labelled result
\citep{ConfusionMatrix}.

\begin{table}[H]
\centering
\begin{tabular}{l|l|l|l|}
\cline{2-4}
& Predicted Left & Predicted Neutral & Predicted Right \\ \hline 
\multicolumn{1}{|l|}{Labelled Left} & \textbf{25} & 9 & 3 \\ \hline
\multicolumn{1}{|l|}{Labelled Neutral} & 7 & \textbf{15} & 8  \\ \hline
\multicolumn{1}{|l|}{Labelled Right} & 6 & 7 & \textbf{40} \\ \hline
\end{tabular}
\caption{Confusion matrix for the combined results of the validation tests.}
\label{tab:ConfMat}
\end{table}

The accuracy of each of the validation tests can be seen in
\autoref{tab:CrossVal}. The table shows how many of the tweets were correctly
predicted, as well as the average accuracy.

\begin{table}[H]
\centering
\cc{
\begin{tabular}{|l|l|l|l|l|l|}
\hline
Cross Validation & Test 1 & Test 2 & Test 3 & Test 4 & Average \\ \hline
Accuracy &  20/30 = 0.67 & 18/30 = 0.60 & 21/30 = 0.70 & 21/30 = 0.70 & 80/120 =
0.67 
\\ \hline
\end{tabular}
}
\caption{Accuracy for each of the cross validations.}
\label{tab:CrossVal}
\end{table}

In addition, it is also of interest to look at the precision, which is how often
the prediction class is the correct class. Calculating the precision helps
measure if the accuracy is skewed towards a specific class. Using
\autoref{tab:ConfMat} the average precision is calculated in \autoref{e:PL},
\autoref{e:PN} and \autoref{e:PR}.

\begin{equation}\label{e:PL}
Precision(Left) = \cfrac{25}{25+7+6} = 0.66
\end{equation}
\begin{equation}\label{e:PN}
Precision(Neutral) = \cfrac{15}{9+15+7} = 0.48
\end{equation}
\begin{equation}\label{e:PR}
Precision(Right) = \cfrac{40}{3+8+40} = 0.78
\end{equation}\nl

Based on these tests, it is possible to see that the model is very bad at
predicting tweets belonging to the \textit{neutral} class, while it is far
better at predicting tweets belonging to the \textit{right} class. The accuracy
and precision could be improved by training on a larger set and ensuring the
validation sets are balanced.

\subsection{Accuracy Test Comparison}
The purpose of this test is to compare the accuracy of the two different
approaches to classification discussed in \autoref{sec:BoW} and
\autoref{sec:NBImp}. The accuracy is determined by comparing the results
from the classification with a set of manually pre-classified Twitter users.
The chosen users consist of politicians, activists and a set of average
Twitter users. A subset of the results from the accuracy test can be seen in
\autoref{speedTestReslabel} together with the percentage success in
\autoref{AccPercent}. The full dataset can be found in
\autoref{app:AccuracyTest}.\nl

A subset of the dataset from the accuracy test can be seen below in
\autoref{speedTestReslabel}. Each test shows one pre-classified user and the
results determined from the Bag-of-Words model and the Naive Bayes model. As
stated in \autoref{cha:classification}, the result consists of a value between
-10 and 10, where a low value implies left-leaning views, and a high value
implies right-leaning views. The Bag-of-Words model also has a conclusion, which is dependent on the
percentage of tweets considered political and the size of the determined value.
For the BoW model, a user with less than 1\% political tweets, or a determined
value between -1 and 1, is automatically considered as neutral. The percentage
accuracy of each model can be seen in \autoref{AccPercent}. The columns
represent the following, from left to right:\nl

\begin{itemize}
  \item The percentage of tweets the BoW model
consider political.
\item Numeric result from BoW model.
\item Conclusion from BoW model.
\item Numeric result from the Bayes model.
\item Conclusion from the Bayes model.
\item Actual political leaning of the analysed user.
\end{itemize}


\begin{table}[H]\centering
\cc{
\begin{tabular}{|l|l|l|l|l|l|}
\hline
\textbf{Pol\%:}	&	\textbf{Alg:}	&	\textbf{Alg Conc:}							&	\textbf{Bayes:}	&
\textbf{Bayes Conc:}
& \textbf{Actual:}
\\\hline 0.64	&	-2.42	&	Left/Neutral (Pol\% too low)		&	5.82	&	Right		&	Left	\\\hline
1.82	&	-0.865	&	Right								&	3.16	&	Right		&	Right	\\\hline
8.25	&	-3.219	&	Left								&	2.55	&	Right		&	Left	\\\hline
1.39	&	0.5		&	Right/Neutral (Result is 0 +/- 1)	&	3.51	&	Right		&	Right	\\\hline
\end{tabular}
}
\caption{Results from the accuracy test.}
\label{speedTestReslabel}
\end{table}




\begin{table}[H]\centering
\begin{tabular}{|l|l|}\hline
\textbf{Accuracy BoW}	&	\textbf{Accuracy Bayes}	\\\hline
96.67\%					&	43.33\%					\\\hline	
\end{tabular}
\caption{Accuracy of the two model.}
\label{AccPercent}
\end{table}

The results of the test show that the Bag-of-Words model is capable of
accurately classifying 96\% of the test users. The Naive Bayes model has
problems with classifying real users, as it has classified every user as being
biased towards the right. This has resulted in an accuracy of 43.33\%. As
such we consider the Bag-of-Words model to be the most successful until the
Naive Bayes model has received more training.

\section{Tweet Retrieval Test}\label{test:tweetRetr}
The purpose of this test is to document the limitations of retrieving tweets
through the Twitter API. This is important as the speed at which we can retrieve
tweets sets an upper limit to how fast we can present an output to the user.
The speed will be determined by calling the Twitter API from the worker class
and documenting the number of tweets we can receive over a predefined amount of
time. The \textc{Worker} class and tweet retrieval are discussed in
\autoref{workerLabel}.\nl

Each test shows the retrieval of tweets from five different users and the
result is the total amount of tweets retrieved per second from all five
concurrent threads. The fastest, slowest and the average tweets per second can be seen in
\autoref{tweetRetReslabel} while the full dataset can be found in
\autoref{app:tweetRet}.

\begin{table}[H]\centering
\begin{tabular}{|l|l|l|l|}
\hline
\textbf{} & \textbf{No.:}	&	\textbf{Time (ms):}	&	\textbf{Tweets/s:} \\\hline
Fastest & 13	&	7705	&	371.3\\\hline
Slowest & 9	&	13203	&	216.7\\\hline
Average & 	&	8905	&	310.0 \\\hline
\end{tabular}
\caption{Subset of the results from the tweet retrieval test.}
\label{tweetRetReslabel}
\end{table}

The results from this test show that we can retrieve an average of 310 tweets
per second. Considering the speed of processing these tweets, as shown in
\autoref{speedtestlavel}, this represents the greatest bottleneck for the
system.
 
\section{Stress Testing}\label{test:stress}
The purpose of this test is to ensure the system's ability to queue requests
from multiple users. This is important, as the system should be scalable such
that it can handle an undefined number of users at a time. The test consists of
calling the code in \autoref{stressTestCode} 25 times. This code sends a request
to our queue API at the address \textc{localhost:62020}.\nl

\begin{minipage}[H]{\linewidth}
\begin{lstlisting}[caption = Code initiating the stress test., label =
stressTestCode] 
bool DatabaseSendDataRequest(params string[] parameters){
...
	HttpWebRequest request = (HttpWebRequest)WebRequest.Create("localhost:62020/API/AnalyzeTwitterAccount");
...
}
\end{lstlisting}
\end{minipage}

The result of this test was that the system was indeed capable of handling
the requests. This shows that the system is scalable as it can handle an
undefined amount of requests from different people at a time.


% \section{Functional Testing}
% \fix{}{Jonathan - Consider removing this section, as it seems more like a user
% guide, and does nothing test worthy}
% 
% In order to assure that the system is working as indended, we have
% performed a number of functional tests, in which we supply an input to the system, and
% monitor that each individual part of the system performs as expected. In the
% first step, we enter the input into the frontend. This can be seen in
% \autoref{GUIfrontPage}.\nl 
% 
% After inputting the twitter name, we make a request to the database for any data
% relating to that name. If the data is present, we go directly to the result in
% \autoref{GUIresult1}. If the data is not present, we prepare to perform the
% tweet retrieval and analysis. As such, we request that the user authenticate using
% Twitter's site. This can be seen in \autoref{GUItwitterAuthTab}.\nl
% 
% After clicking the ``Godkend Applikation'' button, the user is given a unique
% password. This can be seen in \autoref{GUItwitterPin}.\nl
% 
% \figx{GUItwitterPin}{Unique password supplied by Twitter}
% 
% Back in our application, we request that the user enter this password
% together with an email address of their choosing. This can be seen in
% \autoref{GUIauthPopup}.\nl
% 
% 
% After the user inputs the password, we show the screen in \autoref{GUIdonePage}.
% Then the application makes a request to Twitter's API, and recieves the user's
% \textc{access\_token} and \textc{access\_token\_secret}. This information is
% then used to request tweets from all the accounts that the specified user is
% following. These tweets are then analyzed in relation to their sentiment
% (positivity/negativity), and their political leaning (left/right). The output
% log from the program can be seen in \autoref{tweetlogAnal}. \nl
% 
% \begin{minipage}[H]{\linewidth}
% \begin{lstlisting}[caption = Log from tweet retrieval and analysis, label = tweetlogAnal] 
% 12-12-2017 10:37:37 | Debug~ Task entered queue with ID ODotTWJtc3dBQUFBQUE...
% 12-12-2017 10:37:37 | Debug~ Task started ODotTWJtc3dBQUFBQUE...
% 12-12-2017 10:38:10 | Debug~ Donald J. Trump                25073877             2874       
% 12-12-2017 10:38:10 | Debug~ Spliting 2874 tweets
% 12-12-2017 10:38:10 | Debug~ Combining tweets
% 12-12-2017 10:38:10 | Debug~ Result 5,21428571428571
% 12-12-2017 10:38:13 | Debug~ Following 45 users
% ...
% 12-12-2017 10:52:09 | Debug~ Done task with id ODotTWJtc3dBQUFBQUE...!!!
% \end{lstlisting}
% \end{minipage}
% 
% From the log in \autoref{tweetlogAnal} we can see that the entire tweet
% retrieval and analysis process took 15 minutes. The reason for the delay is that
% each user has a limited amount of tweets per 15 minute period. As such, the
% threads responsible for retrieving tweets were put to sleep while waiting for
% more requests.\nl
% 
% When the system backend finishes processing the tweets, it returns the request
% from the frontend to tell it that a result is ready. The frontend then sends a
% mail to the user's specified email account. This email can be seen in
% \autoref{GUImail}, where the user must click the ``See Result" button.\nl
% 
% After clicking the button, the user will be redirected to a custom page on our
% website, which will present the results in various graphs. This can be seen in
% \autoref{GUIresult2}.\nl
% 
% Based on the results of this test, we can confirm that every part of the system
% is capable of interacting correctly with each other.





